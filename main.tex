%%%%%%%%%%%%%%%%%%%%%%%%% DISCLAIMER %%%%%%%%%%%%%%%%%%%%%%%%%
% Die Kommentare sind tatsächlich nur für mich - sollen mich aber nicht davon abhalten, das Ding hier zu veröffentlichen, nur weil die Qaulität des Codes sowohl amateurhaft, als auch null für die Veröffentlichung aufbereitet ist und die Sprache der Kommentare nicht veröffentlichungstauglich ist.
%%%%%%%%%%%%%%%%%%%%%%%%%%%%%%%%%%%%%%%%%%%%%%%
\documentclass[a4paper, 
    12pt, 
    ngerman, 
    listof=totoc,
    titlepage]{scrartcl} % a4paper, damit nicht amerik. letterpaper; 12pt Schriftgröße; ngerman Umlaute äöü oder sowas; titlepage damit keine Seitenzahl bei Titelseite, \subject etc. dazugeknallt werden können, hübscher isses auch; scratrcl soll deutsch und hübsch sein, hat aber keine chapter nur section, aber chapter fangen auch neue Seiten an https://texwelt.de/fragen/15869/seitennummer-in-scrartcl-auf-titelseite-unterdrucken
% hatte bis 2025-05-14 a4paper in [] nach springer nature vorlage habe ich da mal pdflatex reingeschrieben, scheint nix kaputt zu machen. Ich glaube KOMA script mit scrartcl sollte sich schon um DIN A4 kümmern

\usepackage[utf8]{inputenc} % moderner Zeichencode, auch mit ä ö ü und so'n Kram
\usepackage[T1]{fontenc} % font encoding. T1 is best for west european e.g. due to hyphenation - Schriftarten gute diese glaube ich
\usepackage{lmodern} % latin modern Serif Font based on the older Computer Modern - hübsche Schrift? hab's vergessen

\usepackage[ngerman]{babel} % Sprache, ngerman Silbentrennung nach neuer Rechtschreibung und so
% \usepackage{verbatim} % das habe ich gefunden um ganze Blöcke inklusive Zeilenumbrüchen auszukommentieren -- Nachtrag lol, bullshit, verbatim direkt druckt code ohne auszuführen
% \usepackage[normalem]{ulem} % zum Durchstreichen https://texwelt.de/fragen/3223/wie-kann-ich-text-durchstreichen ulem besser wegen äöüß und neuer. soul besser als ulem für Zeilenumbrüche, Zeilenumbrüche bei Links waren nicht existent mit ulem https://tex.stackexchange.com/questions/74893/strikeout-when-which-package-ulem-vs-soul-vs#comment710882_74910 ich versuche also ulem mit normalem als Option https://golatex.de/viewtopic.php?t=3079 prima, scheint zu klappen

% \usepackage{acronym} % added 2024-06-05 https://www.heise.de/tipps-tricks/Abkuerzungsverzeichnis-in-LaTeX-erstellen-4982473.html oder https://de.wikibooks.org/wiki/LaTeX-W%C3%B6rterbuch:_Abk%C3%BCrzungsverzeichnis wiki empfiehlt womöglich package glossaries, hab da gerade kein Bock zu, acronym tut's bestimmt auch https://www.overleaf.com/learn/latex/Glossaries
% printonlyused could be put in []
% [v2 used because compile time in overleaf free]
% I try \usepackage{glossaries} 2025-06-03
\usepackage{amsmath}

\usepackage[onehalfspacing]{setspace} % Zeilenanbstand, Grüße ans Kompendium
\usepackage{geometry} % Seitenränder, ein wahrer Krampf
\geometry{
    left=30mm, 
    right=30mm, 
    top=25mm, 
    bottom=25mm
    } % mm besser als cm, bestimmt wegen , und . als Dezimaltrennzeichen, der Krampf hat sich entspannt, mm rulez
\usepackage[
    autostyle=true, 
    german
    ]{csquotes} % für Anführungszeichen in Zitaten in versch. Sprachen, Schachtelung und so: https://www.namsu.de/Extra/pakete/Csquotes.html

\usepackage[
    backend=biber, 
    style=authoryear-icomp, 
    bibstyle=authoryear, 
    dashed=false, 
    maxcitenames=2, 
    maxbibnames=99, 
    block=none
    ]{biblatex} % Zitationsstil, authoryear kommt Harvard nah -- date=year von Antwort aus https://texwelt.de/fragen/27027/zitationsstil-anpassen-biblatex-authoryear
% 2025-04-22 block=nbpar wäre erklärt bei Seite 51 von der Anleitung 3.1.2.1 https://mirrors.ibiblio.org/CTAN/info/translations/biblatex/de/biblatex-de-Benutzerhandbuch.pdf

\addbibresource{references.bib} % Bibliographie erstellen
% wenn unter Zitation -> Latex Assistant F7 ausgegraut, dann: Extras > Optionen > Zitation, da LaTeX anklicken
% im Moment (2022-09-06) ist TeXStudio angegeben und: quote; autocite; cite; parencite & quote; autocite; cite; parencite
% inspiriert von: https://help.citavi.com/topic/latex-assistent-kopieren
% stil authoryear-icomp s. Handbuch Seite 77 https://mirrors.ibiblio.org/CTAN/info/translations/biblatex/de/biblatex-de-Benutzerhandbuch.pdf
% 2025-04.02 maxbibnames=99 aus [] von biblatex genommen, da standard wohl =maxnames ist, Anleitung S. 49... Ok, doch nicht, gibt dann Fehler. vierlleicht lieber =99, weil ich die Schrägstriche definiert habe? Und sorting=nyt etnfernt, sollte Standard sein. Und dashed=false hinzugefügt Anleitung zu bibtex S. 81 bei Bibliographiestile
% hier steht auch maxbibnames=99 ist beste: https://tex.stackexchange.com/questions/12806/guidelines-for-customizing-biblatex-styles

% erst Nachname, dann Vorname https://www.mrunix.de/forums/showthread.php?75128-Biblatex-Nachname-Vorname klappt nur leider auch mit family-given nicht 2025-04-02 \DeclareNameAlias{default}{family-given}
% 2025-04-02 ohne ist nur beim ersten in der Bibliography der Nachname zu Beginn, danach ist es Vorname Nachname 

% https://texwelt.de/fragen/29060/biblatex-nachname-vorname default durch author (und editor und translator) ersetzt
\DeclareNameAlias{author}{family-given} % 2025-05-28 scheint nur author zu ändern, bei editor bleibts, daher: 
\DeclareNameAlias{editor}{family-given}
% \DeclareNameAlias{editor}{sortname}
% \DeclareNameAlias{translator}{sortname}

%%%%%%%%%%%%%%%% %%%%%%%%%%%%%%%% %%%%%%%%%%%%%%%% 
%%% von copilot für prompt: 
% wie verändere ich textcite in biblatex so, dass alle namen mit komma und und aufgezählt werden?
% \DeclareDelimFormat{andothersdelim}{\addcomma\space}
% \DeclareDelimFormat{finalnamedelim}{\space und\space}
% \DeclareDelimFormat{multinamedelim}{\addcomma\space}
%scheint aber nix zu ändern 2025-05-28

%%% von copilot für prompt: 
% das hat noch nicht geklappt. ich nuze folgenden stil: \usepackage[backend=biber, style=authoryear-icomp, bibstyle=authoryear, dashed=false, maxcitenames=2, maxbibnames=99]{biblatex}
% \DeclareDelimFormat{finalnamedelim}{\space und\space}
% \DeclareDelimFormat{multinamedelim}{\addcomma\space}
%scheint aber nix zu ändern 2025-05-28
%%%%%%%%%%%%%%%% %%%%%%%%%%%%%%%% %%%%%%%%%%%%%%%% 

\DefineBibliographyStrings{german}{
   andothers = {{et\,al\adddot}},
} % damit wird u.a. zu et al. siehe https://tex.stackexchange.com/questions/236854/changing-u-a-of-german-to-et-al

% no p. pp. and : instead of ,
% https://stackoverflow.com/questions/58448860/how-can-i-add-colons-and-remove-p-pp-from-in-text-citations-with-biblatex
\DefineBibliographyStrings{german}{
  page             = {},
  pages            = {},
}  % ich habe german anstatt english ergänzt % 2025-05-27 auf ngmerman geändert
% colon instead of comma bei in text cite
\renewcommand{\postnotedelim}{\addcolon \addspace} % \addspace kommt von mir
%%%%% 2025-06-30 I try german instead of ngerman -- doesnt solve warning, but works as well %%%%%%%%%%%%%%%

% no p.von anderer Quelle
% \DeclareFieldFormat{postnote}{#1}
% \DeclareFieldFormat{multipostnote}{#1}

\renewcommand*{\multinamedelim}{\addslash}
\renewcommand*{\finalnamedelim}{\multinamedelim}
% Schrägstriche zur Abtrennung https://golatex.de/viewtopic.php?t=18487 andere Quelle (hat nicht auf Anhieb funktioniert) https://groups.google.com/g/de.comp.text.tex/c/eSVMHAMphCA

%%%%%%% danke copilot
% Spezielle Anpassung für textcite
\DeclareDelimFormat[textcite]{finalnamedelim}{\space und\space}
\DeclareDelimFormat[textcite]{multinamedelim}{\addcomma\space}

% S. im Literaturverzeichnis wäre nett. da stehen hundert Zahlen hintereinander
%\renewbibmacro*{volume+number+eid}{%
%  \printfield{volume}%
%  \iffieldundef{volume}{}{~Aufl.}%
%  \setunit*{\addcomma\space}%
%  \printfield{number}%
%  \setunit*{\addcomma\space}%
%  \printfield{eid}}

%\renewbibmacro*{pages}{%
%  \iffieldundef{pages}{}{S.\space\printfield{pages}}}
%%%%%%%%%%%%%

% doppelte Nachnamen wegen shorthands (auto. durch citavi erstellt) in den refenerces.bib https://golatex.de/viewtopic.php?t=13563
% keine KURZBELEGE in CITAVI für Harvard Zit. verwenden!! https://www1.citavi.com/sub/manual6/de/index.html?cse_customizing_citation_keys.html

% \usepackage{minitoc} %h Mini Inhaltsverzeichnis https://cs.brown.edu/about/system/managed/latex/doc/minitoc.pdf
% \usepackage{pdfpages} % vielleicht mal pdfs direkt hier als anhang einbinden https://texwelt.de/fragen/19153/pdf-dateien-in-anhang-einbinden-aber-formatierung-des-vorhergehenden-dokuments-ubernehmen
% \usepackage{subfiles} % Best loaded last in the preamble https://de.overleaf.com/learn/latex/Multi-file_LaTeX_projects%23The_subfiles_package


%%%%%%%%%%%%%%%%%%%%%%%%%%%%%%%%%%%%%%%%%%%%%%%%%%%%%%%%%%
\newenvironment{myitemize} % https://tex.stackexchange.com/questions/10684/vertical-space-in-lists 2025-06-29
{ \begin{itemize}
    \setlength{\itemsep}{0pt}
    \setlength{\parskip}{0pt}
    \setlength{\parsep}{0pt}
    % \setlength{\baselineskip}{0pt}
    % \setlength{\topsep}{2pt}    % space before and after the list
    % \setlength{\partopsep}{0pt} % extra space added to \topsep when environment starts a new paragraph
 }
{ \end{itemize} } 

\newenvironment{myenumerate}
{ \begin{enumerate}
    \setlength{\itemsep}{0pt}
    \setlength{\parskip}{0pt}
    \setlength{\parsep}{0pt}
    % \setlength{\baselineskip}{0pt}
 }
{ \end{enumerate}                } 
%%%%%%%%%%%%%%%%%%%%%%%%%%%%%%%%%%%%%%%%%%%%%%%%%%%%%%%%%%%%%%

%%%%%%%%%%%%%%%%%%%%%%%%%%%%%%%%%%%%%%%%%%%%%%%%%%%%
\usepackage{graphicx}   % PDF Support % vorher war hier auch [pdftex] eingebunden. Nachdem ich hyperref in der Präambel nach unten geschoben habe, hat overleaf aber einen Fehler ausgespuckt
\usepackage{float} % addded 2025-06-10
% hier war mal \usepackage[hidelinks]{hyperref} % Hyperlinks % hidelinks keine bunten Kasten um klickbare Elemente: https://texwelt.de/fragen/1121/wie-entferne-ich-die-roten-rahmen-um-hyperlinks? 2025-05-14 ist jetzt nach unten gerutscht 2025-06-19
% \usepackage{times}              % Font fuer PDF besser % 2025-07-01 vielleicht doch lieber beim Standard bleiben: https://tex.stackexchange.com/questions/182906/ugly-usepackagetimes-examples % außerdem habe ich oben schon \usepackage{lmodern} eingebunden
\usepackage{thumbpdf}           % PDF Thumbnails erstellen
% https://www.linux-community.de/ausgaben/linuxuser/2005/04/mit-pdflatex-bessere-pdf-dateien-erzeugen/2/  
% 2025-05-14
%%%%%%%%%%%%%%%%%%%%%%%%%%%%%%%%%%%%%%%%%%%%%%%%%%%%

%%%%%%%%%% TABELLEN %%%%%%%%%%%%%%%%%%%%%%%%%
\usepackage{booktabs}
\usepackage{array}
% \usepackage{tabularx} % columns with X behave better
% \usepackage{longtable} % tables over several pages are possible
% \usepackage{ltablex} % should combine usepackage{tabularx} and \usepackage{longtable}
\usepackage{xcolor} % several colors available, needed for nice color options in \usepackage{tabularray} e.g.
\usepackage{tabularray} % this does everything tabularx and longtable can and even more, requires LaTeX3 (which is included in TeXLive e.g.)
%%%%%%%%%%%%%%%%%%%%%%%%%%%%%%%%%%%%%%%%%%%%%

%%%%%%%%%% LANDSCAPE %%%%%%% UND PDF EINBINDEN %%%%%%%
\usepackage{pdflscape} % I think I need this to have pdf pages in landscape orientation, to fit more pages on on a4paper
\usepackage[]{pdfpages} % very nice options to insert pdf once you understand them

\usepackage[
    xindy, 
    toc, 
    acronym,
    nomain, % ergänzt nach file main.glo which said main.gls is empty 
    nopostdot,
    nogroupskip
    ]{glossaries}
% xindy weil: https://tex.stackexchange.com/questions/199211/differences-between-xindy-and-makeindex/199841#199841
% https://en.wikibooks.org/wiki/LaTeX/Glossary#cite_note-2

    \newacronym{abb}{Abb.}{Abbildung}
    \newacronym{abs}{Abs.}{Abschnitt}
    \newacronym{ank}{ANK}{Arbeitnehmerkammer Bremen}
    \newacronym{arbschg}{ArbSchG}{Arbeitsschutzgesetz}
    \newacronym{bap}{bap}{Bundesausschuss Politische Bildung e.\,V.}
    \newacronym{bbk}{BBK}{Beutelsbacher Konsens}
    \newacronym{bgbl}{BGBl}{Bundesgesetzblatt}
    \newacronym{bmas}{BMAS}{Bundesministerium für Arbeit und Soziales}
    \newacronym{bpb}{bpb}{Bundeszentrale für politische Bildung}
    \newacronym{bspw}{bspw.}{beispielsweise}
    \newacronym{bzw}{bzw.}{beziehungsweise}
    \newacronym{c}{\copyright}{Copyright}
    \newacronym{diw}{DIW}{Deutsches Institut für Wirtschaftsforschung e.\,V.}
    \newacronym{doi}{DOI}{Digital Object Identifier}
    \newacronym{ebd}{ebd.}{ebenda}
    \newacronym{etc}{etc.}{et cetera}
    \newacronym{eu}{EU}{Europäische Union}
    \newacronym{gp}{GP oder GuP}{Gesellschaft und Politik}
    \newacronym{gpje}{GPJE}{Gesellschaft für Politikdidaktik und politische Jugend- und Erwachsenenbildung}
    \newacronym{idr}{i.\,d.\,R.}{in der Regel}
    \newacronym{ki}{KI}{Künstliche Intelligenz}
    \newacronym{kmk}{KMK}{Kultusministerkonferenz}
    \newacronym{lis}{LIS}{Landesinstitut für Schule}
    \newacronym{llm}{LLM}{Large Language Model}
    \newacronym{me}{m.\,E.}{meines Erachtens}
    \newacronym{nw}{NW}{Naturwissenschaften}
    \newacronym{ocr}{OCR}{Optical Character Recognition}
    \newacronym{oä}{o.\,Ä.}{oder Ähnliche(s)}
    \newacronym{s}{s.}{siehe}
    \newacronym{S}{S.}{Seite}
    \newacronym{sgb}{SGB}{Sozialgesetzbuch}
    \newacronym{sek}{Sek.}{Sekundarstufe}
    \newacronym{suub}{SuUB}{Staats- und Universitätsbibliothek Bremen}
    \newacronym{sus}{SuS}{Schülerinnen und Schüler}
    \newacronym{ua}{u.\,A.}{unter Anderem}
    \newacronym{url}{URL}{Universal Ressource Locator }% (einheitlicher Ressourcen Verorter, Webadressen)
    \newacronym{uu}{u.\,U.}{unter Umständen}
    \newacronym{vgl}{vgl.}{vergleiche}
    \newacronym{wuk}{WUK}{Welt-Umweltkunde}
    \newacronym{zap}{zap}{Zentrum für Arbeit und Politik}
    \newacronym{zb}{z.\,B.}{zum Beispiel}
    \newacronym{zit}{zit.}{zitiert}
    \newacronym{zt}{z.\,T.}{zum Teil}

\makeglossaries % das makeglossaries muss vor \begin{document} stehen, damit es funktioniert
% open Console in Visual Studio Code -- STRG+J click Console -- type: makeglossaries main (main being main.tex without extension) if no glossary with abbreviations appears

\usepackage[hidelinks=true,
            pdftitle=Chancen und Risiken von Fremdmaterial_Eine beispielhafte Analyse und Wertung von Unterrichtsmaterial der Arbeitnehmerkammer Bremen für den Politikunterricht an berufsbildenden Schulen in Bremen,
            pdfauthor=Paul Aljsocha Klein,
            pdfsubject=Masterarbeit Universität Bremen,
            pdfcreator=Paul Aljsocha Klein with LaTeX,
            pdfcreationdate=16.07.2025]{hyperref} % Hyperlinks % hidelinks keine bunten Kasten um klickbare Elemente. Das ist rausgenommen, weil keine Leerzeile unter Literatureinträgen: , backref=true, pagebackref=true
\begin{document}
\titlehead{
    \includegraphics[width=0.5 \linewidth]{
    UniBremen_Logo_Rot-Schwarz_Web_NICHT_VERAENDERN.png
    }
}

\subject{
    Masterarbeit
    }

\title{
    Chancen und Risiken von Fremdmaterial
    } 

\subtitle{
    Eine beispielhafte Analyse und Wertung von Unterrichtsmaterial der Arbeitnehmerkammer Bremen für den Politikunterricht an berufsbildenden Schulen in Bremen
    }

\author{
    Paul Aljoscha Klein\\ 
    \normalsize
    $\ast$ 20.07.1995\\ 
    \normalsize
    $\#$ 4134166
    }

\date{Abgabe: 08. Juli 2025}

\publishers{
    \vfill % doesn't seem to help
    \raggedright{
    Erstprüfer: Prof. Dr. Andreas Klee\\
    Zweitprüferin: Dr. Eva Anslinger
    } 
    \hfill % bigger graphic-width inserts bigger vertical space between the two lines with the names. hfill doesn't really help % hfill seems unnecassary due to raggedleft, but without, the second name line gets pulled right
    \raggedleft{
    \includegraphics[width=0.2 \linewidth]{
    zap_Logo_Website.001.png % Bildquelle: https://www.uni-bremen.de/fileadmin/user_upload/sites/soha/zap_Logo_Website.001.png 2025-06-19
    }
    } % raggedleft is needed in this setup
    }


% So sieht's gut aus mit titlepage bei Dokumentklasse scartcl, Danke Markus Kohm

%%%%%%%%%%%%% TitelBrainStorming %%%%%%%%%%%%%%%%%%%%%%%%%
% Nutzung von fremden Unterrichtsmaterialien. Eine exemplarische Analyse für den Politikunterricht an berufsbildenden Schulen in Bremen
% Nutzung von Unterrichtsmaterialien einer externen Institution. Eine exemplarische Analyse für den Politikunterricht an berufsbildenden Schulen in Bremen

% Entwürfe 2025-03-04 Di.
% Nutzung von fremden Unterrichtsmaterialien. Eine exemplarische Analyse für den Politikunterricht an berufsbildenden Schulen in Bremen

% Analyse und Wertung von großflächig angelegtem Unterrichtsmaterial Materialvorlagen: Exemplarisches Vorgehen mit Material der Arbeitnehmerkammer Bremen.

% Analyse und Wertung von Materialvorlagen für den Politikunterricht an berufsbildenden Schulen in Bremen: Eine beispielhafte Analyse mit Material der Arbeitnehmerkammer Bremen.

% Gekürzt:
% Analyse und Wertung von Materialvorlagen: Eine beispielhafte Analyse anhand von Unterrichtsmaterial für den Politikunterricht an berufsbildenden Schulen in Bremen der Arbeitnehmerkammer Bremen.

% Die Realität der Nutzung von institutionellem Unterrichtsmaterial entgegen selbsterstelltem Material: Eine beispielhafte Analyse anhand von Unterrichtsmaterial der Arbeitnehmerkammer Bremen für den Politikunterricht an berufsbildenden Schulen in Bremen.

% Nur selbsterstelltes Unterrichtsmaterial? Chancen und Risiken von Fremdmaterial: Eine beispielhafte Analyse und Wertung von Unterrichtsmaterial der Arbeitnehmerkammer Bremen für den Politikunterricht an berufsbildenden Schulen in Bremen.

% Unterricht von der Stange? Chancen und Risiken von Fremdmaterial: Eine beispielhafte Analyse und Wertung von Unterrichtsmaterial der Arbeitnehmerkammer Bremen für den Politikunterricht an berufsbildenden Schulen in Bremen

% Kriterien zur Bewertung von externen Unterrichtsmaterialien im Politikunterricht. Am Beispiel 
% Geschenke nimmt man gerne an
% Unterricht von der Stange
% Erleichterung durch fertiges Unterrichtsmaterial
%%%%%%%%%%%%%%%%%%%%%%%%%%%%%%%%%%%%%%%%%%%%%%%%%%%%%%%%%%%%%%%%%
\maketitle
\clearpage

\pagenumbering{roman}
\addsec{Abstract}
blub
Bre steht das hier auch in vsc?

\clearpage
\newpage

\tableofcontents % war früher mal vor pagenumbering. Macht das einen Unterschied? Keine AHnung 2024-06-05 ah ja, will ja nicht das Inhaltsverzeichnis Seitenummeriert haben 2024-06-06

% \clearpage
% \newpage

\printacronyms[type=\acronymtype, nonumberlist]{} % Acronymtype ist von glossaries, title und toctitle sind Titel im Inhaltsverzeichnis
% listof=totoc in \documentclass[] replaces %% , title={Abkürzungsverzeichnis}, toctitle={Abkürzungsverzeichnis} %% here
\vspace{24pt} % vertical space

\listoffigures % Abbildungsverzeichnis
\listoftables % Tabellenverzeichnis

\clearpage
\newpage

\setcounter{page}{1}
\pagenumbering{arabic}
\section{Einleitung \& Theoretischer Hintergrund}
% Unterrichtsmaterial kommt nicht nur aus Schulbüchern oder wird von den Lehrkräften selbst erstellt
Die Ansprüche an Schulbildung und an politische Bildung sind seit jeher Bestandteil reger Diskurse. Verschiedene Einflüsse und eine Welt, die sich rapide verändert, schaffen eine herausfordernde Lage. In der Unterrichtspraxis stehen Lehrkräfte hingegen vor alltäglichen Entscheidungen und Aufgaben. Eine davon ist die Auswahl von Themen und die Vorbereitung von entsprechendem Material, um damit einen guten Unterricht zu planen. 

Die Auswahl des Materials ist nicht beliebig und bedingt die Planung für die Durchführung einer gelungenen Unterrichtsstunde. Da das Ziel ist, dass alle Schüler*innen bestimmte Kompetenzen ausbilden, bietet es sich an, Material nicht für jede Unterrichtsstunde von Grund auf neu zu erstellen. 
Fremdes Material bietet dafür Chancen, bringt aber auch Risiken mit sich. In dieser Arbeit sollen einige dieser Chancen und Risiken exploriert werden. 
% wie der Unterricht gelungen durchgeführt werden kann

Ein Fokus wird dabei auf den Umgang mit Quellenarbeit gelegt. Es wird argumentiert, weshalb der Umgang mit der Herkunft von Wissen eng mit dem Begriff \emph{Wahrheit} verknüpft ist. Die gesamtgesellschaftliche Relevanz von Wissensproduktion und Verbreitung und damit auch die Relevanz für die Bildung, sind die Hintergründe für die gelegten Foki.
\bigskip

Die \gls{ank} Bremen plant für den schulischen Politik-Unterricht Materialkarten mit fertig geplantem Unterricht zu veröffentlichen.
Beispielhaft werden zwei ausgewählte Handlungsfelder analysiert und gewertet. In den Materialkarten werden in weiten Teilen Materialien der \gls{ank} als Grundlage für die Schüler*innen-Aufgaben angeboten. Besondere Aufmerksamkeit findet der Umgang mit Quellen in diesem Material und was die Aufgaben daraus machen. 
Da der erfolgreiche Einsatz von Quellenmaterial, von der Einbettung in Aufgabenstellung und Unterricht abhängt, wird auch grundsätzlich untersucht, welche Möglichkeiten und Grenzen in dem Angebot zu finden sind. 

% Aspekte von \enquote{gutem} und \enquote{schlechtem} Unterrichtsmaterial sich darin erkennen lassen. 
% Gutes Material erhöht die Wahrscheinlichkeit für guten Unterricht


% \section{Theoretischer Hintergrund / Begriffsklärungen}
% Um das Material zu analysieren, hilft es zunächst, mehrere Begriffe zu klären. Zum ersten eine Definition, was unter Unterrichtsmaterial verstanden werden soll und.... % Zum zweiten ein Überblick, wie politische Bildung an Berufsschulen in Bremen stattfindet. 
% Dann sollen der Bildungsplan in Politik vorgestellt werden, sowie der \gls{bbk}, auf den jener Bezug nimmt. Um im späteren Verlauf die verschiedenen Lebensweltbezüge aufzeigen zu können, soll daraufhin noch die Zielgruppe der Lernenden umrissen werden. 
% Blub blub, nur ausführen was die Überschriften weshalb sagen........ JA, DER BUMS IST UNABGESCHLOSSEN


\subsection{Von der Theorie in die Praxis \label{theorie in praxis}}
Es gibt extensive Forschung in den Erziehungswissenschaften und der Fachdidaktik. Auch der Bereich der politischen Bildung schlägt zahlreiche interdisziplinäre Brücken in die Sozialwissenschaften und über einen schulischen Bildungsbegriff hinaus. Es ist damit ein Forschungsfeld, welches in seiner Gesamtheit nicht einmal von Menschen zu überblicken ist, die ihr ganzes (Arbeits-)Leben der politischen Bildung gewidmet haben. 
Gleichzeitig sind sowohl die Möglichkeiten zur weiteren Erforschung, als auch der konkrete Forschungsbedarf im Detail von nicht zu überblickender Größe. 
% explizit einschließt, dass mindestens genauso viel unerforscht ist und vieles sicherlich sinnvoll wäre zu erforschen. 

In unserer hochkomplexen Welt gilt dies sicherlich für das Gros an Forschungsbereichen. Die Eingrenzung und damit Abgrenzung von Forschungsgegenständen ist daher stets zwangsläufiger Bestandteil des Forschens. In der Praxis bedeutet das, immer wieder aushandeln zu müssen, inwieweit Querverbindungen zu anderen Bereichen notwendig sind und an welcher Stelle der Fokus wieder auf das Wesentliche zu legen ist.
Diese Entscheidungen sind optimalerweise nachvollziehbar zu begründen. In der Forschung ergibt sich das bestenfalls aus der Sache. Der Versuch eine ursächliche Variable zu isolieren, um ihre Erklärungskraft für ein zu isolierendes Phänomen messbar zu machen, ist häufig der Modus des wissenschaftlichen Arbeitens. Wenn \gls{bspw} der Einfluss eines bestimmten Proteins auf die Wahrscheinlichkeit Krebs zu bekommen untersucht wird, sind zahlreiche Entscheidungsmöglichkeiten aus der Eingrenzung des Untersuchungsgegenstandes natürlicherweise ausgeschlossen worden. Genau wie wenn in der Statistik Einkommens- und Vermögensverhältnisse mitabgefragt werden, um ihre Verbindung zum Wahlverhalten berücksichtigen zu können. Wenn gleichzeitig Bildungsstand, Alter, Geschlecht, Wohnort \gls{etc} abgefragt werden, dient das auch der Notwendigkeit, in einer komplexen Welt, nur näher an die Wahrheit zu kommen, indem Stück für Stück viele Variablen untersucht werden. Trotz der angesprochenen Isolierung, nimmt fast nie nur eine Variable Einfluss auf das Ergebnis. Das führt näher zu dem inhärenten Entscheidungen treffen. Um die Wechselwirkung von Variablen untersuchen zu können, ist es notwendig viel zu forschen. Weshalb nun genau dieses Protein untersucht wird 
oder für die Statistik nicht auch die Körpergröße abgefragt wird\footnote{
    Die damit aufgeworfene Frage nach Kausalität wird nicht weiter aufgegriffen. Körpergröße hat womöglich wenig direkten Kausalzusammenhang zum Wahlverhalten, korreliert aber \gls{bspw} mit dem Geschlecht. In diesem kann durchaus ein Kausalzusammenhang mit dem Wahlverhalten vermutet werden. Darüber hinaus lässt sich durchaus vermuten, dass die Körpergröße \gls{zb} einen Einfluss auf den Erfolg von Karrieren nimmt, was wiederum weitere Variablen, wie das Einkommen beeinflussen dürfte. Sollte jetzt also immer die Körpergröße mitabgefragt werden? Nun, genau daran zeigt sich, dass begrenzte Ressourcen Entscheidungen erfordern, welche Aspekte am erfolgsversprechenden sind und welche daher erstmal ausgeklammert werden müssen. 
} ist dann immer noch eine Entscheidung. 


Analog zu den Entscheidungen, die Forschende treffen müssen, um handlungsfähig zu sein, das heißt fokussiert ihre Forschung vorantreiben zu können, ist die begründete Eingrenzung eines Lerngegenstandes ebenfalls mit Entscheidungen verbunden. 
Lehrende befinden sich diesbezüglich also in einer ähnlichen Situation wie Forschende. 
Es ließe sich allerdings argumentieren, dass die Notwendigkeit im Unterrichten von Weltwissen Entscheidungen zu treffen -- wie es im Schulkontext stattfindet -- auf einer vageren Datengrundlage stattfindet. In der Forschung wird sich wegen der zuvor beschriebenen Notwendigkeit der Eingrenzung zunehmend spezialisiert. Im schulischen Weltverstehen-Lernen hingegen ist in weiten Teilen (noch) kein hochspezialisiertes Verständnis gefordert. Stattdessen steht ein allgemeines Zusammenhänge-verstehen-Lernen im Vordergrund. Das setzt Lernen an exemplarischen Lerngegenständen voraus. 
Dies gilt insbesondere für die politische Bildung -- nicht nur in allgemeinbildenden Schulen, sondern auch für Politikunterricht in der dualen Berufsbildung \autocite[vgl. \gls{abs} \ref{bplan}, \gls{S} \pageref{bplan} \&][4; 9-13]{bplan}.
In der Ausbildung zum Lehramt wird Wert darauf gelegt, Entscheidungen darüber, welche Lerngegenstände ausgewählt werden, umfassend begründen zu können. 
In der Praxis des Lehrens wird Wert darauf gelegt, tatsächlich zu unterrichten.
In der wissenschaftlichen Literatur wurde dieses ständige im Widerspruch stehen, dem Lehrkräfte an verschiedenen Stellen ausgesetzt sind, erkannt und als Antinomie bezeichnet \autocite[\gls{vgl} \gls{zb}][]{Helsper.2001}. 

Das Treffen von Entscheidungen und die Konsequenzen von schlechter werdenden Entscheidungen durch Erschöpfung ist für Proffesionelle im medizinischen Bereich unter dem Begriff \emph{decision fatigue} seit längerem Forschungsgegenstand \autocite{Maier.2025}. Einen guten Überblick über \emph{decision fatigue} in verschiedenen anderen Bereichen gibt ein Artikel aus dem New York Times Magazine \autocite{Tierny.08.08.2011}. 
Ich würde argumentieren, dass auch Lehrberufe zu den Professionen gehören, bei denen Entscheidungen, sowohl in der Vorbereitung als auch im Unterrichtsgeschehen, eine entscheidende Rolle einnehmen. % lol, entscheidende
Ähnlich wie Konzepte der \emph{Handlungsfähigkeit} ebenfalls Relevanz für Pflege- und für Lehrprofessionen haben. 

Den kompletten Wissensstand der unterrichteten Domäne von einzelnen Lehrkräften fachwissenschaftlich oder didaktisch im kompletten Überblick zu behalten, ohne dass Teile des Alltagsgeschäfts hintenüber fallen, erschiene als eine utopische Forderung. Darin liegt das Potential in der Verbreitung von Unterrichtsmaterial. Viele Entscheidungen und deren bereits erarbeiteten Begründungen lassen sich an diversen Stellen sinnvoll einsetzen. Aufgrund der Arbeit, die in solche Entscheidungsprozesse (in den Erziehungs- und Bildungswissenschaften lässt es sich das als \emph{didaktische Reduktion} bezeichnen) einfließen können, bedeutet das auch, dass gesparte Ressourcen anderweitig für den Unterricht frei werden können. 

% erscheint die Benutzung von vorgefertigtem Unterrichtsmaterial eine praktikable Erleichterung. 

% HIER NOCH MEHR BRÜCKE ZU ARBEITSERLEICHTERUNG SCHLAGEN. MATERIALERSTELLUNG UNGLEICH UNTERRICHTEN

Entsprechend ist es eine willkommene Abwechslung, in dieser Arbeit zwar nicht zur Erleichterung des Handwerks des Unterrichtens beizutragen, wohl aber eine Initiative zum Untersuchungsgegenstand zu haben, die womöglich mehr Ressourcen dafür frei werden lässt. 
% der diese Intention unterstellt werden kann. 

% Getroffen werden müssen sie dennoch. Das ist sicherlich mit Arbeit verbunden.


\subsection{Forschungsfrage}
In dieser Arbeit soll diskutiert werden \dots % exploriert
\begin{itemize}
    \item \dots welche Maßstäbe zur Bewertung von Unterrichtsmaterial angelegt werden könnten. 
    \item \dots weshalb gerade diese Maßstäbe relevant sind. 
    \item \dots wie die Metaebene um das eigentliche Unterrichtsmaterial herum den Erfolg und die Verbreitung von Unterrichtsmaterial beeinflusst.
    \item[] Um sich ausgehend von den theoretischen Überlegungen der Praxis anzunähern, soll dazu beispielhaft darauf eingegangen werden \dots
    \item \dots wie die Qaulität des Unterrichtsmaterials, welches bei der \acrlong{ank} veröffentlicht werden soll, bewertet werden könnte. Das schließt eine Überprüfung auf mögliche Fehler und Ungenauigkeiten, sowie das Anschneiden von möglichen Alternativen ein. 
\end{itemize}

% Wie lässt sich das Unterrichtsmaterial der Arbeitnehmerkammer Bremen bewerten?

% Wie und nach welchen Maßstäben lässt sich das Unterrichtsmaterial der Arbeitnehmerkammer Bremen bewerten?


\subsection{Lerngegenstände \& Unterrichtsmaterial}
Unterrichten an Bildungsinstitutionen passiert hauptsächlich unter Zuhilfenahme von Medien im engeren Sinne. Die Planung ist im besten Fall verschriftlicht oder anderweitig aufgezeichnet, \gls{zb} durch Grafiken. Die Planung des Unterrichts und die Auswahl oder Erstellung der zum Unterrichten verwendeten Materialien beanspruchen dabei eine Menge Ressourcender Lehrkräfte\footnote{
    Am maßgeblichsten ist vermutlich Zeit. Aber wie in \gls{abs} \ref{theorie in praxis} (\gls{S} \pageref{theorie in praxis}) aufgeworfen wurde, können auch Dinge wie Entscheidungsfähigkeit und in Verbindung damit Aufmerksamkeit \gls{etc}, als endliche Ressource betrachtet werden. 
}. % nehmen dabei einen erheblichen Zeitaufwand seitens der Lehrkräfte ein.

Das Zurückgreifen auf bestehende Medien ist dabei natürlicher Bestandteil des Lernens. So ist die (Primär-)Quellenanalyse im Geschichtsunterricht, das Lesen eines Haikus in Deutsch, die Analyse einer Inszenierung, das Basteln an einem Motor oder das Erhitzen von Chemikalien die natürliche Benutzung von Unterrichtsmaterial. 

Die \gls{kmk} definiert \emph{Lehr}mittel als Materialen der Schule und \emph{Lern}mittel als Materialien der Schüler*innen \autocite{KMKMittel}. Gerade im Hinblick auf Verbrauchsmaterial und dessen Finanzierung ist das bestimmt eine sinnvolle Unterscheidungsdimension. Papier und Stifte stammen in der Regel von Schüler*innen, die abbrechende Bohrerspitze im Werkunterricht und % die Objektträger zur Mikroskopie 
das Gas zur Erhitzung der Chemikalie im Naturwissenschaftsunterricht, sind häufig aus der Schule oder werden auch mal aus den privaten Ressourcen der Lehrkräfte finanziert. Herausgestellt werden soll, dass Lernen in weiten Teilen in Verbindung mit der dinglichen Welt stattfindet.

Um diversen Gegebenheiten gerecht zu werden, ist das Material jedoch häufig abstrahiert. Eine Dampfmaschine steht womöglich nicht zur Verfügung. In so einem Fall kann \gls{zb} auf eine Explosionszeichnung, einen Film, ein Modell oder eine Beschreibung zurückgegriffen werden. Eine Beschreibung in Worten erfordert dabei eine höhere Abstraktionsebene als ein Modell und birgt die Gefahr, schlechtere Lernergebnisse zu erzielen. Das sollte dann durch entsprechende Einbettung oder die Ergänzung weiterer Materialien, wie einem Schema oder einem Film, versucht werden zu kompensieren. Wie anhand dieses Beispiels bereits deutlich wird, erfordert bereits die Auswahl der Lehrmaterialien Arbeit, bevor überhaupt mit den Lernenden in Interaktion getreten wird. 
Wenn nun in der politischen Bildung eine Rede analysiert wird, sind die naheliegenden Möglichkeiten überschaubar. Eine Verschriftlichung und eine Videoaufnahme bieten sich an. 

Gerade in der politischen Bildung liegt der Lerngegenstand jedoch häufig auf noch ferner liegenden Abstraktionsebenen. Es geht um hochkomplexe Konzepte (wie \gls{zb} \emph{Demokratie}), die in der physischen Welt schwer greifbar sind. Entsprechend viele nah- und fernliegende Möglichkeiten bestehen bei der Auswahl an Unterrichtsmaterial\footnote{Das soll nicht bedeuten, dass die Vorbereitungen im Mathematikunterricht zum ergänzenden Material einer eindeutigen Rechenaufgabe, \gls{zb} durch kleinschrittiges Vorgehen, grafische Darstellung oder Darstellungen in der physischen Welt, nicht ebenfalls hochkomplex sind und zahlreiche Entscheidungen erfordern. Aber viele Lerngegenstände der politischen Bildung sind an sich schon derart abstrahiert, dass dort die Beschäftigung mit der Darstellung unvermeidlich ist.}.
% \bigskip

Der Begriff Unterrichtsmaterial wird in dieser Arbeit nicht vollständig kongruent zur Definition auf der Website der \textcite{KMKMittel} für Lehr- und Lernmittel verwendet werden. Auf den Materialkarten der \gls{ank} selbst, sind Themenzuordnungen, Erklärtexte und Aufgabenstellungen, die sich häufig auf Quellmaterial beziehen, zu finden -- das ist alles \emph{Material}. Das Quellmaterial ist fast ausschließlich anhand eines \gls{url}, also Weblinks, angegeben. Diese führen zu verschiedenen Medienformen. Meist Text, häufig mit Grafiken und bisweilen Videos. Die Trennunschärfe des \emph{Material}begriffs wird hier nicht durch überbordende Kategorisierung versucht aufzulösen. In den Materialkarten selbst wird sich auf das Quellmaterial mit \emph{M1, M2 \gls{etc}} bezogen. Diese Abkürzungen werden genutzt und sollten zur Unterscheidung ausreichen. 


\paragraph{Herkunft von Lerngegenständen} 
In Deutschland gibt es eine Lernmittelfreiheit \autocite[]{KMKMittel}.
Welches Unterrichtsmaterial am Ende auch in Kontakt mit den Schüler*innen kommt, liegt hauptsächlich in den Händen der Lehrer*innen.
Die Lehrkräfte selbst sind bei der Auswahl des Materials jedoch einigen Einschränkungen ausgesetzt:
\begin{itemize}
    \item Zeitliche Zwänge: Der Tag hat nur 24 Stunden und Menschen sind Tiere mit zahlreichen biologischen und sozialen Bedürfnissen. Diese zu erfüllen ist Voraussetzung für eine gute Arbeitsleistung, nimmt jedoch, genau wie Arbeit selbst, Zeit in Anspruch. 
    \item Finanzielle Zwänge: Schulen stehen begrenzte Budgets zur Verfügung, z.B. sind Schulbücher häufig nicht die \enquote{besten} und aktuellsten am Markt und Exkursionen müssen für alle bezahlbar sein. 
    \item Bereits am Markt existierendes Material ist entweder kostspielig und/oder unterliegt nur unzureichenden Qualitätskontrollen. Die Prüfung und Umgestaltung, um im Unterricht eingesetzt werden zu können, nimmt wiederum viel Zeit in Anspruch. 
    \item Die vielfältigen Möglichkeiten Unterricht aufzubauen, und welches Material dafür benutzt werden kann, gehen ins Endlose. Daher kann nur exemplarisch gelernt werden. Das heißt, das jede Unterrichtsplanung immer von Entscheidungen geprägt ist. Das wiederum impliziert immer auch Entscheidungen gegen Alternativen.
\end{itemize}


Es besteht als Lehrkraft zwar meist die Möglichkeit weite Teile des Unterrichtsmaterials selbst zu erstellen.
Aber Unterrichtsmaterial, für welches bereits Entscheidungen bezüglich des exemplarischen Lerngegenstandes und der didaktischen Reduktion getroffen wurden, hat gewaltiges Potential, eine Arbeitserleichterung darzustellen.
Beispielsweise indem es für die Zielgruppe und die Zielvorgaben aus einem Lernplan bereits vorausgewählt und für eine ansprechende Darstellung aufbereitet wurde. 
\bigskip

% Unterrichtsmaterial wird häufig aus dem Internet bezogen \autocite[82]{Neumann2015}. % zitiert nach \autocite[66]{Hedtke2016}
Die Notwendigkeit des Lernens, eben auch Lerngegenstände zu haben, hat verschiedene Akteure mit unterschiedlichen Intentionen auf den Plan gerufen, welche Unterrichtsmaterial anbieten. 
Einsteigend hervorzuheben sind dabei die privatwirtschaftlichen Schulbuchverlage, welche dem klassischen Bild von Unterrichtsmaterial entsprechen. Jedoch haben auch andere Institutionen das Potential einer Einflussnahme auf Lernende durch 
% geschickt gestaltetes 
Material entdeckt. Wirtschaftsverbänden oder großen Unternehmen kann dabei bisweilen eine andere Intention, als eine altruistische, unterstellt werden, die durch, auf den ersten Blick kostenfreies, Material verfolgt wird. So braucht man in der Forschung von Reinhold Hedtke bisweilen nicht weiter als bis zu den römischen Seitenzahlen zu lesen, um darauf zu stoßen, dass Schulen \enquote{sich zunehmend gegen weltanschauliches, wissenschaftliches, wirtschaftliches und politisches Lobbying aus Unternehmen und ihrem Umfeld wehren} müssen \autocite[i]{Hedtke2016}. 

Im Falle von komplett selbst erstellten Unterrichtsstunden, kann von einer Lehrkraft erwartet werden, dass sie hehre Intentionen verfolgt und sich mindestens an die Vorgaben der \gls{kmk} \gls{bzw} die Umsetzung jener Vorgaben von den Bundesländern, sowie insbesondere an ihr eigenes, vernünftiges Urteil hält. 

Arbeit zieht deswegen auch vorgefertigtes Material nach sich, da die Beurteilung der Qualität und die Eignung, sowie die Anpassung für die Lerngruppe gemacht werden müssen. Das Material muss dazu nicht nur durchdrungen werden, um es für die eigenen Schüler*innen aufzubereiten, es sollte auch so intensiv bearbeitet werden, dass geschickt versteckte Beeinflussungsversuche von außen nicht unreflektiert bleiben. 





% Auch die \gls{ank} entwickelt in Kooperation mit dem \gls{zap} an der Universität Bremen Unterrichtsmaterial für den Politikunterricht.

% In dieser Arbeit soll ein Versuch unternommen werden, dieses Unterrichtsmaterial zu analysieren und zu bewerten. % welches den Lehrkräften von dem quasi staatlichen Akteur der Arbeitnehmerkammer Bremen zur Verfügung gestellt wird. 

% \paragraph{Unterrichtsmaterial}

% \subsection{Politische Bildung an berufsbildenden Schulen in Bremen}
% JA HIER FEHLT WAS, VIELLEICHT NEHME ICH DAS RAUS





\subsection{Politische Bildung und verwandte Kategorien \label{polBildung}}
\paragraph{Politische Bildung, Sozioökonomische Bildung, Arbeit, Gesellschaft, Wirtschaft, Poltik, Recht -- ein interdisziplinäres Begriffs-Wirrwarr} 
Politische Bildung % im Kontext staatlicher Bildungsakteure 
ist im Schulkontext häufig nicht nur in den Fächern Politik, Sozial- oder Gemeinschaftskunde verortet, sondern auch häufig in Fächern mit Bezeichnungen, in denen \enquote{Politik} gleichbedeutend neben \enquote{Wirtschaft} genannt wird. 
So finden sich im Sachstandsbericht der Wissenschaftlichen Dienste des Deutschen Bundestags (\citeyear[5]{WD8.2016}) Fächerbezeichnungen wie \enquote{Politik / Gesellschaft / Wirtschaft} für Hamburg, \enquote{Politik und Wirtschaft}, \enquote{Politik-Wirtschaft} \& \enquote{Wirtschaft/Politik} in Hessen, Niedersachsen und Schleswig-Holstein oder \enquote{Gemeinschaftskunde / Rechtserziehung / Wirtschaft} für Sachsen. Damit ist in 5 von 16 Bundesländern politische Bildung schon in der Fächerbezeichnung an Wirtschaft gekoppelt \autocite[vgl. zu der tatsächlichen Zeit, die für politische Bildung im Unterricht an allgemeinbildenden Schulen zur Verfügung steht auch][14 \& 16]{Gokbudak2020}.

Materielle Bedingungen in der Sphäre des Politischen sind nicht von der Hand zu weisen. Die Beschäftigung mit materiellen Verhältnissen in der politischen Bildung respektive mit Wirtschaft ist also naturgegeben.
Als Forschungsfeld wird daher gerne auch von Sozioökonomischer Bildung gesprochen.
Insbesondere zu diesem Begriff lässt sich Forschung zu Unterrichtsmaterial und Einflussnahme externer Akteur*innen finden.


Die Bezeichnungen sowohl von Forschungsdisziplinen als auch von Schulfächern versuchen dieser interdisziplinären Verknüpfung mal mehr, mal weniger gerecht zu werden.
Da die Trennschärfe solcher Begriffe Gefahr läuft, eine philosophische und linguistische Debatte über die Unzulänglichkeiten von Sprache zu eröffnen, wird in dieser Arbeit stets versucht, ein nach Ansicht des Autors möglichst passenden Begriff für den gerade gemeinten Gegenstand zu nutzen, welcher aber explizit nicht beansprucht, dass andere Begriffe aus leicht verändertem Blickwinkel nicht ebenso passend wären. Einen Begriff zu wählen, stellt lediglich eine notwendige Entscheidung dar. 

% Der Begriff Politik Wirtschaft ist dabei 

\noindent Als Tatsache kann angesehen werden, dass sich Gesellschaft und Kultur mit Wirtschaft und Politik alle wechselseitig beeinflussen. 


\subsection{Beutelsbacher Konsens \label{bbk}}% Beeinflussung durch Material.
Da insbesondere durch die Schulpflicht \autocite{BremSchulG} die schulische (politische) Bildung mit Zwang verbunden ist, liegt es nahe, dass versucht wird, diese durch Regeln, trotz des Zwanges, mit freiheitlichen, demokratischen Grundwerten zu vereinbaren. Auch an Berufsschulen, für die das hier untersuchte Material primär entworfen ist, besteht noch ein starkes, strukturvorgegebenes Machtgefälle, welches derartige Überlgungen stets aktuell sein lässt. 

Ein wichtiges historisches Dokument im Rahmen derartiger Fragestellungen ist der \Acrlong{bbk} \autocites[29]{Gesner2016}{Wehling1977} -- so wichtig, dass auch über den deutschen Raum hinaus auf den \gls{bbk} Bezug genommen wird. So wurde im Rahmen der \emph{Aktionstage Politische Bildung} vom 23. April bis 9. Mai 2024 in Österreich formuliert, dass der \gls{bbk} \enquote{[v]on internationaler Bedeutung [...] ist} \autocite{bbkÖsterreich2023}. Entsprechend leitet auch der Bremer Bildungsplan zentrale Kompetenzen aus dem \gls{bbk} ab und gibt die wichtigste Stelle sogar \enquote{im Wortlaut} wieder \autocite[11-12]{bplan}.
Zusammengefasst hat Wehling im \gls{bbk} drei konsensuale Punkte herausgearbeit:
\begin{enumerate}
    \item \emph{Überwätligungsverbot}: Selbstständige Urteile sollen mit dem Ziel der Mündigkeit ermöglicht und nicht verhindert werden. 
    \item \emph{Kontroversitätsgebot}: Durch das Verbot von Indoktrination ergibt sich quasi die Notwendigkeit, alles was \enquote{in Wissenschaft und Politik kontrovers ist} auch \enquote{im Unterricht kontrovers erscheinen} zu lassen. Das sorgt auch dafür, dass die Meinung der Lehrkraft unerheblicher wird.
    \item \emph{Handlungsorientierung}: Als \enquote{logische Konsequenz} müssen Schüler*innen \enquote{in die Lage versetzt werden, eine politische Situation und [die] eigene Interessenlage zu analysieren, sowie nach Mitteln und Wegen zu suchen, die vorgefundene politische Lage im Sinne [der eigenen] Interessen zu beeinflussen.} Dazu müssen realistischer Weise \emph{operationale Fähigkeiten} gelehrt werden. 
    
    \autocite[][179-180]{Wehling1977}
\end{enumerate}



\subsection{Bremer Bildungsplan \label{bplan}} % \& Kompetenzen in der politischen Bildung
Der aktuelle Bremer Bildungsplan für politische Bildung\footnote{Als Fach vom \gls{lis} nur als Politik bezeichnet} für berufsbildende Schulen ist mit der Erscheinung \citeyear{bplan} (\citeauthor{bplan}) vergleichsweise aktuell. Die Bremer Bildungspläne für politische Bildung an der \gls{sek} I und II der allgemeinbildenden Schulen sind alle etwa im vorletzten Jahrzehnt, zwischen \citeyear{vogel2006gy} \autocites{vogel2006gs, vogel2006gy, lower2008} und \citeyear{vogel2010gp} \autocite{vogel2010gp}, erschienen. Die Fächerbezeichnungen sind innerhalb der Bremer Bildungspläne nicht kongruent. Die unstete Bezeichnung der Fächer -- welche politische Bildung, häufig neben anderen Bereichen, mit abdecken -- wird in \gls{abs} \ref{polBildung} (\gls{S} \pageref{polBildung}) weiter diskutiert.

Politik ist neben Deutsch im Jahr 2024/2025 aktuell das einzige andere Fach im Land Bremen, welches für den berufsbildenden Bereich fächerübergreifend für alle Schülerinnen und Schüler unterrichtet wird (vgl. Website des \gls{lis} \citeyear{LisBildungspläne}). Direkt im Bildungsplan \autocite[][4]{bplan} wird darauf eingegangen, dass zwar an \enquote{heterogene[...] Voraussetzungen der Schüler:innen} angeknüpft wird, was auch \enquote{die Unterschiedlichkeit der jeweiligen Bildungsgänge} berücksichtigen soll, aber als Teil der \enquote{allgemeinbildenden Fächer} sollen die \enquote{Zielsetzungen politische Urteilsfähigkeit, politische Handlungsfähigkeit und methodische Fähigkeiten} immer die Grundlage bilden. Allgemeine politische Kompetenzen sind daher die hauptsächlichen Ziele \autocite[9-13]{bplan}.

Für die \enquote{Kompetenzdimensionen in der politischen Bildung} orientiert sich der Bildungsplan stark an den Ausarbeitungen der \gls{gpje}.
Da in dieser Arbeit (\gls{s} \gls{abs} \ref{öffi}, \gls{S} \pageref{öffi}) auch auf die Verfügbarkeit und Nutzfreundlichkeit von Unterrichtsmaterial eingegangen wird, bietet es sich an, an dieser Stelle ein strukturelles Einzelfallbeispiel aufzuzeigen:

\subsubsection{Exkurs: Schlechte Quellenangaben und unzureichende Vorbildfunktion. Beginnt Medienkompetenz im Bildungsplan? \label{gpje}}
Der Bildungsplan selbst wird von fiskalisch finanzierten Stellen erarbeitet und für einen Bildungsbereich veröffentlicht, welcher sich auch als öffentlich versteht -- sowohl in Form von öffentlichen Geldern finanziert als auch in Form von öffentlich zugänglich -- da es sich um einen verbreiteten, staatlich vorgezeichneten \mbox{(Weiter-)Bildungsweg} % \mbox trennt Begriff nicht 
handelt. Der Bildungsplan ist auf der Website des \gls{lis} (\citeyear[]{LisBildungspläne}) auch öffentlich zugänglich im etablierten .pdf-Format vorzufinden und bietet allen Menschen die Möglichkeit, mit diesem öffentlich erarbeiteten Wissen zu arbeiten.
Soweit erscheint das Vorgehen kohärent und sinnvoll.

Der Bildungsplan orientiert sich \gls{ua} an den Bildungsstandards der \gls{gpje}. Konkret wird das \gls{ua} an Fußnote 13 im Bildungsplan \autocite[][9]{bplan} deutlich, vergleichbar zu folgender Fußnote\footnote{\url{http://gpje.de/wp-content/uploads/2017/01/Bildungsstandards-1.pdf}}. Da ein Zitierstil nicht nur kontextabhängig, traditionsabhängig und subjektiv ist, soll dahingehend selbstverständlicherweise keine Wertung stattfinden. Allerdings sind in diesem Fall bereits mehrere zitierstilübergreifende Grundsätze missachtet worden. Sich dafür zu entscheiden, einen Fußnotenzitierstil zu nutzen und an späterer Stelle ein komplettes Quellenverzeichnis, wäre eine solide Wahl. Auch wenn als dabei übliche Kurzform in der Fußnote ausschließlich einen kompletten Weblink anzugeben schon wild ist. Nur dass es schlicht kein komplettes Quellenverzeichnis gibt!  
Es gilt, ausreichend Informationen (lieber eine zu viel als eine zu wenig) zum Auffinden anzugeben, insbesondere und mindestens Äquivalente zu Autor*innen oder Herausgeber*innen, sowie Erscheinungsjahr und Titel und das Zugriffsdatum, falls Weblinks angegeben sind. 
Als findige*r Konsument*in von Text lassen sich nun Teile von Titel und Namen sowie das Akronym der Herausgebenden im Link herauslesen. Allerdings führt der Link im Mai 2025 auf eine weiße Seite und ist damit ein Paradebeispiel, weshalb \gls{ua} wenigstens ein Zugriffsdatum zur eigenen Absicherung angegeben sein sollte.

Als bemühter Autor einer universitären Abschlussarbeit findet man jedoch unter diesem Link\footnote{\url{https://gpje.de/publikationen/} (06.05.2025)} einen Download mit diesem Link\footnote{\url{https://gpje.de/wp-content/uploads/2024/07/Bildungsstandards-1.pdf} (05.05.2025)}, dessen \gls{url} entsprechend der Zahlen, die ein Datum darstellen könnten, womöglich eine aktuellere Version zu der im Bildungsplan \autocite[][9]{bplan} referierten verspricht. Laut Website und Untertitel soll es sich zwar lediglich um einen Entwurf handeln. Wobei \enquote{Entwurf} sich an dieser Stelle wohl eher als Synonym für \emph{Vorschlag} lesen lässt.  Das Dokument kann jedoch nach wie vor in seiner Veröffentlichungsform strukturell kritisiert werden:
\begin{itemize}
    \item Das in der \gls{url} suggerierte Datum impliziert nun eine Veröffentlichung oder Version vom Juli 2024, im Dokument selbst ist ein \gls{c} vom Wochenschau Verlag von 2004 angegeben. Es handelt sich höchstwahrscheinlich um eine Version der viel zitierten 2. Auflage von \citeyear[]{gpje2004}. Nur, dass im Bildungsplan nicht ersichtlich ist, ob er sich tatsächlich auf diese Version bezieht. Höchstwahrscheinlich bezog sich die alte \gls{url}, in der \enquote{2017} zu lesen ist, auf dieselbe Version, wie die, welche hier mit \enquote{2024} in der \gls{url} aufgefunden wurde. 
    
    
    \item Es gibt keinen Versionsverlauf, auf der eigenen Website der \gls{gpje} ließ sich eine Version von 2017, wie die \gls{url} im Bildungsplan \autocite[][9]{bplan} suggeriert, nicht finden. Die Zahlen 2017 und 2024 scheinen in diesem Fall schlicht keine andere Version zu kennzeichnen. Denkbar ist, dass es nur intern erneut hochgeladen wurde.

    \item Wenn die \gls{url} nicht gerade eine \gls{doi} \gls{oä} ist, ist diese Fußnote als einzige Quellenangabe daher ein Paradebeispiel, weshalb meist schon zu Beginn einer universitären Ausbildung darauf hingewiesen wird, dass man sich nicht ausschließlich auf eine \gls{url} verlassen sollte. 
    
    \item Es gibt auch keine vernünftige Suchfunktion auf der Website. Eine miese Suchfunktion und intransparente, nicht anpassbare Sortierungen scheinen allerdings gerade bei Websites mit in Teilen fiskalischer Finanzierung Best Practice zu sein.
    
    Zur Verteidigung sei angemerkt, dass es sich in diesem Fall um eine gemeinnützige Organisation handelt. Dass bei solchen bisweilen die Ressourcen knapp sind, ist keine Seltenheit.  
    
    \item Das Dokument über die Bildungsstandards der \gls{gpje} ist soweit schreibgeschützt, dass sich Text nicht kopieren lässt. Besonders hinsichtlich der zahlreichen Möglichkeiten, den Kopierschutz auf verschiedenen Websites entfernen zu lassen -- um im Jahre 2025 nicht in die Verlegenheit kommen zu müssen abzuschreiben -- erscheint das erstens unnötig und zweitens kontraproduktiv. Es kann dazu verleiten, das Dokument auf ohne große Prüfung hastig herausgesuchte Websites hochzuladen. Das wäre wahrscheinlich nicht die Intention, die mit dem Hinzufügen eines Kopierschutzes verfolgt wurde. 
    
    \item In der Satzung der \gls{gpje} (\citeyear[]{gpje.satzung}) ist unter \enquote{\S 1 Zweck der Gesellschaft} zu lesen, dass es sich bei der Organisation um eine \enquote{wissenschaftliche Fachgesellschaft, die der Förderung der wissenschaftlichen Auseinandersetzung mit Fragen der schulischen und außerschulischen politisch-gesellschaftlichen Bildung in Forschung und Lehre dient} handelt.
    Ferner verfolgt die \gls{gpje} laut ihrer Satzung \blockquote{ausschließlich und unmittelbar gemeinnützige Zwecke [...] Sie leistet dies insbesondere durch:
    
    a. Förderung des wissenschaftlichen Diskurses, der Forschung und der wissenschaftlichen Kooperation,
    
    b.Veranstaltung von Fachtagungen und Kongressen,
    
    c. wissenschaftliche Publizistik,
    
    d. Intensivierung der europäischen und der internationalen wissenschaftlichen Zusammenarbeit,
    
    e. Förderung der Lehre an Hochschulen und die Förderung des wissenschaftlichen Nachwuchses,
    
    f. das wissenschaftlich-politische Engagement für den Ausbau der Disziplin an den Hochschulen.}

    Hinsichtlich der in der Satzung formulierten Ziele, weiß der Widerspruch zwischen Form und Inhalt humoristisch zu begeistern. Bildungsstandards einer wissenschaftlichen Gesellschaft, welche mit der Intention, von verschiedenen -- auch und insbesondere öffentlichen -- Bildungsinstitutionen genutzt zu werden, sind auf der eigenen Website schwer zugänglich (Kopierschutz), es gibt keine Suchfunktion und keinen stabilen Link (oder Versionsverlauf, wenn es sich um verschiedene Versionen handeln würde). Das erschwert wissenschaftlich korrektes Zitieren unnötig -- was sich tatsächlich auch in einem offiziellen Bildungsplan \autocite[][9]{bplan} niederschlägt, der diese Situation allerdings erst durch die unsaubere Arbeit derart verschärft hat. 
    
    \item Man stelle sich vor, man ist etwa Schüler*in, Lehrkraft, Elternteil oder Student*in und wirklich daran interessiert, in die Tiefe der Bildungsstandards einzusteigen, und ist soweit wie oben beschrieben gekommen -- dann kommt man möglicherweise auf die Idee sich sicherheitshalber die wissenschaftliche Publizistik der \gls{gpje} anzuschauen, um nachzuprüfen, ob es doch verschiedene und womöglich aktuellere Versionen gibt. Die Publizistik wird im Wochenschau Verlag vertrieben und ist nicht frei zugänglich. 
    
    Man könnte auf die Idee kommen, in einer Veröffentlichung von \citeyear{Gortler.2017} \gls{zb} (\citeauthor{Gortler.2017}) könnte ja eine passende Version von 2017 zu finden sein. Das auf Vertriebswebsites frei einsehbare Inhaltsverzeichnis reicht, um das zu verneinen. Was hilfreich ist. Denn auch wenn dem Band eine Tagung an der Universität Bremen von 2016 zur Grundlage liegt, ließen sich digitale (und physische) Versionen -- auch im Universitäts-Netzwerk der \gls{suub} eingeloggt -- nur gegen Geld erwerben. Die \gls{suub} besagter Bremer Uni führt immerhin ein physisches Exemplar auf Ebene 3 (Stand 06.05.2025). 

    \item An dieser Stelle ist dann womöglich ausreichend Sicherheit darin gefunden, dass die Version von \citeyear{gpje2004}, die aktuelle ist, auf die sich bezogen wird.

    \item Immerhin gibt es überhaupt eine kostenlose online Version und es müssen nicht tausende Menschen, die was mit dem Bildungssystem in Bremen am Hut haben, auf irgendein physisches Exemplar in einer Bibliothek verwiesen werden. % in der \gls{suub} verwiesen werden (Stand 30.05.2025).    
\end{itemize}
Einem Bildungssystem, dessen Regelungen und Grundlagen nicht hürdenarm, frei und öffentlich zugänglich sind, lassen sich nämlich strukturelle Unzulänglichkeiten im Rahmen einer freiheitlichen Demokratie unterstellen. 
% Es erschiene fast die Arbeitsbschaffungsmaßnahme des Müllsammelns sinnvoller, als sich durch einen derartigen Dschungel an Recherchearbeit schlagen zu müssen. Wenn es nur auch im Müllsammel-Beispiel nicht sinvoller wäre, die Umgebung so zu gestalten, dass Müll korrekt zu entsorgen niederschwelliger ist und damit eine höhere Wahrscheinlichkeit bekommt, sodass Müllsammeln kaum nötig ist.  

An späterer Stelle in dieser Arbeit (\gls{s} \gls{abs} \ref{media}: \gls{S} \pageref{media}ff.) wird das an dieser Stelle dargelegte Beispiel insofern aufgegriffen, als das erläutert wird, weshalb Medienkompetenz und gute Quellenarbeit für die politischen Handlungskompetenzen sowohl Voraussetzung als auch Lerninhalt und damit Vorbildfunktion darstellen. 

\subsection{Kompetenzen in der politischen Bildung}
Im Bildungsplan wurden diverse Entscheidungen getroffen. Insbesondere das Rahmenwerk, wie Kompetenzen für politische Bildung auszuformulieren sind, war und ist Bestandteil des (wissenschaftlichen) Diskurses. Die historische Entwicklung von politischer Bildung soll an dieser Stelle weitestgehend ausgespart werden. 
Markus Gloe und Tonio Oeftering steigen zwar mit einem Überblick tiefer in die Materie ein \autocite[95-100]{Gloe2020}, aber hangeln sich dann über den \gls{bbk} \autocite[101-103; in dieser Arbeit \gls{s} \gls{abs} \ref{bbk}: \gls{S} \pageref{bbk}]{Gloe2020} zu den heute maßgeblichen Kompetenzen \autocite[104-114]{Gloe2020}. Sie umreißen in ihrem Beitrag einen Diskurs bezüglich verschiedener Kompetenzmodelle in der politischen Bildung. Da Unterricht -- und damit auch Unterrichtsmaterial -- das Ziel der Kompetenzentwicklung hat, wird nun angelehnt an \citeauthor{Gloe2020} ein Überblick gegeben:

Das Modell der \gls{gpje} mit der früheren Veröffentlichung \citeyear{gpje2004} hat den Diskurs geprägt und findet deswegen womöglich auch besondere Erwähnung im Bildungsplan -- weniger des Inhalts, aber der Form wegen (\gls{vgl} dazu \gls{abs} \ref{gpje}, \gls{S} \pageref{gpje}).
% diggi, hier vielleicht deskriptiv Kompetenzen der gpje umreißen
Dieses Modell besteht aus den drei untereinander vernetzten Kompetenzen \emph{Politische Urteilsfähigkeit}, \emph{Politische Handlungsfähigkeit} und \emph{Methodische Fähigkeiten}, welche von \emph{Konzeptuellem Deutungswissen} gerahmt werden:
\begin{figure}[H]
    \centering
    \includegraphics[width=1\linewidth]{gpje 2004 Kompetenzmodell S. 13.png}
    \caption{\enquote{Kompetenzbereiche} der \gls{gpje} \autocite[13]{gpje2004}. Genauso auch im Bildungsplan \autocite[10]{bplan} zu finden.}
    \label{gpjeKompetenzmodell}
\end{figure}

\textcite[106-107]{Gloe2020} skizzieren weiter, wie zu dieser Zeit auch 
\enquote{die eher sozialwissenschaftlich geprägten Vertreter[*innen] der Fachdidaktik Günter C. Behrmann, Tilman Grammes, Sibylle Reinhardt und Peter Hampe Kernkompetenzen} der politischen Bildung als fünf \enquote{Demokratie-Kompetenzen} vorschlagen:
\begin{itemize}
    \item Wahrnehmung und Übernahme der Handlungsperspektiven Anderer, auch Dritter, zum Wechsel der eigenen Perspektive, zur Vermittlung des Eigeninteresses mit den Interessen Nah- und Fernstehender und dessen Ausweitung in Richtung auf allgemeinere Interessen (\emph{Perspektivenübernahme}); 
    \item diskursive[\dots] Klärung konkurrierender und konfligierender Ideen und Interessen und zum Aushandeln von Konfliktregelungen und -lösungen (\emph{Konfliktfähigkeit}); 
    \item problemorientierte[\dots] Analyse struktureller Bedingungen und institutioneller Ordnungen sozialen, insbesondere politischen und wirtschaftlichen Handelns und zum Gebrauch sozialwissenschaftlicher Begriffe und Methoden (\emph{sozialwissenschaftliches Analysieren}); 
    \item Einschätzung und Bewertung gesellschaftlicher Problemlagen, politischer Forderungen, Handlungschancen und -alternativen sowie zum reflektierten Gebrauch von Urteilskriterien (\emph{politische Urteilsfähigkeit}); 
    \item Beteiligung an bürgerschaftlicher Selbstverwaltung, sozialen und politischen Initiativen, innerbetrieblicher und -organisatorischer Mitbestimmung, informellen und formalisierten Prozessen öffentlicher Meinungs- und Willensbildung (\emph{Partizipationsfähigkeit/demokratische Handlungskompetenz}) 

    (\textcite[337 f.]{Behrmann.2004} \gls{zit} nach \textcite[106-107]{Gloe2020})
\end{itemize}
\citeyear{weißeno.2010} veröffentlichten Georg Weißeno, Joachim Detjen, Ingo Juchler, Peter Massing und Dagmar Richter \citetitle[\emph{Ein Kompetenzmodell}]{weißeno.2010}. Das darin vorgebrachte Kompetenzmodell % (\gls{s} \gls{abb} \ref{2010kompMod}: \gls{S} \pageref{2010kompMod}) 
sorgte für einen neuen Diskussionsanstoß. 
\begin{figure}[htb]
    \centering
    \includegraphics[width=1\linewidth]{Weißeno et al. 2010 p.12.png}
    \caption{\enquote{Basis- und Fachkonzepte der Politik} \autocite[12]{weißeno.2010}}
    \label{2010kompMod}
\end{figure}
Wie in \gls{abb} \ref{2010kompMod} (\gls{S} \pageref{2010kompMod}) zu sehen, sind 30 Fachkonzepte drei Basiskonzepten zugeordnet. Es wird Wert darauf gelegt, mit Schüler*innen-Vorstellungen zu arbeiten. % Im Sinne des Konstruktivismus keine abwegige Vorstellung.
An Aussagen wie \enquote{Für erfolgreiches Lernen gilt, dass die Schüler/-innen mit dem in dieser Darstellung vorgestellten Fachvokabular angemessen umgehen können} \autocite[13]{weißeno.2010} lässt sich jedoch Kritik üben. Für die Messung des Outputs ist Fachvokabular sicherlich hilfreich, aber es ist nicht zwingend notwendige Bedingung, um die Wirkungsmechanismen von Macht zu verstehen oder das Zustandekommen von kollektiv verbindlichen Entscheidungen nachvollziehen und beeinflussen zu können -- was womöglich auch Dimensionen von Kompetenzen der politischen Bildung sind. Entsprechend vielfältiger Möglichkeiten Kritik zu üben, skizzieren \textcite[108-109]{Gloe2020} wie sich die Autorengruppe Fachdidaktik\footnote{
    Anja Besand, Tilman Grammes, Reinhold Hedtke, Peter Henkenborg, Dirk Lange, Andreas Petrik, Sibylle Reinhardt und Wolfgang Sander} 
zusammenfand und in ihrem Werk \citetitle{Besand.2011} -- wie der Untertitel \emph{Eine Streitschrift} impliziert -- auch entsprechend Kritik am Modell um \citeauthor{weißeno.2010} üben. 

Auch die Autorengruppe Fachdidaktik hat der Komplexität der Realität folgend eine wortreiche und zahlreich gegliederte Graphik in ihrer Streitschrift abgebildet. Diese ermöglicht einen schnelleren Überblick auf deren vorgeschlagene Einteilung der Kompetenzen der politischen Bildung in Basis- und Fachkonzepte, auch wenn sie keine strengen Vorgaben zur begrifflichen Einteilung geben möchte (\gls{abb} \ref{2011kompMod}, \gls{S} \pageref{2011kompMod}).

\begin{figure}[htb]
    \centering
    \includegraphics[width=1\linewidth]{Autorengruppe p. 170 nach GloeOeftering p. 109.png}
    \caption{Basis- und Fachkonzepte der politischen Bildung der \\ \textcite[170]{Besand.2011} \gls{zit} nach \textcite[109]{Gloe2020}}
    \label{2011kompMod}
\end{figure}

Der entfachte Diskurs % (oder die kontinuierliche (Lohn-)Arbeit der Forschenden?) führte zu einer Art Schlagabtausch\footnote{
    % Was veranlasst an dieser Stelle die Nutzung dieses Wortes und der damit verbundenen schriftlichen Tonalität? \Gls{zb} Die schriftliche Tonalität des Beitrages \citetitle{Massing.2011}: \enquote{Auch zu den Vorschlägen der Autorengruppe Fachdidaktik wird hier nichts gesagt, die ein Teil ihrer Kritik nur wiederholt und deren Alternative bestenfalls den Status einer Ideensammlung hat.} \autocite[135]{Massing.2011}.}, 
fruchtete in einem überarbeiteten Modell (\gls{abb} \ref{2012kompMod}, \gls{S} \pageref{2012kompMod}).% aber auch

\begin{figure}[htb]
    \centering
    \includegraphics[width=1\linewidth]{Detjen et al. 2012 p. 15.png}
    \caption{\enquote{Modell der Politikkompetenz} \autocite[15]{Detjen.2012}}
    \label{2012kompMod}
\end{figure}

Im Rahmen dieses erneurten Modells, wurde sich weiter abgearbeitet. So führt \textcite[27]{Massing2012} aus, wie \emph{Artikulieren}, \emph{Argumentieren}, \emph{Verhandeln} und \emph{Entscheiden} die \enquote{Kompetenzfacetten} des \emph{kommunikativen-} und \emph{partizipativen politischen Handelns} bilden. 
Insbesondere für das \emph{partizipative politische Handeln} führt \textcite[27]{Massing2012} aus, dass beim \emph{Verhandeln} Machtpotenziale, Konfliktfähigkeit, ökonomische Ressourcen oder Tausch genutzt werden können, um \enquote{Verhandlungsprozesse abzukürzen und zu hierarchisch autoritären oder hierarchisch majoritären Entscheidungen zu gelangen. Diese lassen sich im Unterricht zwar nicht erfahren, sollten aber gewusst werden}. 
Mit einem kreativeren Ansatz ließe sich zwar überlegen, inwiefern \gls{zb} mit einer interdisziplinären Verstrickung mit dem Fach Darstellendes Spiel nicht auch ein ausreichend geschützter Rahmen geschaffen werden kann, in welchem Machtpotenziale durchaus erfahren werden können. Für \emph{Entscheiden} wird jedoch ausgeführt, dass \enquote{Entscheiden durch kooperative oder handlungsorientierte Methoden geübt werden kann}, es aber \enquote{gegenüber der realen Politik unterkomplex} bleibt \autocite[27]{Massing2012}. Diese Kritik bleibt berechtigt und eine ähnlich gelagerte Problemstellung wird im Schlussteil (\gls{abs} \ref{fakePartizipation}, \gls{S} \pageref{fakePartizipation}) aufgegriffen. 


% Zum Verhandeln gehören neben argumentativen Strategien, die der konsensuellen Entscheidungsfindung dienen, auch Strategien, die durch den Einsatz von Machtpotenzialen, Konfliktfähigkeit, ökonomischen Ressourcen oder Tausch versuchen, Verhandlungsprozesse abzukürzen und zu hierarchisch autoritären oder hierarchisch majoritären Entscheidungen zu gelangen. Diese lassen sich im Unterricht zwar nicht erfahren, sollten aber gewusst werden. Entscheiden als Teil des realen partizipativen politischen Handelns lässt sich im Unterricht nur begrenzt fördern. Auch wenn Entscheiden durch kooperative oder handlungsorientierte Methoden geübt werden kann, bleibt es gegenüber der realen Politik unterkomplex. Allerdings lassen sich an konkreten politischen Fällen, Problemen, Konflikten oder Entscheidungsprozessen unterschiedliche Strategien und ihre Wirksamkeit analysieren, deren Ergebnisse den Lernenden dann als Fachwissen zur Verfügung stehen. 


%Die Dimensionen sind dabei nicht isoliert, sondern interagieren miteinander, um ein umfassendes Verständnis und eine effektive Handlungskompetenz in politischen Kontexten zu fördern.


% DEN BELEIDIGTEN FROSCH VERFOLGE ICH WEITER; WENN NOCH ZEIT IST
% Massing stellt diesen Zusammenhang wie ein beleidigter Frosch nicht her\autocite[23]{Massing.2022}




Dieser Diskurs um die Ausformulierung von Kompetenzen in Worte und zweidimensionale, druckbare Modelle, welche wiederum mit rahmenden Worten das vernetzte, interdependente und quasi dreidimensionalere der Realität eingebaut bekommen, wurde intensiv geführt. Es bleibt mit gewissem Abstand aber zu erkennen, dass sich die Modelle in weiten Teilen dann doch ähneln. 
Die einen \autocite{weißeno.2010} hatten den Fokus auf klare und hilfreiche Strukturen gelegt und sind daher das Wagnis eingegangen, sehr konkrete Begrifflichkeiten zu finden. Um als Lehrkraft oder Schüler*in Orientierung zu bekommen, ist das sicherlich hilfreich. Den Anderen \autocite{Besand.2011} war womöglich wichtiger zu verdeutlichen, dass nicht zwangsweise Fachwissen im Fokus stehen muss, da Zusammenhänge auch ohne die letzten Endes austauschbaren Begriffe verstanden werden können und dieses Verstehen im Vordergrund stehen sollte. 

An welchem Punkt der \gls{bbk} recht deutlich wird und damit \gls{me} auch einen Kern von echter Demokratie trifft, ist die politische Handlungsfähigkeit. In einer pluralistischen Gesellschaft macht es viel Arbeit, wenn alle möglichst gleichen Zugang zu Entscheidungsbeeinflussungen haben sollen. Es unterstützt aber auch den demokratischen Gedanken. 
Frank \textcite[466]{Nonnenmacher2010} beginnt einen Abschnitt damit, dass bisweilen \emph{handlungsorientiert} mit \emph{schüler*innen-aktivierend} gleichgesetzt wird. Von der Wortbedeutung her ist das sicherlich nicht falsch. Aber schon im \gls{bbk} steht, dass man \enquote{die vorgefundene politische Lage nach seinen % [/ihren] 
Interessen [...] beeinflussen} lernen soll (\gls{vgl} \gls{abs} \ref{bbk}: \gls{S} \pageref{bbk}). \textcite[466-467]{Nonnenmacher2010} führt jedoch aus, dass die Debatte aus den 1970er Jahren eher dazu geführt hätte, dass \enquote{das politische Handeln im engeren Sinne -- assoziiert mit Demonstration, Streik und zivilem Ungehorsam -- von vorneherein in den Verdacht gestellt wurde und wird, dem bloßen Aktionismus, also einer entrationalisierten und emotionalisierten Geschäftigkeit Vorschub zu leisten}, was dem \emph{Überwältigungsverbot} widersprechen würde.
Er hebt hervor, wie unentschuldigtes Fehlen im Unterricht wegen Demonstrationen zum Bildungsnotstand sanktioniert werden soll, aber wenn \enquote{karitative Bemühungen hinter einer \enquote{Aktion} stehen, ist öffentliches Lob zu erwarten, weil ehrenamtliches soziales Engagement in einer Gesellschaft, die gerade auf diesem Gebiet immer mehr Verantwortung in den privaten Bereich verlagern will, hoch willkommen ist} \autocite[467]{Nonnenmacher2010}.

Er stellt drei Bedingungen, die eingehalten werden sollten, damit tatsächliche poltische Aktionen im Zusammenhang mit Schule stattfinden können:
\begin{enumerate}
    \item Engagement auf \enquote{breiter Wissensbasis} nach vorhergehender \enquote{Sachanalyse}
    \item Absolute Freiwilligkeit für alle Beteileigten. Demnach auch außerhalb der Unterrichtszeit stattfindend
    \item Öffentlichkeit herstellen. % $\rightarrow$ Das bedeutet 
    
    \autocite[467]{Nonnenmacher2010}
\end{enumerate}


Der Punkt mit der sinnvollen Einschränkung der Freiwilligkeit schließt \enquote{echte} poltische Aktionen für das hier untersuchte Material natürlich weitestgehend aus. Aber es ist davon auszugehen, dass es etwas zwischen \enquote{so banale[n] Tätigkeiten wie Lückentexte ausfüllen} \autocite[466]{Nonnenmacher2010} und einer Demonstration mit zivilem Ungehorsam gibt, was schüler*innen-aktivierend ist und auch die Chance bietet, politisches Handeln wenigstens zu simulieren. 
\emph{Simulieren} unter anderem, weil \textcite[467]{Nonnenmacher2010} dazu korrekterweise anmerkt, dass Schule eine \enquote{Zwangsveranstaltung} sei. Er geht darauf ein, wie verschiedene Initiativen sich immer wieder darum bemühen, Schule zu einem demokratischeren Raum zu machen \autocite[467-469]{Nonnenmacher2010}, kommt jedoch zu einem ernüchternden Schluss, wie insbesondere folgendes Zitat verdeutlicht:
\begin{quote}
    Nichts ist aber an dem Faktum zu ändern, dass diese partizipatorischen Elemente immer nur \enquote{gewährt} werden können; solche Spielräume sind eben wie in allen autoritären Systemen auch jederzeit von den Trägern der Macht wieder zurücknehmbar. 

    \autocite[468]{Nonnenmacher2010}
\end{quote}



\subsection{Arbeitnehmerkammer Bremen (ANK) \label{ank}}
Im Gesetz über die Arbeitnehmerkammer im Lande Bremen \autocite[]{ArbnkG} ist festgesetzt, dass nach der Beitragsordnung \autocite[]{ArbnkB} Beiträge nahezu aller Arbeitnehmenden\footnote{ 
    Auf der Lohnabrechnung für einen Minijob taucht \gls{zb} kein Abzug für die \gls{ank} auf. Diese Grenze orientiert sich an der Geringfügigkeitsgrenze, welche für 2025 556€ beträgt \autocites{b.gering}{banz.gering}. \label{Geringfügigkeitsgrenze}} 
in Bremen zu erheben sind. Für 2025 sind das 0,12\,\% des des steuerpflichtigen Arbeitslohns \autocite{ANKBeitragAb2025}. Systemisch ähnlich zu Steuerzahlungen ist die Mitgliedschaft in der Kammer daher mit Zwangsbeiträgen verbunden. 

Im Gesetz zur \gls{ank} ist in \S2(1) \autocite[1]{ArbnkG} zu den Aufgaben der Kammer unter anderem formuliert: \enquote{Maßnahmen zur Förderung und Durchführung der beruflichen sowie der allgemeinen und politischen Weiterbildung der Kammerzugehörigen zu treffen}.

Bildungsmaterial für die Berufsschulen in Bremen anbieten zu wollen ist also auch im gesetzlichen Rahmen als Aufgabe der \gls{ank} zu rechtfertigen.
Im vorliegenden Fall ist das \gls{zap} der Universität Bremen mit der Entwicklung des Materials betraut, welches später in der Arbeit exemplarisch durchleuchtet werden soll. Zu finden sind die Entwürfe im Anhang ab \gls{S} \pageref{MaterialkartenStart}. 

% Da die \gls{ank}, wie der Name schon vermittelt, durchaus partikulare Interessen der Arbeitnehmer*innen vertritt, 



\subsection{Medienkompetenz und die unbedingte Verknüpfung zu \enquote{Wahrheit} \label{media}} 
% \subsection{\enquote{Wahrheit}, \enquote{Wahrhaftigkeit} und \enquote{Richtigkeit} \label{wahr}}
% Für die Argumentation im Abschnitt der Medienkompetenz (\gls{abs} \ref{media}, \gls{S} \pageref{media}) ist eine Begriffsdefinition der drei Begriffe aus der Überschrift hilfreich. 
Für die Argumentation um Medienkompetenz ist es hilfreich sich dem Begriff \emph{Wahrheit} anzunähern. 
Da diverse Kompetenzmodelle der politischen Bildung auch die politische Handlungsfähigkeit als Ziel attestieren und für diese auch Meinungsbildung und -durchsetzung stattfinden muss\footnote{
    Meist werden diese Kompetenzbezeichnungen eher mit dem Wort \emph{Urteil} gebildet. Dieser Bezeichnungs-Bias mag \gls{ua} durchaus damit zusammenhängen, dass Meinung immer mal wieder diskreditiert wird. Dieser Umstand ist übrigens der Aufhänger bei \textcite[]{Nullmeier2019}.

}, ist es nicht verwunderlich, dass \emph{Wahrheit} in dem Essay \citetitle{Nullmeier2019} von Frank \textcite[]{Nullmeier2019} verhandelt wird\footnote{Die folgende Aufzählung stammt aus meiner Hausarbeit \autocite[4]{Klein2022}}:
% ist es praktisch, dass diese Begriffe im Rahmen eines Kommentars zu dem Begriff Meinung dargestellt werden\footnote{Die folgende Aufzählung stammt aus meiner Hausarbeit \autocite[4]{Klein2022}}:

\noindent \textcite{Nullmeier2019} beobachtet kritisch, wie als Reaktion auf Desinformation eine absolute Wissenschaftlichkeit gefordert wird, welche die Meinung mit einem Anspruch auf Wahrheit \enquote{auf die Seite des zu Verwerfenden} \autocite{Nullmeier2019} stellt.  \citeauthor{Nullmeier2019} unterteilt Wahrheit in drei Begriffe:
\begin{enumerate}
    \item Wahrheit: Als Wahrheit empirischer Aussagen.
    \item Wahrhaftigkeit: Als Begriff, um mit Unwahrhaftigkeit die Lüge kennzeichnen zu können - In der Abgrenzung zu Unwissen oder sich als falsch erweisenden empirischen Wahrheit. In der Lüge ist die Wahrheit enthalten, sie wird nur nicht gesagt. 
    \item Richtigkeit: Als Begriff, um eine Wahrheit für normative Aussagen zu haben, die sich aus \enquote{eine[r] partiell wahrheitsanaloge[n] Logik der Argumentation mit der Annahme einer aus richtig und falsch bestehenden Binarität und einer potenziell allein richtigen Aussage [er]geben}. 
    
    \noindent \autocite{Nullmeier2019}
\end{enumerate} 
Er führt aus, dass Meinung in demokratischen, politischen Entscheidungsprozessen von einer immanenten Wichtigkeit ist. Entsprechend wichtig sollte die Meinung daher auch im Politikunterricht sein.
% Ich behaupte ebenfalls, dass Meinung für die Kompentenzen der politischen Bildung maßgeblich ist. 
Im besten Fall bilden sich solche Meinungen auf der Grundlage von Informationen, die sich in den Dimensionen von \emph{Wahrhaftigkeit} und \emph{Richtigkeit} gebildet haben. % \footnote{In dieser Arbeit wird dann meist weiter Wahrheit geschrieben. In dem Bewusstsein, dass bei vielen Wahrheiten immer noch ein Restrisiko von Irrtum besteht. Aber insbesondere die Abgrenzung zur bewussten Irreführung, also bei \autocite{Nullmeier2019} zur Unwahrhaftigkeit, ist wesentlich.}. 

% Noch konkreter argumentiert: 

% Eine konkrete und praktische Umsetzung von Medienkompetenz kann dabei in Quellenarbeit und Zitation gesehen werden. 
% Medienkompetenz ist jedoch weit mehr als das. 
 
% \subsubsection{Die Zerteilung der Welt durch Worte}
% \bigskip


\paragraph{Die Zerteilung der Welt durch Worte}


Dazu ein kleiner philosophischer Ausbruch in die Verknüpftheit von Wissen und Sprache: Auch in dem Entstehungszeitraum der vorliegenden Arbeit sind \emph{Wahrheit}, \emph{Wahrhaftigkeit} und \emph{Richtigkeit} nicht nur in der Kommunikation grundsätzlich, sondern insbesondere auch in der Öffentlichkeit in der Wechselwirkung mit \emph{politics} eine zentrale Debatte. Es kann nicht davon ausgegangen werden kann, dass alle in bestem Wissen und Gewissen nur wahrhaftig kommunizieren. 
% Im Zeitalter von Desinformation, \emph{framing}, Populismus und Propaganda ist für die Erziehung zu mündige(re)n Menschen die Auseinandersetzung mit der Produktion von Wissen und \emph{Wahrheit} unerlässlich. 
Ob es nun \emph{Fake-News}, \emph{Miss-} oder \emph{Desinformation}, (bewusst) irreführendes \emph{Framing}, unsaubere Arbeit\footnote{
    Dort verwischt die Grenze zwischen einer \emph{empirischen Wahrheit}, die dem Irrtum unterliegen kann und \emph{Wahrhaftigkeit}. Es kann sein, dass es jemand nicht besser wusste und daher irrtümliche Informationen kommuniziert wurden. Es kann aber auch sein, dass schon vermutet wurde, dass die Informationen fragwürdig oder irreführend dargestellt sind, aber Zwänge (\gls{bspw} Zeitdruck, empfundener gesellschaftlicher Druck, hierarchischer Druck im Arbeitsverhältnis) trotzdem zur Kommunikation der Information geführt hat.} 
oder etwas vergleichbares ist, womit man konfrontiert wird; ein mündiges Teilnehmen an gesellschaftlichen Diskursen setzt kritisches Denken voraus. Denn Informationen fragwürdiger Herkunft, bisweilen direkt mit dem Ziel der Beeinflussung, kursieren in nicht kontrollierbarem Maße. 

Die gerade aufgezählten Begriffe enthalten immer eine Form \emph{Unwahrhaftigkeit} oder das (teils bewusste) Auslassen von \emph{Wahrheit}. Um die verschiedenen Formen und Unterschiede des Irrens und der Lüge durchdringen zu können, ist entsprechend ein Anschneiden dieses Themenkomplexes unerlässlich.  

Über das bloße, eher oberflächliche, Verstehen von Inhalt selbst hinausgehend, ist ein wesentlicher Bestandteil von Medienkompetenz, den Inhalt bewerten zu können. In diesem Zusammenhang soll bewerten bedeuten, den Inhalt mit anderen Dingen zu verknüpfen. Denn da Wissen (oder \emph{Wahrheit}) grundsätzlich in Beziehung von Subjekten zur Welt und Beziehung untereinander erschaffen wird, ist es geradezu unabdingbarer Bestandteil eines verstehenden Wissenserwerbs, das Wissen auch mit dem Kontext, in welchem es entstanden ist, zu verknüpfen. Das Verknüpfen von Wissen ist dementsprechend auch Voraussetzung für ein tiefergehendes Verstehen. 

Die Diskussion, inwiefern Wissen auf Logik basierend auch mit wenig Beziehung zu anderen Subjekten und der dinglichen Welt \emph{erschaffen} oder \emph{erkannt} werden kann, soll an dieser Stelle ausgeklammert werden. Denn pragmatisch gesehen ist jede Form von Information in der physischen Welt mindestens als Energiezustand verankert. Menschen lernen das Denken insbesondere an ihren über die Sinnesorgane vermittelten Erfahrungen mit der dinglichen Welt, was dann erst mit der Zeit höhere Abstraktionsebenen (Symbole, Sprache \gls{etc}) ermöglicht. Und auch noch so stark abstrahierte Informationen sind eben immer noch mindestens irgendwie messbar in der physischen Welt verankert. Sei es eine Schallwelle des gesprochenen Wortes, eine neuronale Verbindung im Gehirn, Spannung oder Magneten in einem Computer, Symbole auf Papier oder spezifisch angeordnete Basentriplets einer messengerRNA auf dem Weg, ein Protein synthetisieren zu lassen. 

% Ein eingängiges Beispiel für die unbedingte Verknüpftheit 
Ein eingängiges Beispiel für die Beziehung von \emph{Wissen}, \emph{Wahrheit} und \emph{Sprache} lässt sich am Wort \enquote{Baum} darstellen. Erstens muss -- um dem Wort eine über den Zufall hinausgehende Bedeutung zu verleihen -- mindestens irgendein Subjekt das Wort tatsächlich mit einem Konzept von Baum verknüpfen. Wenn Sprache ihre Funktion der Kommunikation -- unter der wir sie kennen -- erfüllen soll, am besten mehrere Subjekte.
Ferner könnte ein Baum nicht existieren ohne Wasser, die Energie der Sonne, Jahrmillionen an Evolution, Erde \gls{etc} Sprache zerlegt eine zwangsläufig verknüpfte Welt immer in Bestandteile. 
Wer das Konzept (also das Wort und das Wissen, welches es bezeichnet) \enquote{Baum} verstehen möchte, weiß in irgendeiner Form auch Teile dieses zwangsläufigen verknüpften Wissens. Denn um das Wort \enquote{Baum} gebrauchen zu können, wird der Baum von der Sonne und aus der Erde herausgetrennt. Wenn das nicht so wäre, gäbe es keine Worte. Weil konsequent zu Ende gedacht immer nur ALLES ausgedrückt werden könnte, da über genügend Umwege alles miteinander verknüpft ist.
% \bigskip

Alleine auf einen Baum zu zeigen, den alle Anwesenden wahrnehmen können, und zu sagen: \enquote{Das ist ein Baum} bildet also nicht die \emph{ganze} Wahrheit ab. Solche philosophischen Spitzfindigkeiten spielen im Alltag, mit der Notwendigkeit handlungsfähig zu sein, aber eine untergordnete Rolle. % Entsprechend wird auch in dieser Arbeit mit dem vereinfachten Begriff einer überprüfbaren Wahrheit arbeiten. 
 % Erneut spielen solche philosophischen Spitzfindigkeiten im Alltag, mit der Notwendigkeit handlungsfähig zu sein, aber eine untergordente Rolle. Entsprechend wird auch in dieser Arbeit mit dem vereinfachten Begriff einer überprüfbaren Wahrheit arbeiten. 
 % die Notwendigkeit einer guten Quellenarbeit mit dem vereinfachten Begriff einer überprüfbaren Wahrheit arbeiten. 

% \vspace{12pt}     \hrule      \vspace{12pt}    back to it
\bigskip

\noindent Ohne dass in linguistischen Gedankenübungen der Bezug zum Alltag verloren werden muss, sind in der Informations und Screen-Time überfluteten Dekade des begonnenen dritten Milleniums die praktischen Bedeutungen um \emph{Wahrheit} und Verwandte notwendig, um beurteilen zu lernen, ob die normschöne Frau auf TikTok Recht damit hat, dass sie als Frau in die Küche gehört\footnote{
    Um den Begriff \emph{Tradwife}, die traditionelle Frau, die \enquote{granola from scratch} macht, wird der patriarchale Reichtum ihres Mannes zur schau gestellt, indem sie nie bei \enquote{echter} Sorgearbeit zu sehen ist. Es wurde geschafft die \emph{Trophy Wife} nicht nur in der Sphäre ihres obejktifizierten Körpers für patriarchale Herrschafts-Hegemonie zu nutzen -- jetzt wird der \emph{TradWife} auch scheinbar eigene Selbstwirksamkeit \emph{zugestanden}, indem sie performative Arbeit verrichten darf. Eine Hierarchisierung zu Menschen, die selbst putzen und Windeln wechseln müssen wird subtil aufgebaut. Andere Lebensrealitäten werden implizit als minderwertig dargestellt. Die Dinge, die in einem 20 sekündigen Video unterschwellig kommuniziert werden, zu durchdringen ist nicht trivial. Da Beispiele wie dieses aber omnipräsent sind, ist Medienkompetenz wichtig. 
    
    Weshalb Sorgearbeit eine wichtige Konfliktlinie darstellt, taucht tatsächlich immer wieder im Material der \gls{ank} auf.
}. 
Medienkompetenz sollte in allen Fächern der Schule grundsätzlich, aber aufgrund der großen Schnittmenge insbesondere im Politikunterricht mitgedacht und mitbehandelt werden. Ich gehe auch soweit zu sagen, dass Medienkompetenz lediglich ein Begriff für integrale Lerngegenstände von politischer Bildung ist, der bestimmte Bereiche betont. Diese Lerngegenstände ergeben sich sowohl aus der eben dargeleten Argumentation, aber auch aus der Zielsetzung der geforderten Kompetenzen. 





\paragraph{Quellen}

An diese Quellendarstellungen muss sich im Sinne der didaktischen Reduktion jetzt nicht sklavisch gehalten werden.
Als polemisches Beispiel: Im Mathematikunterricht in der Elementarstufe bedarf es womöglich zu viel mental load, um auf die Verschriftlichungen von Adam Ries Bezug zu nehmen. Aber vom Grundsatz her ist es durchaus angebracht, so viel Quellenmaterial wie möglich, so präzise wie möglich auch im Unterrichtsmaterial selbst mitanzugeben. Alleine um mit gutem Beispiel voranzugehen. 
% Diggah, mach halt immer ehrenloses Scheißquellengeballer, anstatt Infos gehaltloser rauszuballern als gmx.de newsflash. Medienkompetenz ist Sack, die Medienlandschaft und Lehrmaterial sind aber auch häufig genug beschämendes Negativbeispiel. Jeder Scheiß Porno hat mit seinen blöden Wasserzeichen mehr Qualität in der Nachverfolgbarkeit der Urheberschaft. Man


Eine gute Vorbildfunktion ist essentielle Voraussetzung für verschiedene Formen des Lernens, wie es theoretisiert wird.
Egal, ob es sich um das Beobachtungslernen \autocite[73ff.]{Kiesel2012} welches maßgeblich von Albert Bandura definiert wurde, handelt, oder ob im Rahmen des (zwar umstrittenen) impliziten Lernens \autocite[83ff.]{Kiesel2012} stattfindet. Ohne ein auch positiv-Beispiel zu haben, wird das Erlernen von der Relevanz und der methodischen Durchführungsmöglichkeiten einer guten Quellenarbeit erschwert. Aus dem Grunde ist der Bildungsbereich von Beginn an gut beraten, stets gute Beispiele für die Quellenarbeit darzustellen. 


Darüber hinaus Einstellung \autocite[130]{Kiesel2012}



Als Begründung für wildes Quellengebashe:
Es kann natürlich entschieden werden, unterschiedliche Maßstäbe anzulegen und starke Hierarchien und Ungleichheiten auszuleben. Die Wahrscheinlichkeit damit etwas zwischen Trotzreaktion und Gegenwehr bis hin zu Desinteresse und Lethargie auszlösen, dürfte damit aber ansteigen. 




Meine persönliche Empfehlung für den Ausbau von Medienkompetenz: Abschreiben. Finland macht das schon länger\footnote{
    Danke an @DougSharp, der mich aus dem Doom-Scrolling geholt hat, damit ich mein privates Gepredige von \enquote{Politik ist kein Diktat, man darf Abschreiben und Kooperieren hilft allen, das ist die Definiton von Kooperation} jetzt in meine Masterarbeit einbringen kann \url{https://www.youtube.com/shorts/geGQVGS6HSc} (14.07.2025)
} \autocites(\gls{vgl} überblicksweise \gls{zb})()[][]{Palsa.2015}{Salomaa.2019}. 


\subsection{Zielgruppe}
Gutes Unterrichtsmaterial ist nicht per se gut, sondern es ist lediglich abhängig von den Rezipient*innen gut. Eine womöglich gut vorbereitete Lerneinheit zu Polynomdivision für Studierende in Höherer Mathematik ist in einer Elementarstufe, die gerade überhaupt dividieren lernt, sicherlich nicht mehr gut. Gutes Unterrichtsmaterial ist also in erster Linie für die intendierte Zielgruppe zu untersuchen. Jede Lerngruppe ist individuell, dennoch lässt sich allein durch das Alter, den Lernort und -- zum Teil  mit dem Ort einhergehend -- die Sprache und viele weitere Faktoren, die Lerngruppe bereits stark abgrenzen. Genau das passiert auch bei einem Englischbuch, welches für Unterricht auf Deutsch in der 5./6. Klasse in Baden-Württemberg veröffentlicht ist. 

Es ist also angezeigt, die Zielgruppe des Unterrichts möglichst klar abzugrenzen, wenn untersucht werden soll, ob Material \enquote{gut} ist. 


Wenn es dann genauer an die Planung von konkretem Unterricht geht, sind insbesondere die Lehrpersonen gefordert, um auf den Wissensstand und bestehende Schüler*innen-Vorstellungen eingehen zu können. Nach dem Modell der didaktischen Rekonstruktion sollen Lerninhalte besser und nachhaltiger vermittelt werden, wenn dies geschieht \autocite[404-406]{Reinfried2009}.

Dadurch, dass das Material mit Fokus auf duale Studiengänge berufsbildende Schulen erstellt ist, unterscheidet sich die Zielgruppe von der an allgemeinbildenden Schulen in dem wesentlichen Punkt, dass gerade der Lebensweltbezug für Arbeitsthemen durch die eigenen Arbeitserfahrungen im Betrieb deutlich weniger theoretisch ist, als wenn an einer Oberschule zum Beispiel über Arbeitsschutz gesprochen wird. 

\enquote{Fremdzuschreibungen und defizitorientierte Ansprachen von Personengruppen sind deshalb problematisch, weil sie die in unserer Gesellschaft ungleich verteilten Zugänge zu u.a. Bildung und Chancen mitunter eher reproduzieren als sie – wie von einer inklusiven politischen Bildung} \autocite[]{Beckmann2022}
Dennoch ist es wichtig, die Zielgruppe zu kennen und darauf einzugehen.



2024-12-11
Wie wird die Zielgruppe angesprochen?
Wird eine heterogene Zielgruppe angesprochen?
Bla über Defizitorientierung
% https://profession-politischebildung.de/grundlagen/grundbegriffe/defizitorientierung/#:~:text=Defizitorientierung%20meint%20die%20Fokussierung%20auf,Bildungsangeboten%20sowie%20im%20p%C3%A4dagogischen%20Handeln.
% die website zitiert z.B: bei Citavi Holzer 2010

\subsubsection{Schüler*innenvorstellungen}
Nach dem Modell der didaktischen Rekonstruktion \autocite[]{Reinfried2009} ist die Inbezugnahme von Schüler*innenvorstellungen ein zentrales Element für besseren Unterricht.
Es soll also untersucht werden, an welchen Stellen das Material Schüler*innenvorstellungen aufgreift oder immerhin Raum dafür lässt. 



Die \gls{ank} als Gegenspieler zu wirtschaftsnahem Material? Engartner 2023:7

\subsection{Ökonomische versus politische Sozialisation?}
In der Forschung zu politischer Bildung in staatlichen Institutionen wird seit geraumer Zeit verhandelt, wie die Einflussnahme externer Akteure zu bewerten sei. Reinhold Hedtke 

Autor*innen, welche einen bildungstheoretischen und/oder einen politikwissenschaftlichen Hintergrund aufweisen, lesen sich insofern ähnlich, als dass ihnen gemein ist eine gute politische Bildung interdisziplinär zu gestalten. Insbesondere für den berufsbildenden Bereich lässt sich feststellen, dass Forderungen erwachsen, die \enquote{betriebswirtschaftlichen Verwertungslogiken}

dritte Säule \autocite[]{kerschensteiner1966}
% Diggah, wie soll ich Forschungsüberblick geben, wenn alleine Reinhold Hedtke gefühlt gut 200 Veröffentlichungen zur gleichen schmackhaften Soße hat?


\subsection{Mehr Platz für Emotionalität}
Gutes Unterrichtsmaterial gibt die Beziehungsebene zwischen Lehrer*innen und Schüler*innen nicht vor, aber sorgt im Idealfall durch gute Strukturierung und guten Umgang mit den bestehenden Ressourcen dafür, dass die Beziehungsebene mehr Platz bekommen kann.

Gleichzeitig sind Emotionen integraler Bestandteil des Politischen \autocite{Heidenreich.2012a} und sind daher nicht in einem veralteten Dualismus (der Rationalität unverbunden und moralisch unterlegen) aus dem Unterricht zu verbannen. % Auch wenn in dieser Hinsicht noch reichlich Nachholbedarf besteht. 

Dass Emotionen gerade in Zusammenhang mit Politik bisweilen einen faden Beigeschmack haben, trägt Hendrik Schröder (\citeyear[4-5]{Schroder.2020}) 



\subsection{Veröffentlichungsreichweite, Auffindbarkeit, Preis, Einsetzungsreichweite \label{öffi}}

Es darf davon geträumt werden, Unterrichtsmaterial, welches durch öffentliche oder fast öffentliche Gelder finanziert wurde, auch der Öffentlichkeit zugänglich ist.



\subsubsection{MUSS WOANDERS HIN}
AAAAnderes Kapitel
Es wird keine Begründung zur didaktischen Auswahl der Themen mitgeliefert. Damit einhergehend wird auch eine didaktische Reduktion nicht begründet. In der Schulpraxis ist das in explizit schriflticher Form sicherlich auch nicht üblich. In der Ausbildung von Lehrkräften ist genau das jedoch gefordert und es wird implizit erwartet, dass auch später stets eine solide Begründung geliefert werden könnte.
Das Material hat die Intention hat verbreitet und genutzt zu werden. Daher wäre es durchaus eine Überlegung wert, inwieweit es die Legitimation  erhöhen könnte, noch tiefer in die Entscheidungsbegründung einzustiegen.

\subsubsection{MUSS Ebenfalls WOANDERS HIN}
Bei der Bewertung von Material sollte Zugänglichkeit nicht übersehen werden. Bei den Zwängen, denen auch die Institution Schule unterliegt, ist eine wesentliche variable, die über Zugänglichkeit entscheidet, die der Kosten. 
Ein Schulbuch kann sehr gut sein, wenn das Budget knapp ist, ist die Chance hoch, das stattdesssen ein nicht aktuelles Schulbuch von geringerer Qualität in der Praxis genutzt wird. 


\subsubsection{MUSS AUCH WOANDERS HIN}
Strukturelle Ähnichkeiten fallen zu dem im Bildungsplan geforderten Verständnis der Möglichkeiten und Grenzen des Föderalismus und folgenden Überlegungen zu Unterrichtsmaterial grundsätzlich auf.
In einem vorstellbaren Ideal, wird eine offene Plattform für die Bereitstelleung von Unterrichtsmaterial bereitgestellt.
Diese sollte mehrere Voraussetzungen erfüllen:
\begin{itemize}
    \item offen in Form von kostenfrei, öffentlich \gls{vgl} \gls{abs} \ref{lizenz}: \gls{S} \pageref{lizenz}
    \item Das würde bedeuten, es sollte gepusht werden, dass das Ding bestenfalls weltweit -- aber um der Egozentrik der politics Sphäre gerecht zu werden vielleicht erstmal als europaweit beworben -- genutzt wird und nicht jedes Bundesland rumknausert und die gleiche Software $\geq$ 16 mal in Auftrag gibt.
    \item Das aus öffentlichen Geldern finanzierte Ding würde damit auch der Öffentlichkeit gehören und es würden nicht hauptsächlich privatwirtschaftliche Akteure davon profitieren, die die Rechte an ihrem schlechten Produkt halten, bei welchem auch niemand eine realistische Chance hätte es zu verbessen.
    \item Daraus folt $\rightarrow$ Die Plattform sollte als veränderbar angelegt sein.
    \item Das Material sollte von allen veränderbar sein. Was explizit nicht heißen muss, dass jede schlechte oder gar destruktive Änderung gesichtet oder gar übernommen werden muss.
    (Womöglich ist hier auch etwas in Richtung Klarnamen-Registrierungszwang angebracht. Hier sollte sich viel aus der Softwareentwicklung abgeschaut werden. Da steckt die Expertise. Insbesondere Funktionen zu forken und einen zweiten Branch aufzumachen bevor gemergt wird ist hier logische Grundlage und in der Entwicklung von Software mit git auch de-facto Standard.)
    \item Es sollte einen Versionverlauf haben (Ebenfalls de-facto Standard in der Software Entwicklung). Die Open Knowledge Foundation (ein eingetragener Verein) hatte schon eigenständig versucht Gesetzgebeung, welche viele strukturelle Ähnlichkeiten zu den Versionsverläufen von Software aufweist, in git abzubilden. Das Projekt ist jedoch nicht aktuell. Die Idee bleibt.
    \item Abstimmungsmöglichkeiten sollten gegeben sein. Dann hat man als womöglich staatlich veratnwortliche Person einen Überblick, was ein Ausschnitt der Öffentlichkeit zu dem Ding denkt.
    \item Im besten Falle verschieden filterbare Bewertungsskalen. \Gls{zb} nach Lehrkräfte, Schüler*innen, andere Externe, Herkunftsorte, Alter etc.
    \item 
    \item Überraschung, ich habe extra \emph{Ding} geschrieben. Weil das ziemlich ersetzbar für viele (informationslastige) Produkte ist, welche mit öffentlichen Ressourcen realisiert werden.
    \item diggah, als ob, ich habe kein bock mehr. es wird eh schlimmer anstatt besser
\end{itemize}


Programmierende Freunde von mir scherzen über \emph{pull requests} über \enquote{git} für eine Möglichkeit moderne, digitale Möglichkeiten für direktere Formen der Demokratie nutzen zu können.



BLUB BLUB

Jemand der in der Arbeitszeit für ein Unternehmen Produkte herstellt ist nicht Eigentümer*in. Das gilt auch für gesitiges Eigentum. Es gibt durchaus Unternehmen, die sich dafür Entscheiden, Dinge auch unter Lizenzen zu stellen, die der Öffentlichkeit zu Gute kommen\footnote{
    Wer neugierig ist, kann \gls{zb} beim eigenen Router für den Heim-Internetzugang mal die Open Source Lizenzen anschauen. Alles voll von MIT, GNU, Creative Commons und weiteren Lizenzen, die sich alle die Intention teilen, ihre lizensierten Produkte auch der Allgemeinheit zu Gute kommen zu lassen.
}. Die Regel ist aber, dass das Produkt nicht Eigentum der Arbeiter*innen ist. Wenn jemand beim Arbeitgeber einen Tisch baut, ist es der Tisch des Arbeitgebers. Wenn jemand bei Microsoft Software schreibt, ist es manchmal Open Source und kostenfrei nutzbar, häufig jedoch Eigentum von Microsoft.

Lehrer*innen werden ausscghließlich mit öffentlichen Geldern finanziert und sind der Verfassung verschrieben. Insbesondere der erhebliche Anteil an Beamt*innen als Staatsdiener*innen verdeutlicht die Rechenschaft gegenüber der sozialen Demokratie, die unser Staat sein soll. 
Lehrer*innen produzieren unheimlich viel Unterrichtsmaterial für diverse individuelle Gegebenheiten. In der Praxis bleibt dieses in der Arbeitszeit produzierte Material aber bei den jeweiligen Erstellenden und wird hauptsählich in Klassenverbänden und womöglich noch auf schulinternen Plattformen wie its-learning verbreitet. 

Das Potential, das mit öffentlichen Geldern finanzierte Material auch der gesamten Öffentlichkeit zu Gute kommen zu lassen, bleibt damit unerschöpft.

Damit einhergeht die Möglichkeit zur Effizienzsteigerung. Material zur Bruchrechnung könnte \gls{zb} weiter optimiert und auf verschiedene Gegebenheiten angepasst werden. Wenn es Evalutionen gibt, die durch in der Institution Schule übliche Tests gewissermaßen inhärent sind, gäbe es für Testungen das Potential nicht nur rigides Selektionsinstrument der Schüler*innen-Leistung zu sein, sondern tatsächlich Qualität von Material messbarer zu machen. 

Zusätzlich dazu wäre einer Öffetnlichkeit die Möglichkeit eröffnet, kritische Beobachterin der eigenen bildung zu sein und demokratische Einfluss nehmen zu können. 

Die Debatte über Gefahren der öffetnlichen Einflussnahme können strukturell mir der Debatte über die Schwächen der Demokratie, wie sie gelebt wird, zusammengelegt werden. Die Gefahr Mindeheiten zu übergehen, indem die Mehrheit wilde Entscheidungen trifft, lässt sich mit einem wandelbaren aber resilienten jedoch ebenso in ein stabiles Regelwerk einbetten, wie es demokratische Kräfte auch für Staaten versuchen. 

Es mag pathetisch wirken, aber im Spiel der multiplen Krisen auf der Welt, wirkt eine derartige Aufgabe durchaus realistisch und das Potential Bildung voranzubringen wird eröffnet anstatt im Kreis zu fahren. 



\section{Analysemaßstab}
Im Sachunterricht in der Elementarstufe wird kritisiert, wenn der Unterricht wenig mit der eigentlichen Sache zu tun hat \autocite[2-4]{Scholz2004}. Im Politikunterricht ist das Anschauen jedoch weniger auf Dinge bezogen. Analog dazu lässt sich jedoch die bereits im \gls{bbk} geforderte Handlungskompetenz sehen. Spannend ist es also zu untersuchen, inwiefern durch das Material politisches Handeln beobachtbar oder gar an der eigenen Gruppe erlebbar wird.


Einen Teil wird immer eine Fehlerprüfung einnehmen. 



\subsection{Fragen an das Material} % Nach welchen Maßstäben Unterrichtsmaterial analysieren?
Ein eher offen formulierter Bildungsplan ist kein Zufall. % Aus ähnlichen Gründen, die einen offen formulierten Bildungsplan nahelegen,
Daher wäre es kontraindiziert, Unterrichtsmaterial nach starren Vorgaben zu bewerten.
Dennoch soll eingegrenzt werden, nach welchen Maßstäben Unterrichtsmaterial bewertet werden könnte. Offen bedeutet nicht beliebig.

Die erste Anlaufstelle dafür ist der Bildungsplan selbst. In Anlehnung an die Gliederung des Bildungsplans kann das Unterrichtsmaterial anhand folgender Punkte untersucht werden:
\begin{itemize} 
    \item Kompetenzen jeweils der beruflichen \& politischen Bildung
    \item Lebensweltorientierungen % Modell der didaktischen Rekonstruktion
    \begin{itemize}
        \item Arbeits-,  Berufs- und Lebensweltorientierungen
        \item Problem-, und Wissenschaftsorientierungen
        \item Zukunfts-, Gegenwarts- und Vergangenheitsorientierungen
    \end{itemize}
    \item Methodische Grundsätze
    \item Und anhand folgender sieben politischen Handlungsfelder:
    \begin{itemize}
        \item Demokratie 
        \item Gesellschaft 
        \item Arbeitsleben
        \item Öffentlichkeit im digitalen Zeitalter
        \item Wirtschaftspolitik
        \item Globale Zusammenhänge 
        \item Nachhaltigkeit 
    \end{itemize}
\end{itemize}


Im Bildungsplan sind die drei wichtigsten Kompetenzen in Anlehnung an den \gls{bbk} formuliert. 

\begin{itemize}
    \item Politische Urteilsfähigkeit
    \item Politische Handlungsfähigkeit
    \item Methodische Fähigkeiten 
\end{itemize}

Daher soll im nächsten Schritt analysiert werden, inwieweit das Unterrichtsmaterial diese Kompetenzen fördert. 

% Maßgeblich dafür können Vergleiche zu Bewertungskriterien sein, die bereits von anderen genutzt worden sind.
% Auch maßgeblich soll sein, inwiefern schon das Unterrichtsmaterial in Bezug auf den Bildungsplan und dessen, unter Anderem aus dem \gls{bbk} abgeleiteten, Kriterien vereinbar scheint. 


Fragen, die darüber hinaus und ergänzend an das Material gestellt werden sollen, sind:
\begin{itemize}
    \item Welche Kompetenzen werden an welcher Stelle gefördert?
    \item Wie und durch welche Operationalisierung werden die Kompetenzen gefördert?
    \item Wie werden Schüler*innenvorstellungen berücksichtigt? % zB diametral entgegen falscher Vorstellungen oder auf richtige Vorstellungen aufbauend? Außerdem: Stichwort: Reifizierung \autocite[]{Reinfried2009}
    Siehe auch die reflexiven Fragen für das Modell der didaktischen Rekonstruktion bei \textcite[411-412]{Reinfried2009}.
    \item An welchen Stellen soll induktiv, an welchen deduktiv vorgegangen werden?
    \item Ist das im Modell der Didaktischen Rekonstruktion sinnvoll? \autocite[]{Reinfried2009} Sollte im Sinne des Konstruktivismus besonders induktives Vorgehen seitens der SuS antizipiert werden?
    \item Wie wird Schüler*innenaktivität erzeugt? % Für Konstruktivismus wichtig
    \item Wird bestehendes Material benutzt und analysiert oder wird auch eigene Produktion angeregt?
    \item Wie ist die Ergebnissicherung eingebunden?
    \item Welche Handlungsfelder werden angesprochen?
    \item Inwieweit bietet ist das Material auf die Lebensrealität und realistische Partizipationsmöglichkeiten ausgerichtet?
    \item Inwieweit impliziert das Material politisches Handeln, welches bestehende Systeme in Frage stellt? Ist das noch vertretbar (im Konkreten: mit dem \gls{bbk} vereinbar)? Ist im Gegenzug ein Weglassen solcher Perspektiven vertretbar (im Konkreten: mit dem \gls{bbk} vereinbar)?
    \item Welche Medien werden eingesetzt? Wie ist die Wahl der Medien begründet?
    \item Medienkompetenz % siehe Demokratie ANK Dinger Kommentare für die Quellen \emph{M1} und \emph{M2} letzte Seite
    \item Inklusion
    \item Digitalisierung (Methodenkompetenz. Sinnvoll eingesetzt?)
    \item Was wird vom Material an Möglichkeiten der Binnendifferenzierung geboten?
    \item Wo findet eine Binnendifferenzierung statt; sowohl im Inhalt als auch den Methoden? Ist die Wahl der Methoden aus der Didaktik zu begründen? Welche alternative Methoden wären möglich gewesen? Werden unterschiedliche Methoden für unterschiedliche Lerngruppen Angeboten?
    \item Handhabbarkeit, Anwendbarkeit, praktisches, pragmatisch. Handwerkliche Betrachtung
    \item Umfang, Zeitvorgaben, zur Verfügung Stellung realistisch?
    \item Unwahrscheinlich: Aber, ist das Material Altersgruppen geeignet oder gar übergriffig?
    \item Welche Beeinflussungen sind zu erkennen? Lassen die sich diese legitimieren?
\end{itemize}
Darüber hinaus soll untersucht werden, inwieweit sich Intentionen der Arbeitnehmerkammer Bremen im Material finden lassen und ob die Beeinflussungen des Akteurs sich in einem Rahmen bewegen, welcher der Intention des \gls{bbk} nicht entgegensteht. \#Kontroversitätsgebot 


\subsection{Bildungsplan als Analysemaßstab}
Der Bildungsplan \autocite{bplan} in Politik für duale Studiengänge im Land Bremen...

Wie zu Kapitelanfang angedeutet ist der Bildungsplan vergleichsweise offen formuliert, ohne detaillierte Vorgaben zu machen. Angesichts der überbordenden Themenauswahl und interdisziplinären Verstrickung eines Faches wie Politik eine konsequente Entscheidung, die im Bildungsplan selbst argumentativ legitimiert wird \autocite[diggah, welche Seite habe ich das gelesen]{bplan}. Reinhold \textcite[17-18]{Hedtke2016} kritisiert für den Wirtschaftsunterricht den fehlenden Bezug auf die Fachwissenschaften sowie auch eine fehlende Berücksichtigung der zahlreichen Interdependenzen zu benachbarten Fachwissenschaften. 




\subsection{Quellenangaben}
\begin{itemize}
    \item Wie lässt sich die Qualität der Quellen bewerten?
    \item Wie wird auf Medienkompetenz eingegangen?
 \end{itemize}




\section{Analyse der Unterrichtsvorschläge \label{Analyse}}
\paragraph{Hinweis}
Das Material befindet sich in der Entwicklung  und ist noch nicht veröffentlicht. Es unterliegt daher noch zahlreichen Veränderungsoptionen. 
Ein beispielhaftes Layout der Materialkarten ist im Appendix auf \acrlong{S} \pageref{ANKPrototyp} zu sehen. Dahinter folgen die Entwürfe zu den Unterrichtsvorschlägen. Es wird das Material zu den zwei Themenfeldern \emph{Demokratie} (ab \gls{S} \pageref{DEMOKRATIE-A1}) und \emph{Arbeitsleben} (ab \gls{S} \pageref{ARBEITSLEBEN-A1}) untersucht. 

\noindent Eine Analyse ohne die Aufgaben und deren Material zu kennen, ist vermutlich nicht gewinnbringend nachvollziehbar. Die Seitenzahlen im PDF sind klickbar und führen in den Anhang. 
\bigskip
% Es wird das Material zu den drei Themenfeldern \emph{Demokratie} (ab \gls{S} \pageref{DEMOKRATIE-A1}), \emph{Arbeitsleben} (ab \gls{S} \pageref{ARBEITSLEBEN-A1}) und \emph{Wirtschaftspolitik} (ab \gls{S} \pageref{WIRTSCHAFTSPOLITIK-A1}) untersucht. \\


Das Material der \gls{ank} ist direkt den sieben im Bildungsplan vorgegebenen \enquote{politischen Handlungsfeldern} \autocite[3, 15]{bplan} zugeordnet.
Obgleich im Bildungsplan explizit formuliert ist, dass durch inhaltliche Offenheit \enquote{der Einsicht entsprochen [wird], dass eine konsensfähige Festlegung relevanter, zeitloser Inhalte weder fachwissenschaftlich noch fachdidaktisch begründbar ist} \autocite[15]{bplan}, wird verlangt, dass vier der sieben \enquote{politischen Handlungsfelder}, darunter \enquote{Demokratie} \enquote{verpflichtend}, bearbeitet werden.

Die Unterrichtsvorschläge folgen einem einheitlichen Aufbau. Zur Orientierung in dieser Arbeit wird in der Regel das \enquote{Themenfeld} genannt, das den \enquote{politischen Handlungsfeldern} aus dem Bildungsplan direkt entspricht. 
Die einzelnen Materialkarten sind in der Form \emph{Themenfeld A\#} angegeben. 
Der Buchstabe kennzeichnet dabei ein Thema innerhalb des Handlungsfeldes/Themenfeldes.
Die Nummerierung ergibt sich in der Regel aus den Phasen \emph{Einstieg}, \emph{Erarbeitung} und \emph{Auswertung/Übertrag}, welche die Materialien eines Themas in zumeist drei mögliche Unterrichtszeiteinheiten aufteilen. So gibt es zu \enquote{Arbeitsleben} die drei Themen A, B und C mit je zwei bis drei Unterrichtszeiteinheiten.
Zu jeder Unterrichtszeiteinheit gibt es eine Leitfrage. % , welche sich innerhalb der Phasen ändern kann, aber nicht muss. 

Die Kompetenzen, die im Material mitaufgeführt sind, finden sich ebenfalls genauso im Bildungsplan und teilen sozusagen das Themenfeld nur genauer ein. Sie bleiben genau wie der als \enquote{Gegenstand der Auseinandersetzung} bezeichnete Überblickstext über die oben angegeben Phasen gleich (also für einen Buchstaben über alle Nummern hinweg). 

% Wer mit dem Blick eines Lehrer*innen-Klischees 
% Mit Blick auf die direkte Übernahme der Struktur des Bildungsplans auf das Material schaut, könnte sich (wenn womöglich auch unterbewusst) fragen, was das Abschreiben soll und ob da nicht eine präzisere Einteilung sein müsste.
Das Zuordnen in die \enquote{politischen Handlungsfelder} und die feinere Verknüpfung mit den Kompetenzen des Bildungsplans erleichtert durch die Vermeidung unnötiger Kompliziertheit eine Einbettung in eine den Bildungsvorgaben entsprechende Grobplanung des Unterrichts, die ohnehin erwartet werden kann.

% Die Kompetenzen in Kompetenzbereiche einzuordnen, ist sicherlich sinnvoll, um einen Überblick zu haben, innerhalb welches Kompetenzbereiches exemplarisch gefördert wird und wo im Bildungsplan man bereits geübt hat und welche anderen Kompetenzen daher in Zukunft womöglich eher Aufmerksamkeit bedarfen. 


% Eine präzisere Ausdifferenzierung wird in diesem Material dann nicht über eine angepasste Kompetenzformulierung erreicht, sondern über das ausformulierte Thema und die Leitfragen gewährleistet.

Da für eine einzelne Unterrichtseinheit jedoch kaum in Anspruch genommen werden kann, einen Kompetenzbereich hinreichend abzudecken -- auch exemplarisch, denn es ist ausgeschlossen, nicht exemplarisch zu lernen -- wäre es womöglich ergänzend sinnvoll, noch konkreter anzugeben, welche exakten Kompetenzen erworben werden.
In dem Fall der Materialkarten geschieht dies nun nicht über eine konkretere Ausdifferenzierung der Kompetenzen, sondern über die \enquote{Leitfrage}, das \enquote{Thema} und \enquote{Keywords}. 


\paragraph{Erwartungshorizont}
Ein Erwartungshorizont ist bei dem Material nicht mitgeliefert. Insbesondere relevant wird ein Erwartungshorizont\footnote{
    Hilfreich kann es sein einen Erwartungshorizont der Zielgruppe anzupassen, im besten Falle dann noch binnendifferenziert, da davon auszugehen ist, dass jede Zielgruppe heterogen ist. 
    Zielgruppenorientierung ist idealerweise über die gesamte Unterrichtsvorbereitung und entsprechend im Unterrichtsmaterial mitzudenken. Gerade Materialvorschläge, die zur Verbreitung gedacht sind und nicht mit einer einzelnen Kohorte im Blick entwickelt wurden, können zwar Vorschläge zur Staffelung beinhalten. Die Zielgruppe kann dann aber sinvollerweise nur nach \enquote{hardfacts} wie Altersspanne, wahrscheinliche Lernorte und Fortschritt im Bildungssystem (Jahrgang \gls{etc}) adressiert werden.} 
 bei Prüfungsaufgaben.
Um die Kritik konstruktiv zu gestalten, wird in dieser Arbeit jedoch versucht einen pragmatischen, einfachen Erwartungshorizont, also eine Art Lösung zu ergänzen. In der Unterrichtspraxis könnte so etwas das Abgleichen beschleunigen und eine weitere Arbeitserleichterung für die Lehrkräfte darstellen. 

Da die Aufgaben des Materials meist verhältnismäßig kurz sind und zum Lernen und nicht direkt zum Werten und Messen dienen, wird auf eine weitere Binnendifferenzierung verzichtet. Bei Aufgabenstelleungen, deren Antworten sich nicht eindeutig aus dem Quellmaterial ergeben, auf welches die Aufgaben verweisen, sondern die sich erst wesentlich im Partner*innen-, Gruppen-, oder Unterrichtsgespräch ergeben, wird ebenfalls auf eine spekulative Lösung verzichtet. 

% Ein exemplarischer Erwartungshorizont oder gar einer -- welcher verschiedene Abstufungen, von einem Minimum, um dem Kompetenzbereich gerecht zu werden, bis hin zu einem Ausblick, wo das eingegrenzte Thema überschritten wird, aber wo man weiter lernen könnte -- wäre eine hilfreiche Ergänzung, um die Vorbereitung der Lehrkräfte zu erleichtern. 


\paragraph{Operationalisierte Arbeitsaufträge}
In den Aufgabenstellungen der Materialkarten wird in der Regel mit Operatoren gearbeitet. Der Bildungsplan für Politik an Berufsschulen selbst hat zwar keine Liste der Operatoren \autocite{bplan}. Aber es ist üblich, in der schulischen, politischen Bildung mit Operatoren zu arbeiten. Die \textcite[14-18]{KMK.2005} gibt für Abiturprüfungen in Politik an, welche Operationalisierungen durch ihren Imperativ die jeweilige Schüler*innenaktivität hervorbringen sollen und kategorisiert diese in drei Anforderungsbereiche. Im Bildungsplan für Politik an Gymnasien in Bremen sind die Vorgaben der \gls{kmk} entsprechend mit einer Operatorenliste umgesetzt \autocite[13-14]{lower2008}.
Die im vorliegenden Material genutzten Operatoren sind vergleichbar. 

\subsection{Demokratie \label{Denmokratie}}
Für den Bereich Demokratie sind bisher zwei Themen angeboten:
\begin{myenumerate}
    \item Demokratie A: \enquote{Föderalismus und Arbeitsschutz – Welche Chancen und Risiken bringt das Mehrebenensystem der BRD für die Sicherheit und Gesundheit am Arbeitsplatz mit sich?}
    \item Demokratie B: \enquote{Wahlen – von welchem Teil des Volkes geht eigentlich die Staatsgewalt aus?}
\end{myenumerate}
% Für beide Bereiche sind je drei Karten mit jeweils ausdifferenzierten Leitfragen vorhanden. Die drei Karten sind in der Struktur 
% \enquote{Einstieg},
% \enquote{Erarbeitung} und
% \enquote{Auswertung/Übertrag} 
% bezeichnet. 



\subsubsection{Demokratie A - \enquote{Föderalismus und Arbeitsschutz – Welche Chancen und Risiken bringt das Mehrebenensystem der BRD für die Sicherheit und Gesundheit am Arbeitsplatz mit sich?} \label{DemokratieA}}

Für den Themenbereich \enquote{Demokratie A} (A1 bis A3) wird als Kompetenzbereich angegeben:
\begin{quote}
    Die Schüler:innen sind in der Lage das Wesen der Demokratie, die Strukturen und Organisationen des politischen Systems sowie die Mechanismen politischer Willensbildung einzuordnen und zu beurteilen.
    
    \autocite[im Bildungsplan:][16]{bplan}
\end{quote}


% Unter den drei Karten für \enquote{Demokratie A} grenzt das Thema: \enquote{Föderalismus und Arbeitsschutz – Welche Chancen und Risken bringt das Mehrebenensystem der BRD für die Sicherheit und Gesundheit am Arbeitsplatz mit sich?} konkreter ein, was durch das folgende Material und die Aufgaben an Kompetenzen vermittelt werden soll. 
Die Leitfragen
\begin{myenumerate}
    \item Demokratie A1: \enquote{Welche Erfahrungen mit Arbeitsschutz habt ihr bisher gemacht?} (\gls{S} \pageref{DEMOKRATIE-A1})
    \item Demokratie A2: \enquote{Wer kümmert sich um den Arbeitsschutz?} (\gls{S} \pageref{DEMOKRATIE-A2})
    \item Demokratie A3: \enquote{Pro und Contra Föderalismus oder warum ist das Teilen von Macht ein Strukturprinzip demokratischer Herrschaft?} (\gls{S} \pageref{DEMOKRATIE-A3})
\end{myenumerate}
teilen das Thema dabei in die drei Karten auf.
Ein passend formulierter Schwerpunkt im Bildungsplan dazu ist: \enquote{Prinzipien des Föderalismus, mit Blick auf Bremen} \autocite[16]{bplan}.


VIELLEICHT ALLGEMEIN ÜBER BESCHISSENEN VERSIONSVERLAUF VON GESETZEN REDEN?

Gesetze unterliegen zahlreichen Änderungen. In aller Regel werden beschlossene Änderungen von Bundesgetzen in einem \gls{bgbl} veröffentlicht  

% HIERRRRRRR FEHLER.TEX EINFÜGEN

Beim \gls{bmas} findet sich tatsächlich eine nette Übersicht vom \enquote{Referenzentwurf} über den \enquote{Kabinettsbeschluss} bis zum \enquote{Abschluss des Gesetzes}.
Alle Schriftstücke sind, wie es sein sollte, verlinkt. Der Referenzentwurf vom \gls{bmas} ist dabei einfach brav als PDF anzutreffen \autocite{BMAS-21.07.2020}. 
Die Drucksache des Bundestages hingegen ist kopiergeschützt, wtf \autocite{Bundestag.31.08.2020}?
Das Bundesgesetzblatt ist zwar ein eher nur maschinenlesbarer Link, aber hat immerhin einen netten Viewer mit ziemlich guter Suchfunktion integriert und ist damit nicht kopiergeschützt \autocite{BGBl.2020-I-Nr67}. 

Gesetze sind komplex und kompliziert. Umso wichtiger mag es erscheinen, in Material, welches für die Verbreitung gedacht ist, nicht auf Korrektheit zu verzichten -- auch nicht im Tausch gegen Verständlichkeit.





Der Einführungstext (\gls{S} \pageref{DEMOKRATIE-A1}) stellt mehrere Daten dar, ohne diese weiter zu belegen.
Die angesprochenen Gestze sind immerhin noch eindeutig zu identifizieren. Das Grundgesetz ist als \enquote{(GG Art. 83/84) ($\rightarrow$ Grundgesetz; $\rightarrow$ Föderalismus)} aufgenommen und schon ohne die \enquote{Keywords} gut aufzufinden. Wobei auch hier ein Bezug auf die Version angebracht wäre. Auch das Grundgesetz wird immer mal wieder geändert.  

Weiter wird ein Gesetz in folgender Form genannt: 
\begin{quote}
    Das 2021 erlassene Bundesgesetz \enquote{Durchführung von Maßnahmen des Arbeitsschutzes zur  Verbesserung der Sicherheit und des Gesundheitsschutzes der Beschäftigten bei der Arbeit} ($\rightarrow$ Bundesgesetze)
\end{quote}

In der Angabe ist dankbarerweise ein Datum angegeben, welches jedoch leider irreführend ist. Das angesprochene \gls{arbschg}, ist 1996 erlassen worden\footnote{Stand 18.06.2025 ist folgendes die aktuelle Version: \enquote{Arbeitsschutzgesetz vom 7. August 1996 (BGBl. I S. 1246), das zuletzt durch Artikel 32 des Gesetzes vom 15. Juli 2024 (BGBl. 2024 I Nr. 236) geändert worden ist}.}. Die betreffende Änderung ist ebenfalls nicht 2021, sondern noch rechtzeitig 2020 erlassen worden. Das Inkraftreten der maßgeblichen Änderung für die im Text angesprochenen \enquote{Mindestbesichtigungsquote} ist durch eine Ergänzung in Absatz 1 und das Anfügen von Absatz 1a in §21 % \footnote{
%    Die Änderung war Teil des: \enquote{Gesetz zur Verbesserung des Vollzugs im Arbeitsschutz (Arbeitsschutzkontrollgesetz) Vom 22. Dezember 2020} veröffentlicht durch das \enquote{Bundesgesetzblatt Jahrgang 2020 Teil I Nr. 67, ausgegeben zu Bonn am 30. Dezember 2020}.} 
durch Artikel 1 ist nach Artikel 11 nun allerdings tatsächlich am 01.01.2021 \autocite[3334–3343]{BGBl.2020-I-Nr67}. 
% des Bundesgesetzblattes 2020 I Nr. 67

Mein Verbesserungsvorschlag zur Korrektur und Vereinheitlichung mit der Kurzangabe des Grundgesetzes ist \gls{zb} folgende Formulierung:
\begin{quote}
Zu 2021 wurde das Arbeitsschutzgesetz\footnote{
    \emph{Arbeitsschutzgesetz} vom 7. August 1996 Änderung durch \emph{Gesetz zur Verbesserung des Vollzugs im Arbeitsschutz (Arbeitsschutzkontrollgesetz)} Vom 22. Dezember 2020 veröffentlicht im \emph{Bundesgesetzblatt Jahrgang 2020 Teil I Nr. 67, ausgegeben zu Bonn am 30. Dezember 2020} \gls{S} 3334–3343.\label{ArbschSchGfooty}} 
    geändert, sodass es jetzt eine Mindestbesichtigungsquote enthält. % vorsieht. 
    Bis 2026 sollen mindestens 5\,\% aller Betriebe abgedeckt sein (\gls{arbschg} §21 (1a))($\rightarrow$ Bundesgesetze). 
\end{quote}

Auch wenn es wonöglich mit minimalistischen Layout Vorstelleungen kollidieren mag, ist zu überlegen, ob es im Bildungsbereich nicht angebracht ist, Fußnoten \gls{oä} für hinreichende Quellenangaben einzubinden (wie \gls{zb} in Fußnote \ref{ArbschSchGfooty}). 


Später im Text wird angegeben, dass \enquote{die Quote bei nur etwas über zwei Prozent [liegt], wodurch Betriebe im Schnitt erst alle 40 bis 50 Jahre überprüft werden.} Leider ebenfalls ohne Quelle.



\paragraph{Erwartungshorizont: Demokratie A1 Einstieg} (\gls{S} \pageref{DEMOKRATIE-A1})

\textsc{Aufgabe 1} a) \quad
Mindestens das Wort \enquote{Arbeitsschutz} und die Möglichkeit UV-Strahlung bedingten Krankheiten vorzubeugen sollten erwähnt sein. 
\\

\textsc{Aufgabe 1} b) \quad
\Gls{zb} Sonnenschutzmittel stellen (Sonnencreme \gls{etc}), persönlichen Sonnenschutz stellen (Kopfbedeckungen, passende Kleidung), Sonnenschutz durch bauliche Maßnahmen \gls{oä} (Fensterglas mit UV-Schutz, Sonnensegel, beschattete Pausenmöglichkeit), verantwortliche Personen bestimmen, welche auf Sonnenpause und persönlichen Sonnenschutzmaßnahme achten. 
Besonders spannend wäre, das Auftragen von Sonnenschutz explizit in der Arbeitszeit stattfinden zu lassen. 
\\

\textsc{Aufgabe 2} a) \quad
Angebotsvorsorge bei mindestens einer Stunde draußen arbeiten (außerhalb des Winters) und das Stellen einer Betriebsärzt*in sollte erwähnt sein. 
\\

\textsc{Aufgabe 2} b) \quad
Die Beschäftigungsfähigkeit als Voraussetzung, Gesellschaft zu gestalten und nicht in prekäre Lebenssituationen abzurutschen, könnten erwähnt werden. 
Aufklärung könnte als notwendige Voraussetzung genannt werden, um diese Ziele zu erreichen.
\\

\textsc{Aufgabe 3} \quad
Mindestens ein Aspekt sollte erwähnt werden. \Gls{zb} \enquote{Mir wurde Arbeitsschutz bisher noch nie im Betrieb bewusst gemacht.}
Wenn das der Fall sein sollte, sollte die Verbindung gezogen werden, dass wahrscheinlich auch im eigenen Betrieb Prävention passieren könnte. 
Ansonsten könnten die gemachten Erfahrungen als Aufklärung (Erinnerung an die Bedeutung von regelmäßigem Aufstehen und Bewegen bei Büroarbeit \gls{etc}) oder konkrete Pävention identifiziert werden (\gls{zb} Schutzausrüstung, Stehschreibtisch \gls{etc}).



\paragraph{Demokratie A1 Einstieg: \enquote{Welche Erfahrungen mit Arbeitsschutz habt ihr bisher gemacht?} \label{DemokratieA1}} 

Die Aufgaben 1 und 2 (\gls{S} \pageref{DEMOKRATIE-A1}) beziehen sich auf zwei Texte aus \emph{M1} und \emph{M2} und operationalisieren schriftliche Einzelarbeitsaufträge in den Aufgabenbereichen II - III durch \emph{erläutern}, \emph{skizzieren} und \emph{erklären}.
Aufgabe 3 operationalisiert mit \emph{erläutern} und \emph{diskutieren} \gls{me} doppelt und im falschen Aufgabenbereich. Die Operatoren sind zwar selbst in unterschiedlichen Anforderungsbereichen (II bis III), aber ich würde argumentieren, dass die geforderten Tätigkeiten von \emph{erläutern} ohnehin unter Voraussetzung für \emph{diskutieren} subsumiert werden können und daher eher verwirren:
\begin{myitemize}
    \item Diskutieren: \enquote{Zu einem Sachverhalt, zu einem Konzept, zu einer Problemstellung oder zu einer These etc. eine Argumentation entwickeln, die zu einer begründeten Bewertung führt} 
    \item Erläutern: \enquote{Wie erklären, aber durch zusätzliche Informationen und Beispiele verdeutlichen}
    \item Erklären: \enquote{Sachverhalte durch Wissen und Einsichten in einen Zusammenhang (Theorie, Modell, Regel, Gesetz, Funktionszusammenhang) einordnen und deuten} 

    \autocite[][13-14]{lower2008}
\end{myitemize}
Andererseits mag es helfen, durch die kleinschrittigere Strukturvorgabe die Arbeitsanweisungen zu gliedern. Dann wäre es jedoch womöglich hilfreich, allgemein mehr Strukturvorgaben in die Aufgabenstellung einzubauen. Der Auftrag, seine eigenen Erfahrungen mit Arbeitsschutz darzustellen, ist in erster Linie eine Aufzählung und die Sortierung von Erinnerungen. Im zweiten Schritt können die Erinnerungen dann mit den Implikationen aus \emph{M1} und \emph{M2} in Zusammenhang gebracht werden. 
Das könnte kleinschrittig wie folgt aussehen:
\begin{quote}
    \emph{Zähle} deine Erfahrungen mit Arbeitsschutz an deinem Arbeitsplatz \emph{auf}.

    \emph{Erläutere} deine Erfahrungen mit Arbeitsschutz in Bezug auf die Implikationen aus \emph{M1} und \emph{M2}. 
\end{quote}

Da diese Phase als \enquote{A1 Einstieg} mit 15$\,^{\prime}$ Dauer veranschlagt ist, wäre von der Lehrkraft ohnehin die Transferleistung gefordert, den Schüler*innen einen Hinweis zu geben, dass die Aufgaben 1 bis 3 jeweils nur mit ein bis drei Sätzen beantwortet werden sollten.

Den Abschluss bildet die Vier-Ecken-Methode. Sie wird erklärt mit: \enquote{Ihr geht zu der Ecke, die Eurer Ansicht am ehesten entspricht}.
Die vier vorgeschlagenen Ecken beschreiben jedoch eher nach Level gegliederte Erfahrungen in Bezug auf Sicherheitsvorkehrungen. Wie man sich zuordnen soll, bleibt ein wenig unklar. Eine konkretere Formulierung wäre: % $\rightarrow$ Umformulierungsvorschlag:
\begin{quote}
    Ihr geht zu der Ecke, die Euren gemachten Erfahrungen am ehesten entspricht. 
\end{quote}

Besonders die Vier-Ecken-Methode bringt für einen Einstieg auch die hilfreiche körperliche Aktivierung mit sich und erzwingt damit ein unvermeidbares Mindestmaß an Aufmerksamkeit.  

Eine präzise auf die Lernziele des Einstiegs bezogene Kompetenz könnte folgendermaßen formuliert werden:
\begin{quote}
    Die Schüler*innen können ihre Erfahrungen mit Arbeitsschutz nennen und grob mit dem Regelwerk der Bundesrepublik Deutschland in Beziehung setzen. 
\end{quote}



\paragraph{Erwartungshorizont: Demokratie A2 Erarbeitung} (\gls{S} \pageref{DEMOKRATIE-A2})

\textsc{Aufgabe 1} a) \quad Mögliche Phasen, mindestens fünf: 
\begin{myenumerate}
    \item Einführung (?)
    \item Industrialisierung, Arbeitsschutz beginnt Thema zu werden % (Dampfmaschine, Webstuhl), Kinderarbeit
% 1839 Preußen Kinder >9 y, bis 16 y maximal 10h pro d
% Arbeitgeber muss Verschulden nachgewiesen werden

% 1872 Verein zur Überwachung der Dampfkessel (Vorläufer TÜV)
% 1883 Gesetz Krankenversicherung der Arbeiter
% 1884 Unfallversicherungsgesetz, Berufsgenossenschaften
% 1890 Arbeitsschutzkonferenz
% 1891 keine Sonntagsarbeit in Industrie, Kinder unter 13 nicht in Fabriken arbeiten
% staatliche Gewerbeaufsicht
    \item Erster Weltkrieg, Arbeitsschutz wird zurückgedreht
% August 1914 Doppelschichten mit 12h Arbeit, auch wieder Sonntagsarbeit
    \item Weimarer Republik, Arbeitsschutz wird ausgebaut
% Unfallvertrauensmänner \& Sicherheitsingenieure
% Arbeitsschutzkampagnen, Werbeplakate
% Rat der Volksbeauftragen führt 8h Tag ein. War Forderung der Arbeiterbewegung
% Franz Kafka ist Arbeitsschutz Fetischist
    \item Machtübernahme der Nationalsozialisten, Deutsche ArbeitsFront (DAF) ersetzt Gewerkschaften
    \item Zweiter Weltkrieg, erneut Rückbau des Arbeitsschutzes
    \item Nachkriegszeit, technischer und sozialer Arbeitsschutz wird fortentwickelt, viele Normen
% wieder auch Werbekampagnen
    \item Arbeitssicherheitsgesetz 1973, Betriebsärzte und Berater für Arbeitssicherheit 
    \item Arbeitsschutzgesetz 1996, Gefährdungsbeurteilung, Prävention, Unterweisung
    \item 2013: psychische Belastungen werden ins Arbeitsschutzgesetz aufgenommen
\end{myenumerate}

\textsc{Aufgabe 1} b) \quad
\begin{myitemize}
    \item \emph{Arbeitsbedingungen} waren schlecht.
    \item \emph{Arbeiter*innen} haben Forderungen gestellt.
    \item \emph{Gesetze} regeln zunehmend die Bedingungen von Arbeit und nehmen vermehrt Arbeitgebende in die Pflicht. 
    \item \emph{Verbesserungen} werden über Zeit schrittweise erzielt und in den Weltkriegen wieder ausgesetzt. 
\end{myitemize}


\textsc{Aufgabe 2} \quad
\begin{myitemize}
    \item § 3 Grundpflichten des Arbeitgebers
    \item Die Bundesregierung
    \item Die zuständigen Landesbehörden
    \item Die Gewerbeaufsicht des Landes Bremen (Staatliche Aufsichtsbehörde) der Senatorin für Gesundheit, Frauen, Verbraucherschutz (oberste Landesbehörde im Arbeitsschutz mit Dienst- und Fachaufsicht)\footnote{\url{https://www.gesundheit.bremen.de/gesundheit/arbeitsschutz-46085} (29.06.2025)}
\end{myitemize}


\textsc{Aufgabe 3} a) \quad
Aufgabe 3 ist noch in Bearbeitung. \emph{M3} steht noch aus.
\\

\textsc{Aufgabe 3} b) \quad
In Bezug auf den Einführungstext kann die schwankende aber tendenziell angestiegene Besichtigungsquote erwähnt werden. 2022 lag sie bei etwa 4,8\,\% ($1100 \div 23000 \approx 0,0478$). Das \emph{Gewerbeaufsichtsamt} hat damit fast das nach \gls{arbschg} §21 (1a) angestrebte Mindestniveau von 5\,\% erreicht. 
\\

\textsc{Aufgabe 3} c) \quad
Das ein \emph{Bundesg}esetz von \emph{Landes}behörden -- im Falle Bremens von der \emph{Gewerbeaufsicht} -- umgesetzt werden muss, indem die einzelnen Betriebe im \emph{Land} kontrolliert werden, ist ein Beispiel für Politikverflechtung im Föderalismus. 



\paragraph{Demokratie A2 Erarbeitung: \enquote{Wer kümmert sich um den Arbeitsschutz?}}
Aufgabe 1 (\gls{S} \pageref{DEMOKRATIE-A2}) bezieht sich auf ein Video, in welchem die Entwicklung des Arbeitsschutzes in Deutschland beschrieben wird. 
Dadurch wird Abwechslung in der Medienauswahl erreicht, indem nicht nur geschriebener Text, sondern gesprochenes Wort in Verbindung mit Bildern als audiovisuelle Komposition angeboten wird
% Dadurch wird Abwechslung in der Medienauswahl erreicht, indem nicht nur Symbolsprache geschriebener Text, sondern Symbolsprache gesprochenes Wort in Verbindung mit Bildern als audiovisuelle Komposition angeboten wird.

In \textsc{Aufgabe 1} a) wird Abwechslung durch Partner*innenarbeit geschaffen.
Es wird wieder vielseitig mit Operatoren gearbeitet, die Aufgabe richtet sich aber hauptsächlich auf das Wiedergeben und Ordnen der Inhalte des Videos.

In \textsc{Aufgabe 1} b) soll \enquote{\emph{Arbeiter:innen}} verwendet werden. Handlungsfähigkeit und Selbstwirksamkeit der \enquote{\emph{Arbeiter:innen}} wird lediglich an einer Stelle im Video deutlich -- als der Rat der Volksbeauftragten den 8h Tag auf Druck der Arbeiterbewegung einführt. Das Video fokussiert sich eher auf die Geschichte der Umsetzung von Arbeitsschutz durch Recht und beleuchtet die gesellschaftlichen Umstände als Missstände, ohne weiterzugehen und möglichen Arbeiskampf zu thematisieren. % SCHREIB DAZU DOCH WAS IN DIE AUFGABENWERTRUNG

\textsc{Aufgabe 2} ist ein kurzer Rechercheauftrag zu verschiedenen Paragraphen des Arbeitsschutzgesetzes, der mit Hilfe des Internets erledigt werden soll. Damit wird wieder das Medium gewechselt und Internetrecherche zu Gesetzen angestoßen. 
Hier besteht die Möglichkeit, über das Üben von Internetrecherchen hianus auch das Angeben von (Internet-)Quellen zu üben und, womöglich im Plenum, die Überprüfung der Vertrauenswürdigkeit von Quellen näher zu beleuchten. Gesetze sind nicht trivial und auf diversen Websites erwähnt. Es gibt neben verschiedenen staatlichen Quellen auch private Anbieter. Grundsätzlich ist das Internet mit verschiedenen Suchmaschinen, deren ausgeklügelten Algorithmen und seit einer Weile auch der Aufdrängung von Ergebnissen mit generativer \gls{ki} durch \gls{llm}.

\textsc{Aufgabe 3} a) \emph{M3} ist noch nicht eingefügt. Die Aufgabe zielt aber wieder auf Aufgabenbereich 1 ab. Ein Text wird gelesen und bestimmte Informationen sollen eigenständig wiedergegeben werden. 



\paragraph{Erwartungshorizont: Demokratie A3 Auswertung/Übertrag} (\gls{S} \pageref{DEMOKRATIE-A3})

\textsc{Aufgabe 1} a) \quad
Es sollten etwa vier von diesen Vor- und Nachteilen erwähnt sein -- Kausalketten für die ähnlichen Vor- und Nachteile müssen nicht genauso auftauchen:
 \begin{myitemize}
    \item Aufteilung von Macht $\rightarrow$ Gegenseitige Kontrolle $\rightarrow$ Geringeres Risiko von Machtausnutzung
    \item lokale Expertise $\rightarrow$ viele, \gls{uu} verschiedene Ideen
    \item Ähnliche Aufgaben werden womöglich mehrmals gemacht $\rightarrow$ das ist teuer
    \item uneinheitliche Regelungen $\rightarrow$ anderes Bundesland, man muss womöglich neue Regeln lernen
 \end{myitemize}

\textsc{Aufgabe 2} a) \quad
Eine Erläuterungs-Möglichkeit wäre, dass durch die ortsspezifischen Umsetzungen verschiedene Ideen parallel evaluiert werden können. 
Wettbewerb als Chance sollte erwähnt werden. 
\\

\textsc{Aufgabe 2} b) \quad
Es wird empfohlen, das Konservieren und Angeben von Quellen im Rahmen dieser Aufgabe zu thematisieren.
Es bietet sich an, Hinweise zum Auffinden von Gesetzen zu geben, \gls{zb} könnte Grundgesetz Artikel 73 erwähnt werden.
\\

\textsc{Aufgabe 3} \quad
---


\paragraph{Demokratie A3 Auswertung/Übertrag: \enquote{Pro und Contra Föderalismus oder warum ist das Teilen von Macht ein Strukturprinzip demokratischer Herrschaft?}}
Aufgabe 1 (\gls{S} \pageref{DEMOKRATIE-A3}) hat nur Teil a) und ist damit uneinheitlich zu \gls{zb} Aufgabe 3, wo keine Teilaufgabe angegeben ist. Das könnte verwirrend sein, ist jedoch wohl dem Entwurf-Zustand des Materials geschuldet. 

Wie schon in Demokratie A1 ist in Aufgabe 2 wieder eine Internetrecherche-Aufgabe enthalten. Wenn das Material konsequent benutzt wird, bieten solche fortlaufenden Arbeitsaufträge, mit ähnlicher Struktur, die Möglichkeit zu üben und Routinen zu etablieren. Das hat dann auch die Chance   Recherche- und damit verbundene Quellen-, also Medienkompetenzen auszubauen und die Effizienz ähnlicher Aufgabentypen zu steigern.

Aufgabe 3 bringt wieder körperliche Aktivierung und bietet zum Vergleich von vorher zu nachher eine anschauliche Möglichkeit, Unterschiede, die durch eine Diskussion entstehen, oder auch ausbleiben können, darzustellen. 



\subsubsection{Demokratie B - \enquote{Wahlen – von welchem Teil des Volkes geht eigentlich die Staatsgewalt aus?}}
Der am Bildungsplan orientierte Kompetenzbereich für den zweiten Lehrvorschlag % unter dem Thema: \enquote{Wahlen – von welchem Teil des Volkes geht eigentlich die Staatsgewalt aus?}, 
lautet: \enquote{Die Schüler:innen sind in der Lage unterschiedliche demokratische Entscheidungsverfahren zu reflektieren und zu beurteilen.} \autocite[][16]{bplan}. 

Die Leitfragen dazu sind: 
\begin{myenumerate}
    \item \enquote{Was denken eure Mitschüler:innen über das Thema Wahlen?} (\gls{S} \pageref{DEMOKRATIE-B1})
    \item \enquote{Wer geht eigentlich wählen und warum ist das ein Problem?} (\gls{S} \pageref{DEMOKRATIE-B2})
    \item \enquote{Warum wird in einer Demokratie gewählt?} (\gls{S} \pageref{DEMOKRATIE-B3})
\end{myenumerate}
Als Schwerpunkt ließe sich
\enquote{Möglichkeiten der politischen Beteiligung und demokratischen Willensbildung} identifizieren \autocite[][16]{bplan}.


\paragraph{Erwartungshorizont: Demokratie B1 Einstieg} (\gls{S} \pageref{DEMOKRATIE-B1})
\textsc{Aufgabe 1} \quad \& 
\textsc{Aufgabe 2} \quad Zur Darstellungsform ist die technische Ausstattung der Schule zu beachten. Womöglich bietet es sich hier an, das Arbeiten mit Tabellenkalkulationsprogrammen und Präsentationssoftware aufzugreifen.  Eine Wiederholung zu Säulendigrammen und Bruch- \gls{bzw} Prozentrechnung könnte angebracht sein. 
\\


\paragraph{Demokratie B1 Einstieg: \enquote{Was denken eure Mitschüler:innen über das Thema Wahlen?}}  (\gls{S} \pageref{DEMOKRATIE-B1})
Die Durchführung einer Umfrage in Aufgabe 1 ist durch die Teilaufgaben relativ kleinteilig gegliedert und mit Hinweisen versehen. Es könnte sinnvoll sein, eine Antwortoption in Richtung: \emph{Ich habe noch nie wählen können, habe es aber vor, sobald ich kann} miteinzubauen. Auch wenn viele Schüler*innen über 16 Jahre alt oder gar volljährig sind, kann es durch die Wahlperioden durchaus vorkommen, dass diese noch nicht wählen konnten. 
Der Einfachheit halber ließe sich das mit \emph{Ich gehe wählen/möchte wählen gehen} \& \emph{Ich gehe nicht wählen/möchte nicht wählen gehen} in die bestehenden zwei Antwortmöglichkeiten integrieren. 

\enquote{Welches Ergebnis, welche Antworten habt ihr so erwartet?} aus Aufgabe 2 sollte womöglich schon in Aufgabe 1 auftauchen und vor der Durchfühurng beantwortet werden. Die Kompetenzen, welche in dieser Aufgabe liegen, könnten mit einer derartigen Herangehensweise strukturell  einem wissenschaftlichen Prozess weiter angeglichen werden, bei dem Hypothesen geprüft werden. 
Natürlich beinhaltet der wissenschaftliche Prozess idealerweise eine gewisse iterative Abwechslung: Von der Formulierung eines Erkenntnisinteresses, welches möglicherweise durchaus induktiv, vom Besonderen auf das Allgemeine, schließen möchte, um daraus Hypothesen zu formulieren, welche dann wiederum eher deduktiv versuchen, einen allgemeingültigen Satz zu falsifizieren. Durch die starke Strukturvorgabe von derartigen Aufgaben in der Schule wird ein solcher Prozess nur näherungsweise dargestellt. Den Blick auf die Meta-Ebene nicht zu verlieren und das Meiste rauszuholen, schadet an dieser Stelle aber bestimmt nicht. 

Als eine Aufgabe, die aus dem Klassenraum und dem Kursverband herausführt, bietet diese Aufgabenstellung Abwechslung und führt leicht spielerisch an Umfragen und Statistik heran. 
\\



\paragraph{Erwartungshorizont: Demokratie B2 Erarbeitung:}  (\gls{S} \pageref{DEMOKRATIE-B2})
\\
\textsc{Aufgabe 1} a) \& \textsc{Aufgabe 1} b)
\begin{myitemize}
    \item Karte 1 Wahlbeteiligung pro Ortsteil in \%. Je höher, desto dunkler eingefärbt
    \item Karte 2: Arbeitslosenquote pro Ortsteil in \%. Je höher, desto dunkler eingefärbt 
    \item Arbeitslosenquote niedriger $\rightarrow$ Wahlbeteiligung höher. Die Karten ergänzen sich ein wenig wie ein Puzzle
\end{myitemize}

\textsc{Aufgabe 2} a) \quad Ein Mensch und eine übergroße, stilisierte Wahlurne sind zu sehen. Der Mensch zählt mehrere Dinge auf, die wichtig sind, Wählen ist nicht dabei.

\textsc{Aufgabe 2} b) \quad Zum Vergleichen und Einordnen von möglicherweise über den Zusammenhang von Arbeitslosigkeit und Nicht-Wählen hinaus gehenden Ergebnissen könnte die Studie zu Nichtwählenden der Friedrich-Ebert-Stiftung herangezogen werden \autocite[]{Hagemeyer.2023}. Darin wird zwischen sechs Typen unterschieden: Die Zurückgekehrten, Die Vergesslichen, Die Berechnenden, Die Krisengestressten, Die Gleichgültigen \& Die Wütenden \autocite[11-13]{Hagemeyer.2023}\footnote{Die Website zur Studie wird auch auf der Website mit \emph{M3} verlinkt.}. 
\\

\textsc{Aufgabe 3} a) \quad
\enquote{Je höher die \emph{Arbeitslosenquote}, desto niederiger ist die \emph{Wahlbeteiligung} innerhalb eines \emph{Stadtteils}.}
\\

\textsc{Aufgabe 4} \quad
Für die Einordnung möglicher Gründe \gls{s} bei Aufgabe 2 b).



\paragraph{Demokratie B2 Erarbeitung: \enquote{Wer geht eigentlich wählen und warum ist das ein Problem?}}  (\gls{S} \pageref{DEMOKRATIE-B2})
Die Trennung in Teilaufgaben a) und b) in der ersten Aufgabe ist etwas gedoppelt. Wer Teilaufgabe a) \enquote{Notiere was du dort alles herausfinden kannst.} gewissenhaft durchführt, wird dann auch b) schon beantwortet haben.

In Aufgabe 1 b) und Aufgabe 3 a) (\gls{S} \pageref{DEMOKRATIE-B3}) wird sich auf die gleiche Korrelation von Wahlbeteiligung zu Arbeitslosenquote bezogen
-- einmal für die Bürgerschaftswahl in Bremen 2023 und einmal für die Bundestagswahl 2021\footnote{
    In Bremen für 88 Ortsteile (5 davon ohne Daten), für das Bundesgebiet sind 979 Stadtteile die Bezugseinheit.}. 
Diesen Bezug explizit herzustellen wird durch die Aufgabenstellungen nicht angestoßen. Es ist von der Fähigkeit der Schüler*innen oder der Lehrkraft abhängig. Es könnte sinnvoll sein, diesen Zusammenhang nicht nur implizit im Raum stehen zu lassen.

Die Karikatur aus \emph{M2} hat keinen direkten Bezug zu den \enquote{Strukturindikatoren}\footnote{
    Zu \emph{M1} gibt es noch weitere Möglichkeiten Karten anzeigen zu lassen. Wie \gls{bspw} \enquote{Abiturienten/-innen der allgemeinbildenden Schulen in Bezug auf einen Durchschnittsjahrgang der Bevölkerung zwischen 17 und 21 Jahren}. Auch dort ist der Korrealtionskoeffizent mit $r = 0,7$ hoch (Wahlbeteiligung zu Arbeitslosenquote $r = -0,8$).
} wie der Arbeitslosigkeit. 

Aufgabe 3 hat wieder nur eine Teilaufgabe. 



\paragraph{Erwartungshorizont: Demokratie B3 Auswertung/Übertrag}  (\gls{S} \pageref{DEMOKRATIE-B3})

\textsc{Aufgabe 1} a) \quad
Sowohl \emph{bing} als auch \emph{google} geben in ihrer \gls{ki}-Zusammenfassung einen Enzyklopädie-artigen Beitrag von der Website \emph{Profession Politische Bildung} des \emph{\gls{bap}}, gefördert von der \emph{\gls{bpb}}, als Quelle an. Dort lässt sich herauslesen, dass Selbstwirksamkeit der Glauben daran ist, dass die eigenen Handlungen einen Unterschied machen \autocite[]{Hufer.2022}. In der Google-Suche wird \gls{zb} auch für die deutschen Ergebnisse die Wikipedia-Seite\footnote{
    \url{https://en.wikipedia.org/w/index.php?title=Political_efficacy} 02.07.2025} 
zu \emph{political efficacy} herangezogen. Dort wird der Zusammenhang zu \emph{political responsiveness} gezogen und in \emph{interne} und \emph{externe} \emph{political efficacy} unterschieden. Wobei die \emph{interne} dem Glauben an das eigene Verständnis von Politik und die Möglichkeit zu Partizipieren entspricht, wohingegen sich die \emph{externe} darauf bezieht, ob Repräsentant*innen es tatsächlich schaffen, die Interessen der Bürger*innen durchzusetzen. 
% \enquote{\enquote{generalisierte Erwartung, Handlungsfolgen selbst unter Kontrolle zu haben} (Neyer/Assendorpf 2018, 191)}
\\

\textsc{Aufgabe 1} b) \quad
Gerade mit der vorhergehenden Einheit \emph{Demokratie B2} bietet es sich an, auf mögliche Faktoren für ein geringes Selbstwirksamkeits-Empfinden wie \emph{Arbeitslosigkeit}, \emph{Bildungsferne} oder \emph{Armut} zu sprechen zu kommen. 
\\

\textsc{Aufgabe 2} a) \quad
Hinweis: Das Interview ist in M1, auch wenn in der Aufgabe \emph{M2} steht. 
Das Interview wird aber in der Präsentation (M2) zusammengefasst. Die Hauptaussagen von \emph{M1} und \emph{M2} sind: 
\begin{myitemize}
    \item Mitbestimmung im Betrieb lässt Selbstwirksamkeit erfahren.
    \item In Betriebsräten werden politische Fähigkeiten trainiert. (Forderungen formulieren, Verhandeln, öffentlich Sprechen)
    \item Das kann sich auch auf die \enquote{parlamentarische Demokratie} auswirken (\emph{spill-over Effekt})
    \item Menschen mit mehr Geld und Ressourcen wählen eher als Menschen mit wenig.
    \item Menschen ohne deutsche Staatsbürgerschaft dürfen lediglich kommunal wählen. % und arbeiten zusaätzlich häufig in prekäreren Verhältnissen. 
    \item Wenig politische Partizipation ist für die demokratische Legitimation bedenklich. 
    \item Frauen sind durch Einbindung in Care-Arbeit weniger in Betriebsräten vertreten. 
    \item Lösungsansätze: 
    \begin{myitemize}
        \item Betriebliche Mitbestimmung stärken
        \item Schnellere Einbürgerungen 
        \item In Betriebsräten dürfen auch Menschen ohne deutsche Staatsangehörigkeit miteintscheiden. 
        \item Repräsentation in den Entscheidungsgremien von Frauen und Menschen mit Migrationsgeschichte ausbauen.
    \end{myitemize} 
\end{myitemize}



\textsc{Aufgabe 2} b) \quad
\emph{Politische Selbstwirksamkeit} wird durch Betriebsräte gestärkt. Dadurch wählen die Menschen eher und auch eher in ihrem Sinne. 
Das verringert Zustimmung zu extremen und menschenfeindlichen Positionen. 
\\

\textsc{Aufgabe 2} c) \quad
\begin{myitemize}
    \item Frauen sind insbesondere durch zeitliche Eingebundenheit in Care-Arbeit seltener in Betriebsräten vertreten.
    \item Menschen in Betriebsräten wird häufig wenig Dankbarkeit entgegengebracht.
    \item Sie müssen jedoch viele verschiedene Themen unter wachsendem Zeitdruck abarbeiten. 
    \item Sie gestalten aber demokratisches Miteinander im Betrieb, können damit Arbeitsbedingungen verbessen und stellen durch die Selbstwirksamkeitserfahrungen im Betrieb die Weichen für Teilhabe an der parlametarischen Demokratie. 
\end{myitemize} 


\textsc{Aufgabe 2} d) \quad
Positive selbstwirksame Erfahrungen erhöhen die Wahrscheinlichkeit, sich zu beteiligen. 
\\

\textsc{Aufgabe 3} \quad
Tipps für Argumente:
Pro:
\begin{myitemize}
    \item Es ist gerechter
    \item Es erpart die Schwierigkeit \enquote{mehr Ahnung} definieren zu müssen. Die Definition selbst wäre von Entscheidungen und Meinung gefärbt
    \item Auch Menschen mit \enquote{wenig Ahnung} müssen mit den Entscheidungen leben 
    \item $\rightarrow$ Demokratische Legitimation steigt 
    \item Meinungen von Menschen, die strukturell bedingt \enquote{wenig Ahnung} haben (können), werden bei \enquote{Expert*innen-Wahl} leichter übergangen
\end{myitemize}
Contra:
\begin{myitemize}
    \item Es wird womöglich weniger aus emotionalisierter Beeinlussung heraus gewählt
    \item Die Chance, dass mehr nach Inhalten gewählt wird, könnte steigen
    \item Personen-Wahl nach Sympathie könnte einen geringeren Einfluss haben 
    \item Es gäbe die Chance, dass Ressourcen für einen intensiven, aber Realitäten verkürzt darstellenden, Wahlkampf eine geringere Rolle spielen
    \item Komplexe Themen haben eine geringere Chance durch populistische Vereinfachung zu Entscheidungen zu führen, die zahlreiche Auswirkungen nicht berücksichtigen 
\end{myitemize}
%     \item Ansichten von unterrepräsentierten Gruppen könnten bei ausgeprägten Mehrheiten wenig Berücksichtigung finden (kann auch bei wenig Wählenden stattfinden) 

\paragraph{Demokratie B3 Auswertung/Übertrag: \enquote{Warum wird in einer Demokratie gewählt?}} 
Aufgabe 1 (\gls{S} \pageref{DEMOKRATIE-B3}) verlangt wieder eine Internetrecherche und weist dieses Mal direkt auf die Notwendigkeit hin, alle Quellen anzugeben. Da \gls{ki}-Antworten von mehreren verbreiteten Suchmaschinen ganz oben platziert werden, ist die Kompentenz der Lehrkräfte gefordert, einen guten Umgang damit beizubringen. 

In \emph{M1} von \emph{Arbeitsleben C2} erwähnt die Interviewte Martina Zandonella eine Studie zu \emph{political responsiveness} \autocite[]{Elsasser.2017}. Die mögliche Verbindung von einem geringen Empfinden politischer Selbstwirksamkeit zu tatsächlich nachgewiesener geringerer Responsivität für finanziell schlechter Gestellte \autocite{Elsasser.2017} könnte an dieser Stelle hervorgehoben werden. Es könnte ein Hinweis gegeben werden, dass die Materialien von \emph{Demokratie B3} und \emph{Arbeitsleben C2} gut zusammenpassen oder eine weiteren Aufgabe, welche den Zusammenhang expliziter werden lässt. 

Innerhalb von \emph{M2} ist nur ein Harvard-Style Kurzverweis im Text zu finden, ohne am Ende eine Folie mit den kompletten Literaturverweisen anzubieten\footnote{
    Das zweite zitierte Werk ist \gls{zb} wahrscheinlich dieses: \textcite[]{Probst.2022}.}. 

Bei den Tipps zu Argumenten für Aufgabe 3 fiel bei der Erstellung der Tipps auf, dass viele Argumente, wenn sie empirisch untersucht würden, auch gegenteilige Ergebnisse liefern könnnten. Daher wurde viel im Konjunktiv und um \emph{Möglichkeiten} und \emph{Chancen} herum formuliert. Mit dem roten Faden der \enquote{Wahrheitsbegriffe}, die diese Arbeit durchziehen, könnte an dieser Stelle gut auf präzise Formulierungen eingegangen werden.
Übermäßiges Benutzen von Konjunktiv kann sprachlich ungeschickt und kompliziert sein. Aber ein Bewusstsein über sprachliche Möglichkeiten, auch insbesondere für die politische Kommunkikation, ist im Schulunterricht richtig platziert. 



\paragraph{Handlungsfeld \emph{Demokratie} -- Ausblick} % Zusammenfassung und 
Die aus dem Bildungsplan übernommene Kompetenzformulierung: \enquote{Die Schüler:innen sind in der Lage das Wesen der Demokratie, die Strukturen und Organisationen des politischen Systems sowie die Mechanismen politischer Willensbildung einzuordnen und zu beurteilen}
ist, wie auch in \gls{abs} \ref{Analyse} (\gls{S} \pageref{Analyse}) erläutert, durch das Thema: \enquote{Föderalismus und Arbeitsschutz - Welche Chancen und Risiken bringt das Mehrebenensystem der BRD für die Sicherheit und Gesundheit am Arbeitsplatz mit sich?} eingegrenzt. 

Eine Eingrenzung ist zwar zwingend notwendig, um überhaupt handlungsfähig ein so komplexes Thema wie \emph{Demokratie} angehen zu können, dennoch wäre es im Rahmen des Handlungsfeldes Demokratie auch spannend gewesen, eine kontroversere Betrachtungsweise der bestehenden Systeme zu explorieren. Auch zum Thema Föderalismus ließe sich in Frage Stellen, inwiefern dieser als eine Form der Machtverteilung tatsächlich zu einem \enquote{Mehr} an Demokratie führt oder ob andere, bürokratieärmere Systeme, nicht wesentlicher sein könnten. 
% weniger Regelwerk aufblähende Stellschrauben, nicht wesentlicher wären. 

Im Grundgesetz ist der Föderalismus eindeutig festgeschrieben (insbesondere \gls{zb} Artikel 70 oder Artikel 83 \& 84). % und auch Roland \textcite[397]{Sturm.2021} erwähnt in seinem Zeitschriftenartikel mit dem einschlägigen Titel \citetitle{Sturm.2021} -- wie auch \emph{M2} von \emph{Demokratie A3} (\gls{S} \pageref{DEMOKRATIE-A3}) -- Winfried Kretschmann als Befürworter des Föderalismus. 
Allgemein lesen sich der Subtext von \emph{Demokratie A} oder auch der Beitrag mit dem einschlägigen Titel \citetitle{Sturm.2021} von Roland \textcite{Sturm.2021} als positiv gegenüber dem Föderalismus\footnote{
    Zufälligerweise erwähnen sowohl \emph{Demokratie A3} (\gls{S} \pageref{DEMOKRATIE-A3}), als auch \textcite[397]{Sturm.2021}  Winfried Kretschmann im Zusammenhang mit Föderalismus.
}. Aber \citeauthor{Sturm.2021} konstatiert auch: \enquote{Deutschland ist ein föderaler Staat ohne Anhänger des Föderalismus geworden} \autocite[397]{Sturm.2021} und schreibt zur Berichterstattung während der Corona-Krise: \enquote{Dass in der öffentlichen Diskussion ein Beratungs- und Koordinierungsgremium als Essenz des Föderalismus wahrgenommen wurde, sagt viel über die Qualität der politischen Bildung}. Aus seiner Sicht ist die Beschäftigung mit den Chancen des Föderalismus an (Berufs-)Schulen daher wahrscheinlich eine sinnvolle. 

Die Verhältnisse verstehen zu lernen, bevor sie als veränderbar und \emph{nicht(!)} alternativlos herausgestellt werden, mag als angebracht empfunden werden. Insbesondere im Hinblick auf das Ziel einer durch Rückhalt des Volkes legitimierten Demokratie und einer gewissen Mitwirkungspflicht der Bildungsinstitutionen zum Erreichen dieses Ziels.

Mit der begrenzten Zeit, die der politischen Bildung zur Verfügung steht, stellt sich jedoch die Frage, ob nicht ein konfliktorientierter Unterricht, der den Status quo herausfordert, ebenfalls gewinnbringend sein könnte. Andere Möglichkeiten, wie direktere Demokratie (wenn auch eine wenig praktizierte Alternative), welche auch eine Form von \enquote{vertikaler Gewaltenteilung} darstellen könnte, böte bei der \enquote{Dominanz von Parteien in der deutschen Parteiendemokratie} \autocite[401]{Sturm.2021} jedenfalls reichlich Konfliktpotential. 
% Andere Möglichkeiten wie direktere Demokratie als mögliche, wenn auch wenig praktizierte Alternative, welche auch eine Form von \enquote{vertikaler Gewaltenteilung} darstellen könnte, böte bei der \enquote{Dominanz von Parteien in der deutschen Parteiendemokratie} \autocite[401]{Sturm.2021} jedenfalls reichlich Konfliktpotential. 

Kontroverse Betrachtungen des Status quo sind dann mit dem darauffolgenden Thema Wahlen in \emph{Demokratie B} schon im Material selbst angelegt. Die Legitimation von Demokratie durch ungleiche Wahlbeteiligung und ungleiche Repräsentation macht eine Konfliktlinie auf und fordert damit auch den aktuellen Zustand heraus. Die Transferleistung mit Rückbezug auf den Föderalismus ist aber von den Schüler*innen selbst zu vollbringen oder von der Lehrkraft außerhalb der vorgezeichneten Aufgaben des Materials anzustoßen. 

Aufgegriffen wird das Konflikthafte von fehlender Repräsentation dann auch in einigen Materialkarten (\gls{zb} C2) des nächsten Handlungsfeldes \emph{Arbeitsleben}. Partizipationsmöglichkeiten auf der betrieblichen Ebene und deren Zusdammenhang mit \emph{Demokratie} werden behandelt. 



\subsection{Arbeitsleben}
Es sind drei Themen für das Handlungsfeld Arbeitsleben vorbereitet:
\begin{myenumerate}
    \item \enquote{Arm trotz Arbeit – ist das gerecht?}
    \item \enquote{Entgrenzte Arbeit – Haben wir bald nie wieder Feierabend?}
    \item \enquote{Mitbestimmen oder nur dabei sein?}
\end{myenumerate}



\subsubsection{Arbeitsleben A -- \enquote{Arm trotz Arbeit – ist das gerecht?}}
Der Kompetenzbereich aus dem Bildungsplan ist: 
\begin{quotation}
    Die Schüler:innen sind in der Lage berufliche und gesellschaftliche Lebenssituationen vor dem Hintergrund der sich verändernden wirtschaftlichen und politischen Rahmenbedingungen zu erkennen, um adäquate individuelle Entscheidungen treffen zu können.

    \autocite[18]{bplan}
\end{quotation}

Die Leitfragen teilen das Thema in A1 und A2 auf: 
\begin{myenumerate}
    \item \enquote{Was ist Einkommensarmut und wie kann sie erklärt werden? Welche Leistungen kann ich in Bremen beantragen, wenn ich von Einkommensarmut betroffen bin?} (\gls{S} \pageref{ARBEITSLEBEN-A1})
    \item \enquote{Wie viel verdienen Arbeitnehmer:innen in Bremen und welche (politischen und wirtschaftlichen) Maßnahmen würden die Situation am Arbeitsmarkt verbessern?}  (\gls{S} \pageref{ARBEITSLEBEN-A2})
\end{myenumerate}

Der Einführungstext ist in weiten Teilen ein Zusammenschnitt von Zitaten aus \emph{M1}. Das wird aktuell leider noch nicht deutlich. Gerade bei harten Fakten, wie Zahlen für Statistiken, sollten Quellen vernünftig angegeben sein. Denn je nach Erhebungsmethode sind die Zahlen entsprechend nicht vergleichbar oder können für -- bewusst oder unbewusst -- irreführende Darstellungen ge- oder missbraucht werden. In \emph{M1} von der \gls{ank} wird genau so ein Umstand sogar hervorgehoben\footnote{
    \gls{vgl} \enquote{Achtung: Aufgrund einer Umstellung bei der Erhebung des Mikrozensus sind die Daten zu Einkommensarmut und Mindestsicherung vor 2020 nicht mit den Daten nach 2020 vergleichbar!} \url{https://www.arbeitnehmerkammer.de/statistik/soziale-lagen.html} (04.07.2025) \label{MikrozensusNichtVgl.bar2020}}. 
In \emph{M1} selbst werden die Quellen zwar genannt, aber leider in einer Form, dass bei einer Überprüfung noch viel eigene Recherche notwendig wäre\footnote{
    Die Quellen sind mit: \enquote{Amtliche Sozialberichterstattung des Bundes und der Länder} oder \enquote{Statistische Ämter des Bundes und der Länder, Agentur für Arbeit} angegeben. Direkte Angaben zur jeweiligen Veröffentlichung und zum Fundort sind dann leider nicht mehr enthalten. Als gesetzlich festgeschriebene Institution ist die \gls{ank} zwar vergleichsweise vertrauenswürdig. Aber auch vertrauenswürdigen Menschen können Fehler unterlaufen, weshalb Angaben überprüfbar sein sollten.}.



\paragraph{Erwartungshorizont: Arbeitsleben A1 Einstieg} (\gls{S} \pageref{ARBEITSLEBEN-A1})
\textsc{Aufgabe 1} Teil 1 \quad
\begin{myitemize}
    \item Von Einkommensarmut betroffen sind Menschen, die weniger Einkommen als 60\,\% des Medianeinkommens haben (kann sich je nach Haushaltsgröße unterscheiden). 
    \item Arbeitslos gemeldete Menschen, müssen noch nicht einkommensarm sein. In der Theorie können sie \gls{bspw} Einkommen aus anderen Quellen beziehen (\gls{idr} Vermögen, wie Immobilien oder Aktien) und wären entsprechend auch nicht leistungsberechtigt. In der Praxis werden derartige Privatiers sich aber nicht arbeitslos melden. Arbeitslose sind daher meist auch Arbeitslosengeld Beziehende, eine Form der Mindestsicherung. 
    \item Unter Mindestsicherung wird nicht nur Arbeitslosengeld, sondern \gls{zb}auch Aufstockungsbeträge für Erwerbstätige, Rentern*innen oder Grundsicherung für nicht arbeitsfähige Menschen verstanden. 
\end{myitemize}

\textsc{Aufgabe 1} Teil 2 \quad
Die im Interview angesprochenen Punkte sind:
\begin{myitemize}
    \item Löhne sind nicht ausreichend gestiegen
    \item abnehmende Tarifbindung
    \item Konkurrenz um knappen Wohnraum mit Wohlhabenderen
    \item Häufig Frauen und Menschen mit Migrationsgeschichte ohne Kapital, mit nur ihrer Arbeitskraft
    \item Kluft zwischen Arm und Reich wächst
    \item Wohnen und Vermögensaufbau ließe sich politisch verändern
    \item Gewerkschaften haben Probleme
    \item Viele Menschen arbeiten alleine für sich verantwortlich (\gls{zb} Honorarkräfte). Das erschwert solidarische Organisation.
\end{myitemize}

Darüber hinaus könnten Kausalketten hervorgehoben werden, wie \gls{bspw}
\begin{myitemize}
    \item geringerer Organisationsgrad und geringeres gewerkschtafliches Engagement $\rightarrow$ weniger Arbeitskampf und Solidarität $\rightarrow$ Tarifbindung nimmt ab
    \item Berufswahl von kulturell bedingt schlechter bezahlten Berufen $\rightarrow$ Care Berufe sind \gls{bspw} in Verbindung zu patriarchalen Strukturen traditionell schlechter bezahlt -- Körperliche Berufe mit geringerem Organisationsgrad der Beschäftigten sind traditionell gerne mit Menschen aus migrantischen Strukrturen besetzt (\gls{zb} Baugewerbe, Reinigung, Lieferdienste)
    \item Care Arbeit wird traditionell eher von Frauen übernommen
    \item Care Arbeit öffnet die Türen zur \enquote{Teilzeitfalle}
\end{myitemize}




\textsc{Aufgabe 2} \quad
Bis zu 1200\,€ Brutto kann ein Betrag anrechenungsfrei bleiben. Dieser setzt sich in jeweils unterschiedlichen Anteilen zusammen: Die ersten 100\,€ dürfen zu 100\,\% behalten werden, von den nächsten 420\,€ bleiben 20\,\%, von den folgenden 480\,€ bleiben 30\,\% und von den letzten 200\,€ sind es 10\,\%. Das Maximum, welches anrechnungsfrei bleibt, wenn man 1200\,€ oder mehr brutto verdient, ist also: 100\,€ + 84\,€ + 144\,€ + 20\,€ = 348\,€ \autocite[59-62]{MerkblattSGBII}. 

Das lässt sich überblicksweise auch in vier linearen Funktionen, die jeweils nur für einen Bereich gelten, ausdrücken:
\begin{align} 
f(x)& = x    & 
    D_{f}& = \{0<x<100\}    \\
f(x)& = 0,2x + 80 & 
    D_{f}& = \{100<x<520\}  \\
f(x)& = 0,3x + 28 & 
    D_{f}& = \{520<x<1000\} \\
f(x)& = 0,1x + 228 & 
    D_{f}& = \{1000<x<1200\}
\end{align}
% Dieses Mal schreibe ich direkt in den Erwartungshorizont meinen Kommentar. 
Zur grafischen Darstellung \gls{vgl} auch die \gls{abb} \ref{Absetzungsbetrag} (\gls{S} \pageref{Absetzungsbetrag}).



\paragraph{Arbeitsleben A1 Einstieg -- \enquote{Was ist Einkommensarmut und wie kann sie erklärt werden? Welche Leistungen kann ich in Bremen beantragen, wenn ich von Einkommensarmut betroffen bin?}}  (\gls{S} \pageref{ARBEITSLEBEN-A1})
Die Lehrkraft sollte ihre Kohorte gut genug kennen, um abschätzen zu können, inwieweit eine Wiederholung der Konzepte von Median und Durchschnitt angebracht ist. Hier bietet sich die Chance auf Implikationen für Einkommen oder Vermögen zu sprechen zu kommen, wo es in der Regel nach oben um einen größeren Faktor Ausreißer gibt, weshalb dort häufig der Median und nicht der Durchschnitt als Grundlage genommen wird. Durchschnittseinkommen oder Vermögen liegt zumeist weit über dem Median. Diese Sicherheit im Verständnis von Statistik bietet für Medienkompetenz und das Erkennen von \enquote{framing} eine gute Grundlage.

Das Wissen über die Gemeinsamkeiten und Unterschiede von Einkommensarmut, Arbeitslosigkeit und Mindestsicherung ist in gewisser Weise schon vor dem Zusammentragen im Plenum notwendig. 
Die dritte Grafik von \emph{M1} könnte sonst \gls{bspw} missverständlich sein. Die \enquote{Quote Arbeitslosigkeit} ($10,6\,\%$) addiert mit der \enquote{Quote Mindestsicherung} ($17,6\,\%$) ergäbe ($28,2\,\%$), das entspräche fast der \enquote{Quote Einkommensarmut} ($28,8\,\%$). Das Hintergrundwissen, dass als arbeitslos gemeldete Menschen gleichzeitig unter Mindestsicherungsempfänger*innen subsumiert werden, ist hier zum Verständnis notwendig. Der Unterschied zwischen Mindestsicherung und Einkommensarmut lässt sich dann damit erklären, dass auch nicht als arbeitsfähig geltende Menschen Leistungen zur Mindestsicherung erhalten können. Erst damit wird der in \emph{M1} relevante Unterschied zwischen Menschen in Mindestsicherung zu Menschen in Einkommensarmut nachvollziehbar. Gewissermaßen ist diese Voraussetzung des Verständnisses auch gleichzeitig das Ziel der Aufgaben. Da eine Erkenntnis optimalerweise von den Schüler*innen selbst konstruiert wird, ist das Ansetzen an eine bestimmte Erkenntnis von verschiedenen Punkten aus gewissermaßen Best Practice. Im vorliegenden Fall ist der obige Hinweis also wieder mehr eine Möglichkeit Fehlvorstellungen bereits in der Vorbereitung im Blick zu haben, um möglicherweise \emph{scaffolding} bieten zu können, als eine wirkliche Kritik an den getroffenen Entscheidungen. 

Es wird nicht deutlich, wo bei den \enquote{Triggerpunkten} pro und contra liegen sollen. Soll sich dafür ausgepsrochen werden, dass Care Berufe schlechter bezahlt werden und Menschen mit Migrationsvordergrund\footnote{
    \Gls{bspw} spricht die schwarze Sängerin Melane auf instagram vom Migrationsvordergrund. Da es zwischen dem migrierten weißen Menschen, wo der Migrationshintergrund nicht direkt sichtbar wird und einer braunen Person, die ihre Haut und ihren Namen kaum verstecken kann, wesentliche Unterschiede gibt. Migrationshintergrund ist als Begriff derartig von rechtsextremistischen Narrativen misbraucht worden, wo es eindeutig um Othering geht, dass ich mich dieser argumentativ gut begründbaren Begriffswahl gerne anschließen möchte.  
} bei Lieferando und Uber prekär beschäftigt sind? 

In \emph{M2} selbst ist leider kein Datum angegeben. Aus dem Text heraus ist eine Nähe zur Corona Zeit herauszulesen, sowie der Bremer Landesmindestlohn von 12\,€. Damit wäre ein Zeitraum vor 2022\footnote{
    \url{https://www.senatspressestelle.bremen.de/pressemitteilungen/senat-stimmt-fuer-anhebung-des-landesmindestlohns-auf-12-29-euro-392437} (06.07.2025)
} und nach 2020 anzunehmen. Ein Blick in die \gls{url} scheint das zu bestätigen, da es sich danach um die \enquote{ausgabe-juliaugust-2021} des auch gedruckt veröffentlichten Magazins der \gls{ank} handelt. Eigentlich eine Lapalie, das Datum des Magazins nicht um den Text herum genannt zu haben, aber dadurch, dass \gls{zb} auf sehr veränderbare Zahlen, wie eben den Landesmindestlohn, Bezug genommen wird, ist eine zeitliche Einordnung relevant. Erst so haben auch Leser*innen mit weniger Hintergrundwissen, eine größere Chance darauf gestoßen zu werden, dass die angesprochenen Fakten im Kontext der Zeit zu sehen sind. 



Es können in \textsc{Aufgabe 3} mehrere Ebenen unterschieden werden, die für Fehler und Missverständnisse sorgen können:

\begin{myitemize}
    \item Die Aufgabe kann pauschal nicht beantwortet werden. Der Bruttolohn für den noch Mindestsicherung zugezahlt wird ist von vielen Faktoren abhängig und auch wenn die meisten der Exemplarität wegen weggelassen werden, bleibt immer noch der individuell berechnete Posten für Wohnraum und Nebenkosten. 
    \item Die Website der Arbeitsagentur gibt falsche Informationen an und sichert sich nicht ab, das wird unten vertieft.
    \item Die Gesetzgebung an sich ist kompliziert und ausufernd. 
    \item Aufgrund der Komplexität wird der einseitige Infoflyer nur für sehr eingeschränkte Fälle möglich sein. Der Infoflyer der Agentur für Arbeit hat voraussichtlich nicht aus der Intention der Schikane 100 Seiten. 
\end{myitemize}
Insbesondere die ausufernde Aufgabenstelleung ließe sich entschärfen, indem Einschränkungen getroffen werden.
Die naheliegendsten, um die Aufgabenstellung bearbeitbarer zu machen wären \gls{bspw}:
\begin{myitemize}
    \item Alleinstehende und -lebende, volljährige Person mit deutscher Staatsbürgerschaft
    \item Keine Kinder
    \item Mietwohnung
    \item Nur Einkommen aus Arbeit
    \item Keine anerkannte Behinderung
\end{myitemize}
Diese Einschränkungen haben natürlich die unangenehme Nebenwirkung, dass sie diskriminierend wirken, indem auf von einer empfundenen Norm abweichendere Leben nicht eingegangen wird. Das lässt sich zwar mit der Notwendigkeit eines didaktisch reduzierten Vorgehens begründen, ist aber nicht elegant. 
Das Spannungsfeld um die Reduzierung der Aufgabe ist ein gutes Beispiel für einen der Gründe, weshalb Regelungen, die für alle gleichermaßen gelten sollen, in einer komplexen Welt ebenfalls ausufernd und komplex werden. Wie viel Zeit eine derartige  Betrachtung der Metaebene im Unterricht einnehmen könnte oder womöglich sollte, wird an dieser Stelle nicht weiter exploriert. 


Die Wesbite der Arbeitsagentur, also \emph{M3}, lässt sich in mehreren Hinsichten kritisieren: 

\begin{myenumerate}
    \item Es geht aus \emph{M3} nicht direkt hervor, weshalb der Absetzungsbetrag 223\,€ beträgt. % oder ob sich dieser je nach Einkommen individuell errechnet. 
    Eine Formel (\gls{bzw} vier) zur Berechnung ist nicht direkt aufzufinden (\gls{vgl} den Erwartungshorizont). 
    Es könnte sogar das Missverständnis entstehen, der Absetzungsbetrag betrüge immer 223\,€.
    
    Erst im Downloadbereich findet sich der hundertseitige Infoflyer zum SGB II in dessen Kapitel 9.2 wird nach etwa 60 Seiten %\footnote{\url{https://www.arbeitsagentur.de/datei/merkblatt-buergergeld_ba043375.pdf} (08.07.2025)} 
    erläutert, was anrechnungsfrei bleibt \autocite[59-62]{MerkblattSGBII}. 
    
    Immerhin beim Download Bereich ist der \enquote{Hinweis: Bitte beachten Sie den Stand der Dokumente.} (\gls{vgl} \emph{M3}) angegeben; was zum nöchsten Punkt führt:
    \item Die Zahlen aus dem Rechenbeispiel sind 2025 falsch. Es findet sich kein Hinweis auf den Stand des Rechenbeispiels.
    
    Im \gls{sgb} XII ist eine Tabelle der Regelbedarfsstufen zu finden \autocite[Anlage zu §28]{Bundestag.2003}. Aus dieser lässt sich schließen, dass das Beispiel mit 502\,€ \enquote{für jede erwachsene Person, die in einer Wohnung nach § 42a Absatz 2 Satz 2 lebt und für die nicht Regelbedarfsstufe 2 gilt}, nur im Jahr 2023 galt \autocite[Anlage zu §28]{Bundestag.2003}. 2025 sind der Regelbedarf 563\,€ \autocite[Anlage zu §28]{Bundestag.2003}.

    Optimalerweise wären die Zahlen des Rechenbeispiels direkt an die in den jeweiligen veröffentlichten Paragraphen veröffentlichten Zahlen gekoppelt. Wobei auch dort die Herkunft angegeben sein sollte. Da das allerdings utopisch erscheinen mag\footnote{
        \dots obwohl sich, wie im Erwartungshorizont gezeigt, auch die Bestimmungen in textlastigen Gesetzen als Formeln ausdrücken lassen. Was \gls{bspw} an der durch die Formeln und ChatGPT erstellten \gls{abb} \ref{Absetzungsbetrag} (\gls{S} \pageref{Absetzungsbetrag}) zu sehen ist. \Gls{vgl} dazu auch die Tabelle \ref{KIHilfsmittel}, \gls{S} \pageref{KIHilfsmittel} und den eingegebenen Prompt -- es ließ sich in ChatGPT sogar der Python-Code zu der Erstellung des Plots anzeigen. 
        
        Wenn es eine Sache gibt, die in der Natur von Code liegt, ist es Berechnungen durchzuführen. Es ist keine weitere Innovation notwendig, um so etwas auf Websites zu implementieren. Außerdem hätten vergleichbare Praktiken für staatliche Websites, richtig umgesetzt, neben der Qualitätsverbesserung die Chance weniger Arbeit in Zukunft zu verursachen. 
    }, folgender Hinweis wie sich das Ganze auch mit reinem Text verbessern ließe:
    
    Direkt unter der Überschrift \enquote{Beispiel für die Berechnung des Ergänzungsbetrags} könnte stehen:

    % Aber auch dort ließe sich mit einer präzisen Formulierung vorbeugen: Die Höhe des Regelbedarfes ist im SGB XII unter der Anlage zu §28 zu finden (Stand 2023). Dieses Gesetz ist unter: \url{https://www.gesetze-im-internet.de/sgb_12/} (Stand: Datum) zu finden.

    Stand des Beispiels für 2023. Die Zahlen ändern sich regelmäßig. Der aktuelle Regelbedarf ist in der Anlage zu §28 des \emph{SGB XII}\footnote{
        Das Hervorgehobene könnte im Einklang mit dem Rest der Website dann als klickbarer Link gestaltet sein. Die Agentur für Arbeit könnte dazu nachfragen, wie stabil die direkte \gls{url} \url{https://www.gesetze-im-internet.de/sgb_12/BJNR302300003.html} ist, oder ob lieber nur \url{https://www.gesetze-im-internet.de/sgb_12/} angegeben werden sollte und sich die Menschen leider noch durcklicken müssen (beide 10.07.2025).} zu finden.
\end{myenumerate}

Die folgende Grafik veranschaulicht wie viel Geld in Abhängigkeit zum Bruttolohn beim Empfang von Arbeitslosengeld für eine kinderlose Person einbehalten werden darf. Es ist auch gezeigt, wie sich die vier Funktionen mit einer einzelnen approximieren ließen. Letzten Endes ließen sich aber auch vier Funktionen so einbinden, dass die Zahlen nach Gesetzesänderungen akutell bleiben. Weiter unten wird auf den vermutlich nicht willkürlichen Übergang nach 520\,€ eingegangen. Solche Abhängigkeiten besser einzubinden, würde viel Arbeit einsparen können. 
\begin{figure}[h!]
    \centering
    \includegraphics[width=1\linewidth]{Absetzungsbetrag.png}
    \caption{Absetzungsbetrag in Abhängigkeit zum Einkommen laut den Angaben aus \emph{M3} und des dort verlinkten Merkblattes \autocite[59-62]{MerkblattSGBII}.}
    \label{Absetzungsbetrag}
\end{figure}

Allgemein täte es einigen staatlichen Informationsangeboten gut, sich mehr auf ihre tatsächlichen Quellen zu berufen. Nicht nur für die Bürger*innen, sondern auch für die Behörden selbst. Sich direkt auf die Gesetze und den gemeinten Stand zu berufen sichert ab.
% In der Aufgabe wird vor allem eines evident: % Dass Zahlen zu Leistungsempfangenden gut zu finden sind und Formeln zur Berechnung der Höhe der möglichen Leistungen besser unter Verschluss gehalten sind, als manche Geheimnisse. 

Das wird auch an diesem Einzelfallbeispiel evident. Die eigentlich zur Hilfe gedachte Seite (\emph{M3}), lässt sich ohne den langen Flyer hinzuzuziehen nicht ausreichend nachvollziehen, ist neben komplex auch noch falsch und sichert sich fast nicht ab. 
% Es ist so, als ob die Aufgabe das Ziel hat zu zeigen, wie hochkomplex das Verlangen nach Rechtssicherheit und Kompromissen soziale Sicherungssysteme werden lässt.

Da die 520\,€ Grenze, nach der sich der Faktor wieder ändert, wieviel Einkommen zusätzlich zur Mindestsicherung behalten werden darf, mit der ehemaligen Gerungfügigkeitsgrenze übereinstimmt, sei noch erwähnt, dass auch die Daten aus dem Merkblatt vermutlich ebenfalls veraltet sind \autocite[59-62]{MerkblattSGBII}. Im Merkblatt selbst sind auch keine Quellen angegeben. Der Minimal-Aufwand zur eigenen Absicherung ist ein Hinweis auf \url{www.gesetze-im-internet.de} ganz vorne. \emph{Stand 2023} dazuzuschreiben, oder tatsächlich die jeweiligen Fundstellen in den einzelnen Kapiteln anzugeben, wurde vernachlässigt. Diese zu finden ist für die Zielgruppe des Flyers voraussichtlich nicht trivial. 

Diese Ausführungen werden nun nicht noch weiter ausgeführt, sollen aber legitimieren, dass sich in dieser Arbeit soviel Zeit genommen wird, auf die Fehler der (fehlenden) Quellenangaben einzugehen. Das Weglassen sorgt nänmlich nach hinten raus insgesamt für mehr Arbeit im System, als durch das Ignorieren von vernüftigen Quellenangaben realistisch eingespart wird. Auch mehr Übersichtlichkeit kann kein vernünftig zu argumentierender Trade Off sein, wenn dadurch Informationen in derartigen Produkten \gls{zt} nur eine Halbwertszeit von einem halben Jahr, bis zur nächsten Gesetzesänderung, haben.
\bigskip

Die Auswirkungen davon tragen sich über die eigentlich betroffene Sphäre hinaus noch weiter. In diesem Fall in den Bereich der Bildung. Man wird als Lehrkraft vor das Dilemma gestellt, dass ein aufmerksamer Unterricht, der sich mit der Realität auseinandersetzt, nicht umhin kommt zahlreiche Schwächen unserers Systems aufzeigen. Darin verborgen liegt die Gefahr, dass als subtiler, womöglich unbewusster Lerneffekt, das Vertrauen in unser demokratisches System untergraben wird. Das ist nicht das Ziel von Bildung. 
Solche unbeabsichtigten Effekte verlangen von der Lehrkraft hohe Kompentenzen in vielen Bereichen ab, um die unausgesprochenen, aber vorhandenen Subtexte zu reflektieten; diese dann bewusst korrekt darzustellen, aber um Fatalismus zu umgehen, so zu framen, dass sie als veränderbare Schwächen eines Systems gerahmt werden, die sich mit gemeinsamer Anstrengung ändern ließen. Wird das nicht angesprochen, kann viel unintendierte und unbewusste Konstruktion in den Köpfen aller Beteiligten stattfinden. 
\bigskip

Durch die unzureichende Qualität des Materials, sehe ich vor Allem zwei mögliche, aber unbeabsichtigte Outcomes der Aufgabe. Diese sind so nur bei Betrachtung der unterschwellig übermittelten Informationen nachzuvollziehen:

Erstens, könnte es sein, dass ein ebenfalls nicht korrekter Flyer erstellt wird, bei dem der Fokus darauf liegt sich irgendwie durch die Aufgabe zu schlänglen ohne aufzufallen. Dabei würde \gls{uu} eher gelernt wie man ein gut wirkendes, anstatt ein gutes Produkt produziert. Das kann häufig der Outcome von schulischer Bildung sein, sollte \gls{me} aber nicht antizipert werden. Um den Kreis der Metaebene zu schließen: Wer so ausgebildet wird, produziert am Ende auch hundertseitige Merkblätter, die nach kurzer Zeit veraltet sind. Vermutlich noch mit der ehrhaften Intention die Dinge nachvollziehbar darzustellen - aber wie oben dargestellt: Ein gut wirkendes ist eben nicht unbedingt ein gutes Produkt. 

Zweitens, könnte es sein, dass vor allem die frustige Erfahrung nacherlebt wird, die mehrere Millionen Menschen\footnote{
    2022 waren es über 7 Millionen: \url{https://www.bpb.de/kurz-knapp/zahlen-und-fakten/sozialbericht-2024/553338/mindestsicherungssysteme/} (10.07.2025). % Bei der Internetrecherche um das Thema \emph{Mindestsicherung} oder auch spezifischer \emph{Arbeitslosengeld} % sind mir übrigens weitaus mehr Angaben zu der Höhe der Empfangenden untergkommen, als Informationen zu der Mindestsicherung selbst.
    } 
machen, die auf Mindestsicherung und den Umgang mit Ämtern und Gesetzen angewiesen sind. 

Andreas Peichl vom ifo Institut für Wirtschaftsforschung wird in deren Pressemitteilung ebenfalls als kritisch zitiert: \enquote{Das System ist zu kompliziert, und wirkt, als wäre es gemacht, um es den Empfängern schwer zu machen. Denn der Staat spart zwischen 6 und 10 Milliarden Euro im Jahr, weil Berechtigte im Antragsdschungel abgeschreckt werden und ihre Ansprüche nicht geltend machen. Das ist seit Jahrzehnten so.} Er führt weiter aus: \enquote{Das ist völlig absurd. Arbeitsagentur, Jugendamt, Wohnungsamt legen unterschiedliche und teilweise widersprüchliche Kriterien an, was zum Einkommen zählt. Da fehlt es an Abstimmung} \autocite{ifo.05.05.2021}.


% Es wird in dieser Arbeit nun nicht weiter darauf eingegangen, wie hoch die Fehler Quote bei staatlichen Infoseiten ist.



\paragraph{Erwartungshorizont: Arbeitsleben A2 Erarbeitung} (\gls{S} \pageref{ARBEITSLEBEN-A2})
\textsc{Aufgabe 1} \quad 
--- 

\textsc{Aufgabe 2} \quad
Niedriglohn ist Bruttoeinkommen unterhalb von \enquote{zwei Dritteln des deutschlandweiten mittleren Bruttomonatseinkommens}.
In der Quelle \emph{M4} sind für 2023 2.530\,€ brutto angegeben. 

\begin{myitemize}
    \item Mindestlohnerhöhungen \dots
    \item Von Gewerkschaften miterkämpfte Tarifabschlüsse auch für untere Einkommensschichten \dots
\end{myitemize}
\dots können den schrumpfenden Niedriglohnsektor erklären. 
\bigskip

\textsc{Aufgabe 3} \quad
\begin{myitemize}
    \item Im Vergleich der Brutto-Stundenlöhne verdienen Frauen weniger als Männer
    \item Das kann verstärkt werden durch andere strukturell diskriminierte Merkmale, wie Migrationsvordergrund oder Behinderung 
    \item Mehrere Kriterien definieren die prozentuale Lücke zwischen den durchschnittlichen Verdiensten der Geschlechter:
    \begin{myitemize}
        \item <10 Beschäftigte $\rightarrow$ Betriebe werden nicht berücksichtigt
        \item Sonderzahlungen (Weihnachtsgeld, Prämien) werden nicht berücksichtigt
        \item Öfffentliche Verwaltung wird nach Definition der \gls{eu} nicht berücksichtigt
        \item Menschen ohne binäre Geschlechtszughörigkeit werden gleichmäßig auf Frauen und Männer aufgeteilt
    \end{myitemize}
    \item Der Gender-Pay-Gap liegt in Deutschland 2024 bei etwa 16\,\%
    \item Es sind verschiedene Urasachen für den unbereinigten Gender-Pay-Gap berechnet:
    \begin{myitemize}
        \item Frauen arbeiten \emph{im Schnitt} häufiger in schlechter bezahlten Branchen und Berufen ($\approx$ 25\,\% Erklärungsanteil)
        \item Besonders in Bremen gibt es viele gut bezahlte Industriearbeitsplätze mit hohem Männeranteil
        \item Männer sind häufiger in Führungspositionen ($\approx$ 10\,\% Erklärungsanteil), sogar in frauendominanten Branchen
        \item Teilzeit bei Führungskräfrten ist selten. Frauen sind durch häufigere Sorgearbeit eher in Teilzeit. \Gls{zt} lässt sich der geringere Führungskräfteanteil bei Frauen dadurch erklären
        \item Teilzeit ist auch pro Stunde schlechter bezahlt ($\approx$ 20\,\% Erklärungsanteil)
        \item Minijobs sind ebenfalls por Stunde schlechter bezahlt ($\approx$ 6\,\% Erklärungsanteil)
    \end{myitemize}
    \item Zusammenhang zwischen Gender-Care-Gap und Gender-Pay-Gap wird hervorgehoben
    \item Der Gender-Pay-Gap von $\approx$ 16\,\% lässt sich aber nur für $\approx$ 10\,\% durch die obigen Faktoren erklären
    \item Die anderen $\approx$ 6\,\% nennen sich bereinigter Gender-Pay-Gap und sind die ungleiche Bezahlung trotz gleicher Qualifikation, gleichen Berufs und gleicher Position
    \item Teile des unbereinigten Gender-Pay-Gaps lassen sich durch Kinder bekommen erklären
    \item \Gls{ua} liegt das an unzureichender Kinderbetreuung
    \item Insbesondere langfristig verdienen Frauen dadurch weniger
    \item Negative Rückkopplung: Weil Frauen wenioger verdienen, übernehmen sie eher Sorgearbeit, wodurch sie wiederum weniger verdienen
\end{myitemize}

\noindent Die erste Grafik in \emph{M5} sagt aus: Je höher der Frauenanteil, desto niedriger der Lohn (für ausgewählte Berufe).

\noindent Die Politik kann insbesondere an Kinderbetreuungsangeboten ansetzen, um den Gender-Pay-Gap durch ein Ansetzen an den Gender-Care-Gap zu entschärfen. 



\paragraph{Arbeitsleben A2 Erarbeitung -- \enquote{Wie viel verdienen Arbeitnehmer:innen in Bremen und welche (politischen und wirtschaftlichen) Maßnahmen würden die Situation am Arbeitsmarkt verbessern?}}  (\gls{S} \pageref{ARBEITSLEBEN-A2})

In Aufgabe 2 sollte \enquote{Rechne aus, wie hoch der monatliche Bruttolohn \emph{für eine unverheiratete Person ohne Kinder} sein müsste, um nicht mehr als Niedriglohnempfänger:in eingestuft zu werden.} durch das Kursive präzisiert werden, da ansonsten die Steuerklasse zu vage bleibt. 

Aufgabe 3 hat einen präzisen, aufschlussreichen, aber langen Text zur Grundlage. Die Aufgabenstellung \enquote{Dieses Problem muss die Politik lösen.} wird in der überblicksweisen Analyse der Materialkarten aufgegriffen. 



\subsubsection{Arbeitsleben B -- \enquote{Entgrenzte Arbeit – Haben wir bald nie wieder Feierabend?}}
Arbeitsleben B ist im gleichen Kompetenzbereich wie Arbeitsleben A verortet: 
\begin{quotation}
    Die Schüler:innen sind in der Lage berufliche und gesellschaftliche Lebenssituationen vor dem Hintergrund der sich verändernden wirtschaftlichen und politischen Rahmenbedingungen zu erkennen, um adäquate individuelle Entscheidungen treffen zu können.

    \autocite[18]{bplan}
\end{quotation}

Arbeitsleben B1 bis B3 stehen allesamt unter der Leitfrage:
\enquote{Welche gesetzlichen Grundlagen regulieren die Arbeitszeiten und wie kann ich mich vor entgrenzter Arbeit schützen?} (\gls{S} \pageref{ARBEITSLEBEN-B1})

Insgesamt greift der Einführungstext zur Arbeitszeit in \emph{Arbeitsleben B} gut die in \emph{Arbeitsleben A} \gls{ua} durch den Gender-Pay- und Gender-Care-Gap augeworfenen Problemlagen auf. 


\paragraph{Erwartungshorizont: Arbeitsleben B 1 Einstieg} (\gls{S} \pageref{ARBEITSLEBEN-B1})
\begin{myitemize}
    \item[] \textbf{Pro-Argumente} aus \emph{M1}
    \item Keine Pendelzeiten
    \item Vereinbarkeit von Beruf und Familie
    \item Für einige höhere Konzentration und Effizienz
    \item Nach Beratungen der \gls{ank} gibt es keine Qualitätsabfall der Arbeit
    \item[]    
    \item[] \textbf{Contra Argumente} aus \emph{M1}
    \item Keine sozialen Kontakte im Unternehmen
    \item Identifikation mit dem Unternehmen leidet 
    \item Gefühl von Entgrenzung und ständiger Erreichbarkeit 
    \item Ruf von schlechterer Arbeitsqualität
\end{myitemize}

\paragraph{Arbeitsleben B1 Einstieg -- \enquote{Welche gesetzlichen Grundlagen regulieren die Arbeitszeiten und wie kann ich mich vor entgrenzter Arbeit schützen?}} (\gls{S} \pageref{ARBEITSLEBEN-B1})
Die Debatte ist kleinschrittig in sehr klare Strukturen eingebettet. Das dürfte einem reibungslosen Ablauf zuträglich sein. Was der Vorteil an der Einschränkung ist, nur ein Argument zum entkräften zu nutzen, \enquote{das nicht sein/ihr eigenes vorbereitetes ist} wird nicht klar. Wenn eines der vorbereiteten Argumente gut als Antwort passt, weshalb sollte es nicht genutzt werden?  

Wenn die Lehrkraft sich überlegt, ob es notwendig ist, die Gruppeneinteilung durch Regeln zu steuern, kann das (durch die etwaige Steuerung möglichst faire) Wettkampf Element durch eine Abstimmung am Ende in der Altersklasse sicherlich motovierend wirken. Durch die Kriterien nach denen bewertet werden soll, trainiert es Werturteile begründet abzugeben. Dadurch, dass auch rhetorisches Geschick als Kriterium vorgegeben wird, eröffnet sich die Chance, darauf zu sprechen zu kommen, dass nicht nur das beste sachliche Argument überzeugen kann, sondern auch die Form der Darstellung wesentlichen Einfluss nehmen kann. 


\paragraph{Erwartungshorizont: Arbeitsleben B2 Erarbeitung} (\gls{S} \pageref{ARBEITSLEBEN-B2})
\textsc{Aufgabe 1} a) \quad ---

\textsc{Aufgabe 1} b) \quad Als Arbeitszeitregelungen sind in \emph{M2} erwähnt:
\begin{myitemize}
    \item Eine \enquote{Verpflichtung der Arbeitgeber zur Erfassung der Arbeitszeit der einzelnen Beschäftigten auf Grundlage der Entscheidung des Europäischen Gerichtshofs von 2019 und des BAG-Beschlusses von 2022}, diese sollte laut \gls{ank} \enquote{zur Schaffung von Rechtssicherheit und zur Stärkung des Gesundheitsschutzes zügig in deutsches Arbeitszeitrecht umgesetzt werden}
    \item \enquote{Durchsetzung der 40-Stunden-Woche und des arbeitsfreien Samstags in den 1960er-Jahren}
    \item \enquote{teilweise Einführung einer 35-Stunden-Woche mit vollem Lohnausgleich}
    \item \enquote{In vielen Bereichen sind (wieder) Arbeitszeiten an oder gar über der 40-Stunden-Grenze zur Regel geworden}
\end{myitemize}
In \emph{M3} finden sich folgende Arbeitszeitregelungen für Azubis: 
\begin{myitemize}
    \item Umziehen und Reinigung von Arbeitsplätzen gehört zur Arbeitszeit
    \item Die Wege zur und von der Arbeitsstelle sind keine Arbeitszeit
    \item Berufsschulpflicht: Wenn die Berufsschule vor 09:00 Uhr beginnt, darf am Vortag nur bis 20:00 Uhr gearbeitet werden
    \item VWenn der Unterricht vor 09:00 Uhr beginnt, darf an Berufsschultagen überhaupt nicht früher gearbeitet werden
    \item Volljährige Azubis können auch nach der Berufsschule noch in den Betrieb müssen
    \item bei zwei oder mehr Berufsschultagen, kann das auch für Minderjährige gelten
    \item Die Pausen in der Schule und der Weg von der Berufsschule zur Arbeitsstätte werden bei betriebsüblichen Arbeitszeiten als Arbeitszeit angerechnet
    \item Volljährige haben 6\, bis 9\,h Arbeit Recht auf 30$\,^{\prime}$ Pause, >\,9\,h sind es 45$\,^{\prime}$
    \item Überstunden nur wenn kein Erwachsener sie übernehmen kann.
    \item Innerhalb von drei Wochen müssen Überstunden ausgeglichen werden
    \item Überstunden müssen vergütet oder durch Freizeit ausgeglichen werden. Ohne Einigung darf der Arbeitgeber entscheiden, was von beidem
    \item Ansprechstellen für Schwierigkeiten, sind der Betriebsrat (im Unternehmen) und der Personalrat (im öffentlichen Dienst). Diese arbeiten mit der Jugend- und Auszubildendenvertretung eng zusammen    
    \item Für Minderjährige (15 bis 17 Jahre) gilt
    \begin{myitemize}
        \item wöchentlich höchstens 40\,h Arbeiten
        \item täglich höchstens 8\,h, außer an vorherigem Tag ware es weniger als 8\,, dann dürfen bis zu 8,5\,h gearbeitet werden
        \item frühestens ab 06:00 Uhr, längstens bis 20:00 Uhr darf gearbeitet werden 
        \item Ausnahmen \gls{bspw} in Bäckerein oder Landwirtschaft
        \item Ab 16 Jahren darf in Gaststätten bis 22:00 Uhr und in Mehrschichtbetrieben bis 23:00 Uhr gearbeitet werden
        \item Einmal wöchentlicher Berufsschulunterricht mit mindestens 5 Unterrichtsstunden à 45$\,^{\prime}$ wird mit 8\,h Arbeit gleichgestellt
        \item Ruhepausen müssen feststehen -- Pausen frühestens 1\,h nach Beginn und 1\,h vor Ende der Arbeit
        \item Ab 4\,h Arbeitszeit müssen es 30$\,^{\prime}$, ab 6\,h muss es 1\,h Pause sein 
        \item Nur in bestimmten Branchen darf Samstag gearbeitet werden. \Gls{zb} in Bäckereien, Lebensmittelmärkten, Friseurläden oder in Kfz-Betrieben
        \item Nur in Branchen wie Krankenhäusern und Gaststätten darf Sonntag gearbeitet werden
        \item Zwei Samstage und Sonntage pro Monat sollten frei sein
    \end{myitemize}
\end{myitemize}


\paragraph{Arbeitsleben B2 Erarbeitung-- \enquote{Welche gesetzlichen Grundlagen regulieren die Arbeitszeiten und wie kann ich mich vor entgrenzter Arbeit schützen?}} (\gls{S} \pageref{ARBEITSLEBEN-B2})
\emph{M2} kann lobend hervorgehoben werden, unübersehbar ist \emph{Mai 2024} angegeben, sogar bei den meisten Downloads, die nochmal rechts am Rand auftauchen, steht überblicksweise direkt beim Titel ein Jahr dabei. Im Text selbst werden auch Forderungen von Beschäftigten dargelegt und verhandelt, wie sich zukünftige Arbeitszeitmodelle verwirklichen ließen und was die Argumente für die Änderungen sind. Diese Informationen spielen in Aufgabe 1 noch keine Rolle, können aber genutzt werden, um eine Argumentation für bestimmten Regelungen in den folgenden zwei Aufgaben zu gestalten. 



\paragraph{Erwartungshorizont: Arbeitsleben B3 Auswertung/Übertrag} (\gls{S} \pageref{ARBEITSLEBEN-B3})
\textsc{Aufgabe 1}  \quad 
Falls Hinweise zur Präsentation gebraucht werden, kann auf die zahlreichen Zahlen zu denen auch schon Abbildungen im Bericht zu finden sind verwiesen werden. Diese bieten die Visualisierung im Hintergrund. 


\textsc{Aufgabe 2} \quad ---


\textsc{Aufgabe 3} \quad ---


\paragraph{Arbeitsleben B3 Auswertung/Übertrag -- \enquote{Welche gesetzlichen Grundlagen regulieren die Arbeitszeiten und wie kann ich mich vor entgrenzter Arbeit schützen?}} (\gls{S} \pageref{ARBEITSLEBEN-B3})
\emph{M4} gibt in einem kleinen Kasten Informationen zur Erhebung der Daten an. Sozusagen ein überblicksweiser Methodenteil. Aufgrund der Vorbildfunktion kann man die Quelle daher gut nutzen. Schade ist wiederum, dass zu den erwähnten Modellprojekten auf \gls{S} 6 keine Quellen angegeben sind. Erneut gilt, die \gls{ank} produziert in weiten Teilen gutes Material mit hoher Transparenz (so sind auch in \emph{M4} Verantwortliche und Mitarbeitende angegeben). Aber bei dem konsequenten Angeben von Quellen ist auch dort noch Verbesserungspotential. 

Im Bildungsplan zählt das Herstellen von Produkten, hier also \gls{zb} Präsentieren und Plakate erstellen, zu den \enquote{Methoden des realen Handelns} \autocite[13-14]{bplan}. Diese schrammen durch Forderungen, die in Aufgabe 3) Teil 3) an die Politik formuliert werden sollen, an die \enquote{Methoden des simulativen Handelns}. Auf mögliche Implikationen des begrenzten Raumes Schule, wird, wie auch zu \emph{Arbeitsleben A2} anhand der Aussage  \enquote{Dieses Problem muss die Politik lösen.} angekündigt, im Gesamtüberblick näher eingegangen. 



\subsubsection{Arbeitsleben C -- \enquote{Mitbestimmen oder nur dabei sein?}}
Als Kompetenzbereich wird dieser Unterrichtseinheit: 
\enquote{Die Schüler:innen sind in der Lage, die derzeitigen Arbeitswelten mitsamt ihren Institutionen, Kulturen und Praktiken als historisch gewachsen beschreiben und als politisch veränderbar herauszuarbeiten}, zugeordnet. 
Die Leitfragen \dots
\begin{myitemize}
    \item \enquote{Welche Aufgaben haben Gewerkschaften und wie ist der Interessenausgleich von Arbeitnehmer:innen und Arbeitgeber:innen (gesetzlich) organisiert?} (\gls{S} \pageref{ARBEITSLEBEN-C1})
    \item \enquote{Politik meets Betrieb: Wie kann betriebliche Mitbestimmung die Demokratie stärken?} (\gls{S} \pageref{ARBEITSLEBEN-C2})
\end{myitemize}
\dots teilen die Zeiteinheiten ein. 


\paragraph{Erwartungshorizont: Arbeitsleben C1 Einstieg} (\gls{S} \pageref{ARBEITSLEBEN-C1})
\textsc{Aufgabe 1} \quad Es geht in \emph{M1} um Partizipationsmöglichkeiten am Unternhemen für Beschäftige nach \emph{Betriebverfassunggesetz (BetrVG)}.  \Gls{ua}durch \emph{Betriebsratgründung} und passives oder aktives \emph{Wahl}recht zum oder von \emph{Betriebsratsmitglieder/n}. Für Azubis wird die Partizipationsmöglichkeit am Betriebsrat durch die \emph{Jugend- und Auszubildendenvertretung (JAV)} dargestellt. 
Die Punkte aus \emph{M2} können direkt so übernommen werden. \\ 

\textsc{Aufgabe 2} \quad Die Punkte aus \emph{M2} können wieder als Vergleich herangezogen werden. \\ 

\textsc{Aufgabe 3} \quad Die Gestaltung einer Checkliste mit möglicher Umstrukturierung der Informationen aus \emph{M3} ist die eigentliche Arbeitsleistung. 


\paragraph{Arbeitsleben C1 Einstieg -- \enquote{Welche Aufgaben haben Gewerkschaften und wie ist der Interessenausgleich von Arbeitnehmer:innen und Arbeitgeber:innen (gesetzlich) organisiert?}} (\gls{S} \pageref{ARBEITSLEBEN-C1})

Die Hans-Böckler-Stiftung hat mit dem Informationsvideo \emph{M1} vermutlich eine Zielgruppe im Kopf gehabt. Für alle die das Video nicht nur zum Unterrichtseinstieg gucken und neugierig geworden sind, ist es schade, dass die Chance verpasst wurde direkt unter dem Video auf die Studien zur Profitsteigerung durch Betriebsräte zu verlinken. 
Ansonsten bieten das kurzweilige Video und die Verknüpfung der Informationen zu den persönlichen Erfahrungen einen angemessenen Einstieg. 
Insbesondere \emph{M2} stellt schon einen guten Überblick, über die Aufgaben eine Betriebsrates dar. Inwieweit in Aufgabe 2 noch gewinnbringend weitere Informationen dazugewonnen werden können halte ich für fraglich. 

Aufgabe 3 ist sozusagen auch hauptsächlich ein Wiedergeben der Informationen aus \emph{M3}. Durch die leicht abwechselnden Darstellungsformen (Mind-Map, Checkliste \gls{etc}) der inhaltswiedergebenden Aufgabenstellungen aus Aufgabenbereich I, wird über die Materialkarten hinweg versucht nicht nur schriftliche zusammenfassungen zu fordern. Auch das Einbetten in ein vorzustellendes Szenario kann dabei helfen die Motivation zu beeinflussen. 


\paragraph{Erwartungshorizont: Arbeitsleben C2 Erarbeitung} (\gls{S} \pageref{ARBEITSLEBEN-C2})
\textsc{Aufgabe 1} \quad 
\begin{myitemize}
    \item Mitbestimmung stärkt grundsätzlich das Vertrauen in die Demokratie
    \item Menschen aus den unteren Einkommensgruppen sind wegen ausländischer Staatsbürgerschaften seltener wahlberechtigt
    \item Menschen in Arbeitslosigkeit, mit weniger Geld und niedrigerem Bildungsniveau gehen seltener zur Wahl
    \item \enquote{Diese Menschen haben die Erfahrung gemacht: Für mich zahlt es sich nicht aus, meine Stimme abzugeben}. Der Deutsche Budnestag hat in den letzten Jahren tatsächlich \enquote{Entscheidungen getroffen [...], die zugunsten der oberen Einkommensgruppen verzerrt waren} \autocite{Elsasser.2017}
    \item \enquote{Vielleicht ist für sie die Entscheidung, nicht mehr mitzumachen, sogar ganz rational: Ihre Stimme zählt weniger als andere, obwohl die Demokratie verspricht, dass alle Menschen gleich viel wert sind}
    \item \enquote{Demokratie bedeutet ja auch, dass wir die Regeln unseres Zusammenlebens gemeinsam bestimmen}
    \item \enquote{neue direkte Partizipationsmöglichkeiten zu schaffen} reicht nicht aus. \enquote{Da gehen wieder vor allem diejenigen hin, die auch die anderen Chancen zur Teilhabe schon nutzen}
    \item Menschen sind viel  Lebenszeit auf der Arbeit. Wenn Mitbestimmung auf der Arbeit stattfindet, hält fehlende Energie nach Feierabend nicht mehr von Partizipation ab. 
    \item \enquote{Die Demokratie im Betrieb kann gerade bei unteren Einkommensgruppen einen deutlichen Effekt erzielen}
    \item Betriebsräte sind wichtig für Demokratiearbeit
\end{myitemize}

\textsc{Aufgabe 2} \quad
\begin{myenumerate}
    \item Laut der \emph{Political Spill-Over Theorie}: Erfahrungen auf der Arbeit beeinflussen auch das Verhalten außerhalb der Arbeit. 
    \item Auch Selbstwirksamkeitserfahrungen im Kleinen, wie \gls{zb} die selbsttätige Herstellung oder Reperatur eines Hockers, könnten für selbstbewussteres Auftreten an anderer Stelle sorgen. Es ließe sich Argumentieren, dass jede Art von Erfolg Selbstwirksamkeit erfahren lässt und für ein höheres Selbstbewusstsein sorgt. 
    \item 
    \begin{myitemize}
        \item Mitbestimmung im Betrieb eröht die Wahrscheinlichkeit das Gefühl zu bekommen, dass die eigenen Interessen Einfluss haben können \emph{Verstärktes Gefühl politischer Selbstwirksamkeit}
        \item Öffentliches Sprechen und Forderungen Formulieren wird geübt \emph{Erwerb politischer Fähigkeiten}
        \item Beschäftigte erfahren, \enquote{dass Arbeitsbedingungen nicht naturgegeben sind}, sondern \enquote{vielmehr von arbeitsrechtlichen Normen und damit letztlich von politischen Entscheidungen außerhalb des Betriebs} abhängen. \emph{Erhöhte Aufmerksamkeit für politische Themen}
        \item Dadurch dass Arbeitnehmende, um ihre Stärke einzusetzen, dem Arbeitgeber gegenüber geschlossen auftreten müssen, lernen sie untereinander Kompromisse zu schließen. \emph{Verstärkte Solidarität}
    \end{myitemize}
    \item Demokratie ist nicht die Existenz demokratischer Institutionen, sondern Menschen, die etwas bewirken wollen. Dafür ist politisches Interesse Voraussetzung. Zur Verdeutlichung könnte zugespitzt werden: Wenn es demokratische Insitutionen gibt, aber niemand wählen geht und sich zur Wahl aufstellen lässt, ist es dann noch eine Demokratie?
    \item \enquote{Betriebliche Mitbestimmung erhöht das Bewusstsein für die politische Dimension von Arbeit} $\rightarrow$  Parteien die für \enquote{die für stärkere Arbeitnehmerrechte, Gleichstellung und Umverteilung stehen} werden eher gewählt 
    
    \enquote{Das gemeinsame Engagement der Beschäftigten führt zu mehr Solidarität} $\rightarrow$ \enquote{wenn eine generalisierte Solidarität die gesamte Arbeitnehmerschaft} ohne diskriminierte Merkmale einschließt, nimmt die Neigung zu rechtsextremen Parteien ab 
\end{myenumerate}


\textsc{Aufgabe 3} \quad Es ließe sich Argumentieren, dass jede Art von Erfolg Selbstwirksamkeit erfahren lässt und für ein höheres Selbstbewusstsein sorgt. Wenn Selbstwirksamkeit in Kooperation erfahren wird, stärkt das den Gruppenzusammenhalt und erhöht die Chance gemeinsam gestalten zu wollen. 

\paragraph{Arbeitsleben C2 Erarbeitung -- \enquote{Politik meets Betrieb: Wie kann betriebliche Mitbestimmung die Demokratie stärken?}} (\gls{S} \pageref{ARBEITSLEBEN-C2})

In \emph{M2} wird anschaulich und in verständlichen (kurzen) Sätzen das wissenschaftliche Vorgehen erläutert. 






\section{Schluss}
% \subsection{Wirtschaftspolitik} % ansonsten wohl Gesellschaft
\subsection{Den Themenfeldern gemeinsame Beobachtungen}
Häufiger schreibe ich, dass an einigen Stellen Dinge vielleicht \enquote{expliziter} gemacht werden könnten. Ein Stück weit ist das auch dem Umstand geschuldet, das Unterrichtsmaterial eben noch kein fertiger Unterricht ist und vieles von den Schüler*innen und den Lehrkräften abhängig ist. Es ließe sich durchaus auch argumentieren, dass das Material, welches es häufig schafft wissenschaftliche Erkenntnisse einzubringen, durch die Aufgaben für sich spricht, und das \emph{expliziter Machen} nur ein Ausdruck davon ist, dass ich dem Material mehr Kontrolle über die Prozesse im Unterricht einbauen möchte. 
\bigskip

Auch 



\paragraph{Muss Material perfekt sein?} 
Ich kritisiere für den schulischen Bildungsbereich akribisch den Umgang mit Quellen. Auch wenn ich das argumentativ zu begründen versucht habe, stellt sich die Frage, ob man ausschließlich einwandfreies Material nutzen darf?
Meine Antwort wäre ein eindeutiges \emph{Nein}. Es lässt sich sogar katastrophal schlechtes Material gewinnbringend nutzen. Nämlich wenn es darum geht, die Kriterien von Glaubwürdigkeit Erkennen und Bewerten zu lernen. Dazu muss das Material dann aber entsprechend in den Unterricht eingebettet sein. Am ehesten dürfte das auch durch passende Aufgabenstellungen realisiert werden. 

Die andere Möglichkeit ist es, mit allen Schüler*innen ein Ritual zu schaffen, bei welchem jedes Material, welches in der Schule verwendet wird, auch auf die Qualität der Nachvollziehbarkeit der Informationsherkunft zu untersuchen. Das Ergebnis dürfte gerade bei Verwendung von didaktisch stark aufbereitetem Material von Verlagen für Schulmaterial ernüchternd ausfallen. Das heißt aber nicht, dass dieses ritualisierte Analysieren immer maximal in die Teife gehen und einen daher lähmen muss. Bei einem gut etablierten Ritual besteht die Möglichkeit, \gls{zb} bei Nutzung von Schulbüchern, schnell das bekannte Muster zu erkennen und diesen Teil schnell abgehakt zu haben. 

An so einem Vorbild können Schüler*innen auch beispielhaft ihre e\emph{xekutive Funktion} trainieren. Wenn es im Unterricht geschafft wird durch derartige Rituale aufzuzeigen, dass die Darstellung der Institutionen in Deutschland im Schulbuch nicht ausreichend mit Gestzestexten belegt ist, aber das aktuell in Kauf genommen wird, weil der Überblick über die Institutionen für die kommende Einheit gebraucht wird, in welcher anhand der Institutionen beispielhaft Gewaltenteilung und der Einfluss von Veto-Optionen auf den Regelungs-Output dargestellt werden sollen, dann ist das eine geeignete Möglichkeit die Priorisierung transprent zu gestalten. 

Ich argumentiere, dass die Resilienz gegenüber Einflussnahme nur durch regelmäßige Praxis ausgebaut werden kann. Sobald Zeitungsartikel mit Kommentaren \& Meinungen oder Essays, politische Reden \& Parteiprogramme gelesen werden, wird so ein Ritual dann mehr Platz im Unterricht einnehmen können und müssen. 

Eine gute Urteilsfähgikeit über die Qualität von Informationen ist Voraussetzung für Medienkompetenz. Wenn Spitzfindigkeiten bei Seite gelassen werden, ist beides Voraussetzung für die doch recht konsensual zu erwerbende politische Urteilskompetenz, welche wiederum mit Voraussetzung für die politische Handlungsfähigkeit ist. In dieser lässt sich auch \gls{me} nach eine Art Zusammenfließen aller anderen methodischen und inhaltlichen Kompetenzen der politische Bildung sehen. 

Grundsätzlich böte so ein Vorgehen die Chance, das Umschalten zwisachen Meta-Ebene und der eigentliche Aufgabe zu üben. Das Ziel ist, sich über die möglichen Implikationen einer Meta-Ebene bewusst zu sein, während man es schafft sich auf die vorliegende Aufgabe zu konzentrieren.
\bigskip

Fremdes Unterrichtsmaterial, welches nicht durch eigene Strukturvorgaben ein vergleichbares Ritual etabliert, ist bei nicht überdurchschnittlich herausragender Quellenarbeit im eigenen Material, in dieser Hinsicht also restlos der Kompetenz der Lehrkraft ausgeliefert. 



\subsubsection{Arbeitsblatt-Charakter}
Letzten Endes bleibt das institutionalisierte Lernen Arbeit für alle Beteiligten. Dennoch sollte hinsichtlich der hier untersuchten Unterrichtsvorschläge nicht unerwähnt bleiben, dass auch diese es selten schaffen, aus dem in der Überschrift erwähnten Arbeitsblatt-Charakter auszubrechen. Abwechslung der Methoden und zwischen Lesen, Schreiben und Sprechen, Wechsel zwischen Einzel-, Partner*innen- und Gruppenarbeit sowie Methoden im Plenum sind vorhanden und stellen damit eine solide Grundlage bereit. Der Ausbruch, um durch Ungewöhnliches Aufmerksamkeit oder Nachhaltigkeit des Lernens zu erzeugen, wird hingegen nicht gewagt. 
Auch das Training der \enquote{politischen Handlungsfähigkeit} bleibt auf dem für Schule nicht untypischen Niveau des entfremdeten Lernens. 

Der Anspruch der Unterrichtsvorschläge ist allerdings auch kein revolutionärer. Ohnehin ist es eine andere Diskussion, inwieweit die Schule als staatliche und stark bürokratisierte Institution bisweilen im Widerspruch zum freien Denken steht, welches immer wieder auch Bestandteil der Dinge ist, die sich in ihren Zielsetzungen lesen lassen. 

\enquote{Arbeitsleben B1} Debatte

\enquote{Stellt Euch folgende Situation vor.} \enquote{Arbeitsleben B2}

\enquote{Welche politischen Rahmenbedingungen müssten geschaffen werden, um den Vorschlag umzusetzen?} \enquote{Arbeitsleben B3}

\subsection{Vorbildfunktion, (Beobachtungs)-lernen und Fake Partizipation \label{fakePartizipation}}
Die Schule ist eine restriktive Institution mit zumeist höchst undemokratischen Wirkmechanismen.
Mitbestimmung, die in der Schule versucht wird zu realisieren ließe sich von Kritiker*innen als \emph{performative Mitbestimmung} bezeichnen. 


Die Inhalte der Unterrichtsvorschläge der \gls{ank} zeigen durchaus Möglichkeiten zur realen Partizipation auf. Die  

Berufsschule 

So wird in einer Aufgabe zum Gender Pay Gap in \emph{Arbeitsleben A2 mit M5}

\begin{quote}
    Vergleicht in Lerntandems die Ergebnisse. Formuliert gemeinsam eine Antwort auf die Aussage \enquote{Dieses Problem muss die Politik lösen.}
\end{quote}

Distanz geschaffen und es wird einerseits geübt Forderungen auszumachen und zu formulieren. Andererseits wird subtil von den Schüler*innen erfahren, wie schnell womöglich sinnvolle Forderungen fomruliert werden können und wie gering die Chancen darauf sind, diese Forderungen im \enquote{Monstrum Bürokratie} (Friedrich Merz 2025, andauernd) mittelfristig umgesetzt zu erleben. 

Im Bildungsplan werden \enquote{Methoden des realen Handelns} \autocite[13]{bplan} neben explizit als \enquote{Methoden des \emph{simulativen} Handelns} bezeichneten Methoden dargeboten \autocite[14]{bplan}.

Das experimentell brauchbar untersuchbare Beobachtungslernen \autocite{Bandura.1977}, auf welches referiert werden kann, wenn von Vorbildern gesprochen wird und das sich anführen lässt, um schlechte Medienkompetenz kritisieren zu können ist wahrscheinlich nicht ausreichend, um die folgende Argumentation zu unterfüttern.
Aber in Kombination mit unterbewusst ablaufenden Lernprozessen wie dem \emph{impliziten Lernen} \autocite[82-93]{Kiesel2012} kommt man der Sache näher.


% Nach wie vor gilt der Ausspruch des englischen Staatsmannes Winston Churchill vom 11. November 1947 bei einer Rede im Unterhaus: "Demokratie ist die schlechteste aller Regierungsformen – abgesehen von all den anderen Formen, die von Zeit zu Zeit ausprobiert worden sind." Oder, mit dem Demokratieforscher Manfred G. Schmidt formuliert, " […] die zweitbeste Demokratie ist immer noch besser als die beste Nicht-Demokratie". Die Demokratie mag nur als das kleinere Übel angesehen werden, vereint aber andererseits so viele Vorteile auf sich, dass sie weiterhin als die beste bekannte Herrschaftsform bezeichnet werden kann.
%% https://www.bpb.de/shop/zeitschriften/izpb/demokratie-332/248593/demokratie-in-der-krise-und-doch-die-beste-herrschaftsform/ 16.06.2025




Beobachtung \autocite[72-81]{Kiesel2012}
implizit 

Welche Kategorisierungen für welche beim Lernen im Gehirn ablaufenden Prozesse nun auch dem aktuellen Wissensstand entsprechen mögen, die Arbeitshypothese, dass Menschen auch außerhalb von deutlich formulierten Regelsystemen lernen, ist vertretbar. 
JUNGE ICH WILL NICHT MEHR



In \enquote{Arbeitsleben c} wird in \emph{M1} auf ein Interveiew mit Martina Zandonella \autocite{Zandonella.2024} Bezug genommen. Es geht \gls{ua} um niedrigere Wahlbeteiligung von Menschen aus sozioökonomisch benachteiligten Schichten.

Herausgestellt werden soll das Zitat:
\begin{quote}
    Ja. Diese Menschen haben die Erfahrung gemacht: Für mich zahlt es sich nicht aus, meine Stimme abzugeben.   
\end{quote}

Die Qualität der Quelle von der \gls{ank} ist als hoch anzusehen. Für Interviews nicht unbedingt Standard, sind mehrere getroffene Aussagen von Martina Zandonella mit einem Link belegt. Der eine Link führt zum Beispiel zu einer Präsentation\footnote{
    \url{https://www.arbeitnehmerkammer.de/fileadmin/user_upload/Veranstaltungen/Veranstaltungsdokumentation/Downloads/Demokratische_Dividende_Jirjahn.pdf} (besucht am 28.06.2025)
} von Uwe Jirjahn, Porfessor an der Universität Trier. An anderer Stelle bezieht sich Frau Zandorella auf eine Studie, welche 
\enquote{erstmals für Deutschland nachweisen [konnte], dass politische Entscheidungen mit höherer Wahrscheinlichkeit mit den Einstellungen höherer Einkommensgruppen übereinstimmen, wohingegen für einkommensarme Gruppen entweder keine systematische Übereinstimmung festzustellen ist oder sogar ein negativer Zusammenhang} \autocite[177]{Elsasser.2017}.

\citetitle{Elsasser.2017} 

Wahlbeteiligung vgl. \enquote{Demokratie}

Da Rhetorik durchaus hilfreich sein kann, soll folgende Aussage die Problemstellung illustrieren:
\begin{quote}
    Lernende sind nicht dumm. Wenn in der Schule demokratische Übungen immer das bleiben: Übungen. Lernen sie, dass ihr politisches Handeln keine Auswirkungen hat. Das ist gefährlich.
\end{quote}

Von anderer Seite ließe sich bestimmt arguzmentieren, dass es auch gefährlich ist, wenn Menschen dazu erzogen werden, ihre Bedingungen als veränderbar wahrzunehmen. Zum Beispiel indem sie sich als kooperierend sehen und ihnen das Zusammenschließen als Option bewusst ist.
Aber da genau das vom demokratischen Gedanken gedeckt ist, können die auf Rechtsgrundlage ihre faschistische Fresse halten. HIER WISSENSCHAFTLICH MECKERN MEIN FREUND, ABER RECHT HAST DU JA


\subsection{Subjektive Einschätzung}
Gleichwohl das wissenschaftliche Arbeiten nach Objektivität streben sollte und sich diverser Kulturtechniken bedient, um diesem Ideal nahe zu kommen, ist eine subjektive Einflussnahme der Forschenden kaum auszuschließen. Aus diesem Grunde soll im Anschluss an % wirklich nach?
die kriteriengeleitete Analyse noch der subjektive Eindruck des Autors dargestellt werden. 

\subsection{Reflexion der Materialkarten, ihrer Quellen $Mn$ und der Metaebene}
Die Frage nach der eigenen Produktionstätigkeit der \gls{sus} ist in Bezug auf die geforderte politische Handlungsfähigkeit interessant. Strukturell bleibt auch eine gut ausgebildete Analysefähigkeit auf einer Beobachtungsebene. Das ist für politische Handlungsfähigkeit zweifelsohne wichtig. Spannend ist jedoch, wie im Rahmen von Schulunterricht tatsächliche Erfahrungen von Gestaltung verwirklicht werden können. Die demokratische Legitimation ist am höchsten, wenn möglichst viele Menschen entweder tatsächlich an den Entscheidungen beteiligt waren oder zumindest eine hohe Identifikation mit den Entscheidungen aufweisen, auf welche sich geeinigt wurde. 

Ausschließlich auf der Beobachtungsebene zu bleiben, birgt die Gefahr, ausgezeichnet, aber ohnmächtig Entwicklungen beschreiben, nachvollziehen und erklären zu können, mit denen man sich überhaupt nicht identifizieren kann. 

Daher ist es immer wieder spannend zu reflektieren, an welchen Stellen Unterricht es schafft, politische Selbstwirksamkeit im Kleinen erproben zu lassen.  


\paragraph{Quellen  $Mn$}
Es fällt auf, dass der Großteil der Quelmaterialien direkt von der \gls{ank} stammen.
Die Quellen sind aber von vergleichsweise hoher Qualität. Eine einfache Checkliste, wie sie vielleicht zur ersten Qualitätseinschätzung an Schüler*innen gegeben würde, wäre erfüllt. Die Mitarbeitenden sind transprent angegeben (Interviewer*in, Autor*innen \gls{etc}), Datum  ist angegeben (bis auf die Ausnahme, dass man bei dem Magazin kurz seinen engen Blick erweitern muss), es ist eindeutig wo veröffentlicht wurde und wer das finanziert hat (die \gls{ank}). Über die Website werden auch keine Intentionen absichtlich intransparent gestaltet und durch den gesetzlich festgeschriebenen Rahmen sind die Herkunft der Finanzierung und die Ziele öffentlich nachzuvollziehen. 

Die Themen sind natürlich auf die Arbeitswelt bezogen, haben aber \gls{idr} stets die gesamtgesellschaftliche Implikationen im Blick. Da policy-Sphären nicht unabhängig voneinander existieren, ist das eine Sichtweise, die politischer Bildung entgegenkommt. 
Auch kommen im \gls{ank}-Material unterschiedliche Exper*innen außerhalb der \gls{ank} zu Wort, häufig aus der Wissenschaft.

Auch das Material direkt von der \gls{ank} und dessen weitere Quellenangaben sind vergleichsweise gut. Wird \gls{zb} in \emph{M4} (bei \emph{Arbeitsleben A2}) von einer \gls{diw}-Studie gesprochen, oder wird in einem Interview (\emph{M1} von \emph{Arbeitsleben C2}) gesagt, dass der Bundestag \enquote{in den letzten 30 Jahren überwiegend Entscheidungen getroffen hat, die zugunsten der oberen Einkommensgruppen verzerrt waren}, sind die Quellen beim Material der \gls{ank} auch direkt verlinkt \autocite{Elsasser.2017}. 

\paragraph{Metaebene}
Die Intention, die eigene Arbeit zu verbreiten, ist gewissermaßen naturgemäß. Die riesige Welt der Werbung ist Ausdruck davon -- so stark, dass auch öffentliche Institutionen immer wieder werben, obwohl sie keinen Profitdruck haben. 
Nun ist es so, dass die Arbeit der Veröffetnlichungen der \gls{ank} über die Zwangsbeiträge (\gls{s} \gls{abs} \ref{ank} \gls{S} \pageref{ank}) ohnehin schon finanziert ist; gewissermaßen von der Bremer Öffentlichkeit. Dankbarerweise verschwendet die \gls{ank} keine Ressourcen darauf ihre ohnehin schon bezahlten Veröffentlichungen nur den Beitragszahlenden zugänglich zu machen. Es ist für die gesamte Bremer- und darüber hinaus auch für die gesamte Weltöffentlichkeit zugänglich. \Gls{me} nach eine sinnvolle Verwendung von Ressourcen. Weshalb sollte extra Aufnwand betrieben werden, um bayrischen Menschen wertvolle Informationen vorzuenthalten, es ist doch schon alles bezahlt? 
Gleichzeitig ist das ein spannendes Beispiel dafür, wie die \gls{ank} als Bremer Institution, dem Gedanken des Föderalismus entsprechend, auch auf lokale Gegebenheiten eingeht, diese aber nicht einer Konkurrenzlogik folgend unter Verschluss hält. Es ist davon auszugehen, dass sich viele Strukturen, die dort verhandelt werden, auch anderswo auf der Welt wiederfinden. Und wenn Mecklenburg-Vorpommern geholfen werden kann, ist das gut für Deutschland, wenn Österreich geholfen werden kann, ist das gut für die \gls{eu} und sollte tatsächlich jemand aus Südafrika auf diese deutschpsrachige Website stoßen und sie kann ihm helfen, Strukturen, wie \gls{bspw} die Ungerechtigkeiten, die die traditionelle Aufteilung von Care-Arbeit verursachen, zu verstehen, ist das gut für die Welt. 

Außerdem erhöht es die Legitimation der Arbeit der \gls{ank}, wenn ihre Arbeit auch mehr Impact, eben auch durch Reichweite, verursachen kann. Da die Intentionen der Arbeit der \gls{ank} und auch ihre Qualität in dieser Arbeit als gut bewertet wurden, ist es also insgesamt eine sinnvolle Sache, wenn diese bildenden Informationen auch durch ihren Einsatz im Bildungsbereich Verbreitung finden und eben auch die Chance haben, ein wenig Arbeit der Lehrkräfte abzunehmen, um damit insgesamt das Arbeitsvolumen der Welt in den Griff zu bekommen. 

% teilweiser Notizen Übertrag vom 2024-09-04 Mi. (auch Rückseite verkehrt rum)
% weshalb vergesse ich immer Handlungsbums. Referenz auf Gefühl von geringer Veränderungswirkung. Stichhwort agency. Werden Schüler*innen zu Handlungsfähhigkeit ohnmächtig politische Entwicklungen auf gesellschaftlicher Ebene nachhvollziehen zu können erzogen. Oder werden sie tatsächlich in die Lage versetzt auch gesamtgesellschaftliche Entwicklungen anzustoßen? Keine Ahnung diggah. 

% Notizen Übertrag vom 2024-10-05
% Sich trauen SuS zu "guten" Menschen zu erziehen.. Was ist gut? Dieses Gedankenexperiment mit der zufälligen Geburt. Stichwort: Empathie
% -- Leute ohne Vorstelleungskraft ausschließen 
% -- Extremisten auschließen?
% -- Aber im idealen Diskurs sind auch Stimmen bedacht, welche nicht gut gehört werden -- Habermas
% -- für bbs abkürzen bla bla
% 20225-04-22 John Rawls ist das mit dem Gedankenexperiment, danke Disarstar


\subsection{Handlungsfähigkeit = Reproduktion des Status quo versus Handlungsfähigkeit = Überwindung von Resignation und Zynismus}
Im Zuge der vorliegenden Arbeit wird an verschiedenen Stellen auf das Potential zur Arbeitserleichterung durch vorgefertigtes Unterrichtsmaterial eingegangen. Ein Blick, der pragmatisch ist und in sich Handeln zum Ziel hat. Angesichts der auch im dritten Jahrtausend bewegten Weltgeschichte lässt sich als geschulter Beobachter schnell in Zynismus verfallen. Zu Ende gedacht und gemacht -- \gls{bzw} gelassen -- käme das einem Aufgeben gleich. Den Lehrtätigkeiten und den Wissenschaften, welche einem gelingenden Lernen zuarbeiten sollen, würde ein solches Aufgeben die Legitimationsgrundlage entziehen\footnote{
    Das würde Folgendes bedeuten: 

    \url{https://youtu.be/E9UgBzmU30E?si=3UG7mIwTNfkn2C0H&t=1414} (Zugriff am 21.05.2025) Zeitstempel Beginn schon im Fußnoten-Link \autocite[][Als Meme von $23\,^{\prime}34\,^{\prime\prime}$ bis $23\,^{\prime}50\,^{\prime\prime}$ schauen]{Wolle}. Zur Einordnung, da dort auch % eine Antwort geschrieben steht und weil es % https://proofwiki.org/wiki/Symbols:Prime/Minutes_and_Seconds
    eine Auseinandersetzung an den Konfliktlinien von Pragmatismus, Bürokratie und Rechtsstaatlichkeit illustriert wird zwei Artikel des RedaktionsNetzwerk Deutschland \autocites{Schwarzer.05.02.2021}{Schwarzer.08.02.2021}.
}. Wenn man das Denkmodell des kategorischen Imperativ Immanuel Kants darauf anwendet, erschiene ein Aufgeben unlogisch. Die Erhaltung der Handlungsfähigkeit ist innerhalb des Systems der Wissenschaft und der Bildung\footnote{
    Das gilt natürlich für zahlreiche Bereiche. 
} also inhärent wichtig. 


HIER ZU \GLS{bbk} KRITIK SCHREIBEN, HAT MIR GUT GEFALLEN
\autocite[]{Roler2016}

\subsection{Schlussbetrachtung}
Keine Hinweise im Material auf mögliche andere Interessen der Arbeitgeber und interessensnaher Institutionen.

Aufgabe der Lehrkraft Kontroversitätsgebot möglicherweise klarer darzustellen.

Auf jeden Fall Bestandteil der Lebenswelt der Schüler*innen.

Im Interesse der Schüler*innen informiert zu sein.

Theoretisch auch für zukünftige Führungskräfte. Sogar bei kapitalistischer Konkurrenzlogik im Gegensatz zur Kooperationslogik. Kenne deinen Feind!

Wie gut werden die Intentionen einer Gegenseite aus Arbeitnehmendensicht dargestellt??
Auch das hilft bei der Durchsetzung eigener Interessen.

Das Material hält sich nett an die vom Bildungsplan vorgegebenen Bereiche. 

Der bplan selbst ist weniger an wirtschaftswissenschaftlicher Logik ausgerichtet und nimmt eher das komplexe Gesamtbild von Politik in den Blick.
Verweis auf verzeckte Bremer Ausgestaltung?

ALLES SCHWIERIG




% \paragraph{Was fehlt in dieser Arbeit?}
\paragraph{Reflexion des Bearbeitungsprozesses}
% Die Grenzen und Schwierigkeiten der eigenen Forschung aufzuzeigen, ist Bestandteil des wissenschaftlichen Arbeitens. Die Schwächen einer Abschlussarbeit, die primär den Zweck verfolgt ein Zertifikatserwerb zu rechtfertigen, seperat herauszustellen, ist nicht das standardisierte Vorgehen. 

Zu Beginn der Bearbeitung war geplant, dass ein Analyseraster einen wesentlichen Bestandteil der Arbeit ausmachen soll. 

% Der Erwartungshorizont formuliert kaum präziser Kompetenzen. Er bietet hauptsächlich Lösungen für die Fragestellungen, die sich eindeutig beantworten lassen und sich nicht erst wesentlich im Partner*innen-, Gruppen-, oder Unterrichtsgespräch ergeben. 

Die Aufgabenstellungen und das Material auf die sie sich beziehen, werden \gls{zt} nach den voregegebenen Maßgaben tiefgehend analysiert. Eine anwendbare Rasterung durch die man beliebiges Unterrichtsmaterial schieben kann, um ein ersets Urteil über die Qualität zu erhalten, wurde hingegen nicht erreicht. 








\subsection{Ausblick und Evaluationsschwierigkeiten}
Mit bemüht standardisierten Tests wie \gls{zb} PISA oder Abiturprüfungen wird auch versucht zu evaluieren, wie gut der Lern- und damit auch Lehrerfolg ausgefallen ist. Für verschiedene Bereiche von verschiedenen Fächern hat das Testdesign unterschiedliche Hürden zu überwinden. 
In Biologie kann auswendig gelerntes Wissen vergleichsweise gut getestet werden, Geschicklichkeit und Genauigkeit bei der handwerklichen Durchführung des Experimentierens wird aber kaum auf großskalierter Ebene vergleichbar getestet werden (können).
Grammatik, Orthographie und Aussprache lassen sich ebenfalls gut testen. Kreatives Schreiben hingegen ist in hohem Maße subjektiv und verlangt nach präzise ausformulierten Erwartungshorizonten, um das auszugleichen und überhaupt eine vernünftige Bewertung vornehmen zu können. 
Auch im Fach Politik lässt sich auswendig gelerntes Wissen von historischen Abläufen oder der Institutionenkunde gut auf großangelegter Ebene vergleichbar testen. Aber die politische Handlungsfähigkeit, die ein Stück weit auf die anderen Kompetenzen aufbaut -- dort ist eine Testung in vergleichbarer Weise kaum zu realisieren. 

Das führt zu dem Problem, dass viele Dinge im Unterrichten von Politik ins Blaue hinein gemacht werden muss und mehr auf persönlicher Erfahrung und Einstellung basieren, als auf empirischen Erkenntnissen.
Noch spannender wird die Frage nach dem Erfolg der Bildung, wenn man sie Jahre oder gar Jahrzente später, also die Nachhaltigkeit, evaluieren möchte. Gebetsmühlenartig sage ich auch zu alltäglichen Erklärungsversuchen von vielem: \enquote{(fast) alles ist multikausal}. Entsprechend aufwendig wird eine empirische Annäherung, der im es im Optimalfall gelingt, ein paar hauptsächlich erklärende Variblen ausreichend zu isolieren, um eine Signifikanz feststellen zu können. 


Ein denkbares Testdesign hätte dermaßen viele Variablen auszuklammern, dass erstens die Gruppe riesig sein müsste und zweitens gäbe es viele Variablen, dür die auch schlicht keine gut geeignete Kontrollgruppe gefunden werden kann. Der Einflus von historischen Entwicklungen, die die gesamtgesellschaftliche Ebene betreffen können vielleicht noch geeignet mit vorherigen Genrationen verglichen werden. Bei der steten Entwicklung des Bildungssystems und der Gessellschaft lässt sich aber womöglich nur mit gewagter Datemassage und den wildesten statistischen Kniffen rausrechnen, was jetzt wie viel Einfluss gehabt hat. 
Bei den vorhergehenden Überlegungen denke ich \gls{zb} an den Versuch den Einfluss des Smartphones an Lernerfolg auszuklammern. Schon beim ersten Brainstorming denke ich an den Einfluss die ein stets griffbereites Internt oder ein konstanter Social Media Strom hat. Konzentration, Informationsverarbeitung, die Einflüsse sind zahlreich und für die aktuellen Genrationenin der Schule gibt es kaum eine Kontrollgruppe, da ab einem bestimmten Alter fast alle Menschen ein Smartphone haben. In 20 Jahren den Lernerfolg für die politische Handlungsfähigkeit messen zu wollen, würde ein gottgleiches Testdesign verlangen, wenn es den Anspruch hat, den Einfluss des Unterrichtes von den gesamtgesellschaftlichen getrennt ausmachen zu wollen. 







\clearpage
\newpage
\printbibliography[title=Literaturverzeichnis, heading=bibintoc, nottype=unused] 
% [fields={organization}] geht nicht

\appendix % page 158: https://ftp.rrze.uni-erlangen.de/ctan/macros/latex/contrib/koma-script/doc/scrguide-de.pdf oder: https://ctan.net/macros/latex/contrib/koma-script/doc/scrguide-de.pdf

\section{Appendix}
\subsection{Lizenzierung: \acrfull{c} oder Copyleft?: CC BY-SA 4.0 \label{lizenz}}
In dieser Arbeit werden die gesellschaftlichen Implikationen von der zum Teil \enquote{unfreien}  Veröffentlichung von Wissen genannt und kritisiert -- mal deutlicher, mal mehr im Subtext.
Es mag wohl ungewöhnlich sein eine universitäre Abschlussarbeit mit einer Lizenz zu versehen, die voraussichtlich nie gebraucht wird. Aber es würde sich inkonsequent anfühlen, es nicht zu tun.
Dank der Vorarbeit von Menschen, die sich ähnliche Gedanken gemacht haben, ist die Auswahl einer Lizenz deutlich hürdenfreier, als man vermuten mag. Diese Arbeit steht inspiriert von Copyleft\footnote{Diese Lizenz ist für Software verbreitet. \gls{vgl} \gls{zb} \url{https://www.gnu.org/licenses/copyleft.en.html} (zuletzt geprüft am 10.06.2025)} unter der ähnlichen, für Textprodukte (und viele weitere) gebräuchlicheren, creativecommons Lizenz: \textbf{CC BY-SA 4.0}\footnote{\emph{Creative Commons Attribution-ShareAlike 4.0 International License} \url{https://creativecommons.org/licenses/by-sa/4.0/} (zuletzt geprüft am 10.06.2025).} Die Idee von Copyleft und der CC BY-SA 4.0 Lizenz, ist die Möglichkeit, nahezu alles mit dem Material unter Quellenangabe machen zu dürfen, nur nicht plötzlich freie Werke für unfreie weitere Werke zu verwenden (SA = share alike). 

Dass Share Alike ist dabei eine durchaus starke Einschränkung. Aber eine die bei Produkten (für die sich überhaupt jemand interessiert) viel Freiheit erzwingt -- eine fantatstische rhetorische Figur, finde ich, Freiheit erzwingen\footnote{Dieses Oxymoron passt auch wieder gut zu den Antinomien mit denen Bildung, auch insbesondere politische Bildung im System Schule, konfrontiert ist.}!

\subsection{Kommentar zu Lizenzen}
Fast schon schade, dass ich meine Lebensaufgabe gerade nicht darin sehe Springer VS die kognitive Dissonanz zwischen ihrer Geschäftstätigkeit und den publizierten Inhalten um die Ohren zu hauen. Zumal sie ja nichtmal direkt etwas dafür können. Das System der wissenschaftlichen Publizistik ist auf gönnende Philantropen angewiesen -- \Gls{zb} wenn die Max Planck Society mal Bock hat eine CC Lizenz für eine Veröffentlichung rauszuhauen \autocite[Beispiel von hier:][178, übrigens ohne SA, damit auch die kommerzielle Publizistik Spaß damit haben kann]{Elsasser.2017}. Ansonsten scheint es Usus zu sein, dass Menschen, die ihr Geld hauptsächlich aus einer Anstellung beim Staat verdienen, ihre Forschungsergebnisse bisweilen in einem Verlag veröffentlichen müssen, der sie dann für viel Geld weitervermarktet \autocite[schade, dass sich das wahrscheinlich einige nicht anschauen werden, siw nicht über den akademischen Bereich \gls{oä} Zugang haben. Ist spannend, aber Paywall ftw (e-book {66,99€} am 12.06.2025)]{Schroder.2020}. Und auch die Autor*innen verrichten nur ihre Arbeit brauchen wahrscheinlich, eine prestige-trächtige Veröffentlichung, da Wissenschaft sich auch den Martkzwängen von Reichweite als Priorität unterworfen sieht. Es ist fast schon nervig, sich kein vernünftiges Feindbild gestalten zu können. %, wenn man sein Gehirn benutzt. 
Immer muss ich grummelig auf ein abstraktes System sein. 

Falls jemand das lesen sollte und sich fragt, wie man die Antinomie einer verzwickten Gesellschaft und der eigenen Handlungsfähigkeit an so einer Stelle auflösen könnte ohne über den Umweg der \gls{suub} \gls{oä} zu verfügen: \gls{zb} \url{https://www.sci-hub.ren/} (07.07.2025; SciHub in eine Suchmaschine eingeben, aus guten Gründen ist SciHub redundant und mit mehreren Adresssen angelegt). 
Für die angesprochene Monographie käme man mit der \gls{doi} auch ans Ziel\footnote{
    \url{https://www.wellesu.com/10.1007/978-3-658-30656-4} (07.07.2025)
}. Rechtlich haben die Verlage ein Profit-Interesse daran freies Wissen zu verhindern. Auch wenn sie auf Seite III ihre Buchreihe \gls{bspw} folgendermaßen vorstellen: 
\begin{quote}
    Die Reihe Politische Bildung vermittelt zwischen den vielfältigen Gegenständen des Politischen und der Auseinandersetzung mit diesen Gegenständen in politischen Bildungsprozessen an Schulen, außerschulischen Einrichtungen und Hochschulen. Deshalb werden theoretische Grundlagen, empirische Studien und handlungsanleitende Konzeptionen zur politischen Bildung vorgestellt, um unterschiedliche Zugänge und Sichtweisen zu Theorie und Praxis politischer Bildung aufzuzeigen und zur Diskussion zu stellen. Die Reihe Politische Bildung wendet sich an Studierende, Referendare und Lehrende der schulischen und außerschulischen politischen Bildung.

    \autocite[III]{Schroder.2020}
\end{quote}
Da ich den Text nicht kopieren konnte, habe ich ihn Buchstabe für Buchstabe abgetippt, weil das die Innovation ist, die Springer VS vorantreibt. Anderen würde ich empfehlen Google Lens \gls{oä} auf den Bildschirm zu halten. Die Arbeit mit der eingebauten \gls{ocr} ist ziemlich Nutzer*innen freundlich. % fast leichter als rauskopieren, dabei werden häufig die Absätze mitgenommen und müssen manuell rausgelöscht werden. 

Mit diesem Wissen hoffe ich \enquote{zwischen den vielfältigen Gegenständen des Politischen und der Auseinandersetzung mit diesen Gegenständen in politischen Bildungsprozessen an Schulen, außerschulischen Einrichtungen und Hochschulen} beitragen zu können \autocite[III]{Schroder.2020}. Auch darüber hinaus für eine demokratische Öffentlichkeit, \enquote{um unterschiedliche Zugänge und Sichtweisen zu Theorie und Praxis politischer Bildung aufzuzeigen und zur Diskussion zu stellen.} \autocite[III]{Schroder.2020}. Auch wenn man sich nicht institutionen-offiziell als \enquote{Studierende, Referendare und Lehrende der schulischen und außerschulischen politischen Bildung} qualifizieren durfte \autocite[III]{Schroder.2020}. 


Mal sehen wie freigiebig ich als Lehrer später mit meinem Unterrichtsmaterial sein werde. Es ist natürlich peinlich, miese Qualität zu publizieren. Aber wenn es viele machen, geht es wieder und man könnte gemeinsam an Qualität feilen und kooperieren anstatt zu konkurrieren, was auch zur Effizienz beitragen müsste. Ferner könnte es zur Transparenz von Bildung beitragen und damit auch mehr den Weg zur Demokratisierung dieser öffnen. 










\section{Anhang der Materialkarten \label{MaterialkartenStart}}
Die Materialkarten sind \emph{unveröffentlichte} Entwürfe und NICHT geistiges Eigentum des Autors. 

\includepdfset{    
    pages=-, 
    landscape=true, 
    rotateoversize=true, 
    link=true, 
    linktodoc=true,
    pagecommand={}, % for pagenumbers
    }


\includepdf[
    nup=1x1, 
    addtotoc={
        1, % no idea what this changes, changing number has no effect, it says on page 7:  ⟨page number⟩: Page number of the inserted page. https://texdoc.org/serve/pdfpages.pdf/0
        subsection, 
        2, 
        Angedachtes Layout der Materialkarten, 
        ANKPrototyp
        }
    ]
{AKB-24-032_Prototyp_2024-02-26.pdf}


\includepdf[
    nup=2x1, 
    addtotoc={
        1, 
        subsection, 
        2, 
        Demokratie, 
        DEMOKRATIE
        }
]
{DEMOKRATIE.pdf}


\includepdf[
    nup=2x1, 
    addtotoc={
        1, 
        subsection, 
        2, 
        Arbeitsleben, 
        ARBEITSLEBEN
        }
]
{ARBEITSLEBEN.pdf}


\includepdf[
    nup=2x1, 
    addtotoc={
        1, 
        subsection, 
        2, 
        Wirtschaftspolitik, 
        WIRTSCHAFTSPOLITIK
        }
]
{WIRTSCHAFTSPOLITIK.pdf}

%%%%%%%%%%%%%%%%%%%%%%%%% from page 7-8 of: https://texdoc.org/serve/pdfpages.pdf/0
% Experimental options: (Syntax may change in future versions!)
% addtotoc Adds an entry to the table of contents. This option requires five arguments, separated by commas:
% addtotoc={⟨page number⟩,⟨section⟩,⟨level⟩,⟨heading⟩,⟨label⟩}
% ⟨page number⟩: Page number of the inserted page.
% ⟨section⟩: LATEX sectioning name– e.g., section, subsection, …
% ⟨level⟩: Number, denoting depth of section– e.g., 1 for section level, 2 for
% subsection level, …
% ⟨heading⟩: Title inserted in the table of contents.
% ⟨label⟩: Name of the label. This label can be referred to with \ref and \pageref.

% Note: The order of the five arguments must not be mixed. Otherwise you will get very strange error messages.
% The addtotoc option accepts multiple sets of the above mentioned five arguments, all separated by commas. The sets must be sorted such that the ⟨page number⟩s are in ascending order. (Strictly speaking they must have the same order as the page numbers specified by the pages option.)
% The proper recursive definition of the addtotoc option is:
% addtotoc={⟨toc-list⟩}
% ⟨toc-list⟩ → ⟨page number⟩,⟨section⟩,⟨level⟩,⟨heading⟩,⟨label⟩[,⟨toc-list⟩]

\clearpage
\newpage


\addsec{Dokumentation der Nutzung von KI-basierten Anwendungen und Werkzeugen}
Die folgende Tabelle wurde in Anlehnung an die Vorlage der Universität Bremen erstellt:
\\

\footnotesize{
    \url{https://www.uni-bremen.de/zpa/formulare} führt zu: 
    \\

    \url{https://view.officeapps.live.com/op/view.aspx?src=https%3A%2F%2Fwww.uni-bremen.de%2Ffileadmin%2Fuser_upload%2Fsites%2Fzpa%2Fpdf%2Fallgemein%2FDokumentation_Nutzung_KI_-_AI_Use_Documentation.docx&wdOrigin=BROWSELINK} 
    \\
    
    beide 29.06.2025
    }

% Adaptive column width and multipage support using longtable and tabularx

\clearpage

\newgeometry{left=10mm, right=10mm, top=10mm, bottom=20mm}

\begin{landscape}
\begin{longtblr}[
    caption = {Dokumentation der Nutzung von KI-basierten Anwendungen und Werkzeugen -- Documentation of the Use of AI-based Applications and Tools},
    label = {KIHilfsmittel}]
    {colspec={| c |[1.2pt] X |[dashed] X |[dashed] X |[dashed] X |[dashed] X |}}
\hline
                                & 
    KI-basiertes Hilfsmittel 
        
    AI-based Tool               & 
    Einsatzform 

    Purpose                     & 
    Betroffene Teile der Arbeit 
        
    Aspect of the Work Affected & 
    Beschreibung der Eingabe (Prompt) 
  
    Prompt (Entry)              & 
    Bemerkung 
    
    Comment                     \\ 
    \hline[1.2pt]

    
    1                                                                                                                                           & 
    GitHub Copilot (Chat) in Visual Studio Code

    Inline Chat \& Chat on Secondary Sidebar

    mostly with GPT-4.1 as \gls{llm}                                                                                                            & 
    Fragen zu \LaTeX{} Code, keine inhaltlichen Fragen                                                                                          &
    alle, insbesondere \gls{zb} der Code zur Form dieser Tabelle oder Fragen zum Darstellen von eingebunden PDF-Seiten im Inhaltsverzeichnis    & 
    \gls{zb} \enquote{ia there \textbackslash{}vfill in latex?},  
    
    \enquote{what does \textbackslash{}arraybackslash in tabularx do?}, 
    
    \enquote{what can the parameter in lines 14 or 27 change?} oder
    
    \enquote{how do I make several entries to the toc like I treid in lines 26 to 62?}                                                          & 
    Es werden nicht alle Prompts aufgeführt. Es wurde ausschließlich für den Code Hintergrund benutzt und nie für den Inhalt. 
    
    Das Meiste wurde dann ohnehin doch wieder über Suchmaschinen und dann Foren und Anleitungen lesen erledigt. Häufig eben erst im Anschluss an das initiale Ausprobieren von \gls{ki} \\ 
    \hline
    %%%%%%%


    2                                                                                                                               &
    Visual Studio Code Autovervollständigung                                                                                        &
    Codevervollständigung und Autovervollständigung von einzelnen Worten, ähnlich dem Tippen auf Smartphones                        &
    alle                                                                                                                            &
    keine Prompts. 
    
    Bei der Eingabe \enquote{Effiz} wird dann \gls{zb} \enquote{Effizienz} vorgeschlagen und nach Druck auf Enter ausgeschrieben    &
    Die Grenze von maschinellem Lernen und Skripts oder überhaupt Software Hilfsmitteln ist fließend. Auch wenn schon länger existierende Autovervollständigungen für einzelne Worte keinen \gls{ki}-Boom und derartige gesellschaftliche Diskurse ausgelöst hat, wie es die \gls{llm} derzeit hervorrufen.                                             \\ 
    \hline

% \pagebreak

    3                                               &
    ChatGPT.com                                     &
    Erstellen einer Graphik                         &
    \gls{abb} \ref{Absetzungsbetrag} auf \gls{S} \pageref{Absetzungsbetrag}                                      &
    \enquote{Bitte plotte  die vier Funktionen unten in einer Abbildung. Beschrifte die x-Achse in hunderter Schritten. Mach keine Überschrift. Nenne die x-Achse anzurechnendes Einkommen in €. Nenne die y-Achse Absetzungsbetrag in € \\

    f (x) = x Df = {0 < x < 100} (1) \\

    f (x) = 0, 2x + 80 Df = {100 < x < 520} (2) \\

    f (x) = 0, 3x + 28 Df = {520 < x < 1000} (3) \\

    f (x) = 0, 1x + 228 Df = {1000 < x < 1200} (4)}
                                                    &
    Danach wurde mit weiteren Prompts das Aussehen der Grafik noch weiter verfeinert. Weil ich mir kein ChatGPT Plus oder Pro leisten kann, wurden mir keine Grafiken mehr ausgegeben. Also wurde Python installiert, der generierte Code kopiert und Visual Studio Code eingefügt, um eine Grafik mit gleichmäßiger Skalierung auf dem eigenen Rechner zu erstellen.\\
    \hline

\end{longtblr}
\end{landscape}

\restoregeometry

% \footnote{\url{https://www.uni-bremen.de/zpa/formulare} führt zu \url{https://view.officeapps.live.com/op/view.aspx?src=https%3A%2F%2Fwww.uni-bremen.de%2Ffileadmin%2Fuser_upload%2Fsites%2Fzpa%2Fpdf%2Fallgemein%2FDokumentation_Nutzung_KI_-_AI_Use_Documentation.docx&wdOrigin=BROWSELINK} beide 29.06.2025}


\includepdf[
    landscape=false, 
    addtotoc={
    1, section, 1, Eigenständigkeitserklärung, EIGENSTÄNDIGKEITSERKLÄRUNG
    }
]{Eigenstaendigkeitserklaerungen__schriftlArbeiten_deu_engl.pdf}

\end{document}