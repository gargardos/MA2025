\titlehead{
    \includegraphics[width=0.5 \linewidth]{
    UniBremen_Logo_Rot-Schwarz_Web_NICHT_VERAENDERN.png
    }
}

\subject{
    Masterarbeit
    }

\title{
    Chancen und Risiken von Fremdmaterial
    } 

\subtitle{
    Eine beispielhafte Analyse und Wertung von Unterrichtsmaterial der Arbeitnehmerkammer Bremen für den Politikunterricht an berufsbildenden Schulen in Bremen
    }

\author{
    Paul Aljoscha Klein\\ 
    \normalsize
    $\ast$ 20.07.1995\\ 
    \normalsize
    $\#$ 4134166
    }

\date{Abgabe: 08. Juli 2025}

\publishers{
    \vfill % doesn't seem to help
    \raggedright{
    Erstprüfer: Prof. Dr. Andreas Klee\\
    Zweitprüferin: Dr. Eva Anslinger
    } 
    \hfill % bigger graphic-width inserts bigger vertical space between the two lines with the names. hfill doesn't really help % hfill seems unnecassary due to raggedleft, but without, the second name line gets pulled right
    \raggedleft{
    \includegraphics[width=0.2 \linewidth]{
    zap_Logo_Website.001.png % Bildquelle: https://www.uni-bremen.de/fileadmin/user_upload/sites/soha/zap_Logo_Website.001.png 2025-06-19
    }
    } % raggedleft is needed in this setup
    }


% So sieht's gut aus mit titlepage bei Dokumentklasse scartcl, Danke Markus Kohm

%%%%%%%%%%%%% TitelBrainStorming %%%%%%%%%%%%%%%%%%%%%%%%%
% Nutzung von fremden Unterrichtsmaterialien. Eine exemplarische Analyse für den Politikunterricht an berufsbildenden Schulen in Bremen
% Nutzung von Unterrichtsmaterialien einer externen Institution. Eine exemplarische Analyse für den Politikunterricht an berufsbildenden Schulen in Bremen

% Entwürfe 2025-03-04 Di.
% Nutzung von fremden Unterrichtsmaterialien. Eine exemplarische Analyse für den Politikunterricht an berufsbildenden Schulen in Bremen

% Analyse und Wertung von großflächig angelegtem Unterrichtsmaterial Materialvorlagen: Exemplarisches Vorgehen mit Material der Arbeitnehmerkammer Bremen.

% Analyse und Wertung von Materialvorlagen für den Politikunterricht an berufsbildenden Schulen in Bremen: Eine beispielhafte Analyse mit Material der Arbeitnehmerkammer Bremen.

% Gekürzt:
% Analyse und Wertung von Materialvorlagen: Eine beispielhafte Analyse anhand von Unterrichtsmaterial für den Politikunterricht an berufsbildenden Schulen in Bremen der Arbeitnehmerkammer Bremen.

% Die Realität der Nutzung von institutionellem Unterrichtsmaterial entgegen selbsterstelltem Material: Eine beispielhafte Analyse anhand von Unterrichtsmaterial der Arbeitnehmerkammer Bremen für den Politikunterricht an berufsbildenden Schulen in Bremen.

% Nur selbsterstelltes Unterrichtsmaterial? Chancen und Risiken von Fremdmaterial: Eine beispielhafte Analyse und Wertung von Unterrichtsmaterial der Arbeitnehmerkammer Bremen für den Politikunterricht an berufsbildenden Schulen in Bremen.

% Unterricht von der Stange? Chancen und Risiken von Fremdmaterial: Eine beispielhafte Analyse und Wertung von Unterrichtsmaterial der Arbeitnehmerkammer Bremen für den Politikunterricht an berufsbildenden Schulen in Bremen

% Kriterien zur Bewertung von externen Unterrichtsmaterialien im Politikunterricht. Am Beispiel 
% Geschenke nimmt man gerne an
% Unterricht von der Stange
% Erleichterung durch fertiges Unterrichtsmaterial
%%%%%%%%%%%%%%%%%%%%%%%%%%%%%%%%%%%%%%%%%%%%%%%%%%%%%%%%%%%%%%%%%