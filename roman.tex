\maketitle
\clearpage

\pagenumbering{roman}

\section*{Hinweis zur Leser*innen Führung}
Wer das Dokument digital liest, kann die Zahlen anklicken, die auf einen \gls{abs}, eine \gls{S} oder eine Fußnote innerhalb dieses Dokumentes verweisen. 
Auch bei Literaturverweisen kann auf das Jahr (oder \gls{ebd}) geklickt werden, um direkt ins Literaturverzeichnis zu gelangen. 


\clearpage
\newpage

\tableofcontents % war früher mal vor pagenumbering. Macht das einen Unterschied? Keine AHnung 2024-06-05 ah ja, will ja nicht das Inhaltsverzeichnis Seitenummeriert haben 2024-06-06

% \clearpage
% \newpage

\printacronyms[type=\acronymtype, 
    nonumberlist, 
    title=Abkürzungsverzeichnis, 
    toctitle=Abkürzungsverzeichnis]{} % Acronymtype ist von glossaries, title und toctitle sind Titel im Inhaltsverzeichnis
% listof=totoc in \documentclass[] replaces %% , title={Abkürzungsverzeichnis}, toctitle={Abkürzungsverzeichnis} %% here
% \vspace{24pt} % vertical space

\listoffigures % Abbildungsverzeichnis
\listoftables % Tabellenverzeichnis

\clearpage
\newpage

\setcounter{page}{1}
\pagenumbering{arabic}