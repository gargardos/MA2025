
\addsec{Dokumentation der Nutzung von KI-basierten Anwendungen und Werkzeugen}
Die folgende Tabelle wurde in Anlehnung an die Vorlage der Universität Bremen erstellt:
\\

\footnotesize{
    \url{https://www.uni-bremen.de/zpa/formulare} führt zu: 
    \\

    \url{https://view.officeapps.live.com/op/view.aspx?src=https%3A%2F%2Fwww.uni-bremen.de%2Ffileadmin%2Fuser_upload%2Fsites%2Fzpa%2Fpdf%2Fallgemein%2FDokumentation_Nutzung_KI_-_AI_Use_Documentation.docx&wdOrigin=BROWSELINK} 
    \\
    
    beide 29.06.2025
    }

% Adaptive column width and multipage support using longtable and tabularx

\clearpage

\newgeometry{left=10mm, right=10mm, top=10mm, bottom=20mm}

\begin{landscape}
\begin{longtblr}[
    caption = {Dokumentation der Nutzung von KI-basierten Anwendungen und Werkzeugen -- Documentation of the Use of AI-based Applications and Tools},
    label = {KIHilfsmittel}]
    {colspec={| c |[1.2pt] X |[dashed] X |[dashed] X |[dashed] X |[dashed] X |}}
\hline
                                & 
    KI-basiertes Hilfsmittel 
        
    AI-based Tool               & 
    Einsatzform 

    Purpose                     & 
    Betroffene Teile der Arbeit 
        
    Aspect of the Work Affected & 
    Beschreibung der Eingabe (Prompt) 
  
    Prompt (Entry)              & 
    Bemerkung 
    
    Comment                     \\ 
    \hline[1.2pt]

    
    1                                                                                                                                           & 
    GitHub Copilot (Chat) in Visual Studio Code

    Inline Chat \& Chat on Secondary Sidebar

    mostly with GPT-4.1 as \gls{llm}                                                                                                            & 
    Fragen zu \LaTeX{} Code, keine inhaltlichen Fragen                                                                                          &
    alle, insbesondere \gls{zb} der Code zur Form dieser Tabelle oder Fragen zum Darstellen von eingebunden PDF-Seiten im Inhaltsverzeichnis    & 
    \gls{zb} \enquote{ia there \textbackslash{}vfill in latex?},  
    
    \enquote{what does \textbackslash{}arraybackslash in tabularx do?}, 
    
    \enquote{what can the parameter in lines 14 or 27 change?} oder
    
    \enquote{how do I make several entries to the toc like I treid in lines 26 to 62?}                                                          & 
    Es werden nicht alle Prompts aufgeführt. Es wurde ausschließlich für den Code Hintergrund benutzt und nie für den Inhalt. 
    
    Das Meiste wurde dann ohnehin doch wieder über Suchmaschinen und dann Foren und Anleitungen lesen erledigt. Häufig eben erst im Anschluss an das initiale Ausprobieren von \gls{ki} \\ 
    \hline
    %%%%%%%


    2                                                                                                                               &
    Visual Studio Code Autovervollständigung                                                                                        &
    Codevervollständigung und Autovervollständigung von einzelnen Worten, ähnlich dem Tippen auf Smartphones                        &
    alle                                                                                                                            &
    keine Prompts. 
    
    Bei der Eingabe \enquote{Effiz} wird dann \gls{zb} \enquote{Effizienz} vorgeschlagen und nach Druck auf Enter ausgeschrieben    &
    Die Grenze von maschinellem Lernen und Skripts oder überhaupt Software Hilfsmitteln ist fließend. Auch wenn schon länger existierende Autovervollständigungen für einzelne Worte keinen \gls{ki}-Boom und derartige gesellschaftliche Diskurse ausgelöst hat, wie es die \gls{llm} derzeit hervorrufen.                                             \\ 
    \hline

% \pagebreak

    3                                               &
    ChatGPT.com                                     &
    Erstellen einer Graphik                         &
    \gls{abb} \ref{Absetzungsbetrag} auf \gls{S} \pageref{Absetzungsbetrag}                                      &
    \enquote{Bitte plotte  die vier Funktionen unten in einer Abbildung. Beschrifte die x-Achse in hunderter Schritten. Mach keine Überschrift. Nenne die x-Achse anzurechnendes Einkommen in €. Nenne die y-Achse Absetzungsbetrag in € \\

    f (x) = x Df = {0 < x < 100} (1) \\

    f (x) = 0, 2x + 80 Df = {100 < x < 520} (2) \\

    f (x) = 0, 3x + 28 Df = {520 < x < 1000} (3) \\

    f (x) = 0, 1x + 228 Df = {1000 < x < 1200} (4)}
                                                    &
    Danach wurde mit weiteren Prompts das Aussehen der Grafik noch weiter verfeinert. Weil ich mir kein ChatGPT leisten kann wurden mir keine Grafiken mehr ausgegeben. Also wurde Python installiert, der generierte Code kopiert und Visual Studio Code eingefügt, um selbsttätig eine Grafik mit gleichmäßiger Skalierung auf dem eigenen Rechner zu erstellen.\\
    \hline

\end{longtblr}
\end{landscape}

\restoregeometry

% \footnote{\url{https://www.uni-bremen.de/zpa/formulare} führt zu \url{https://view.officeapps.live.com/op/view.aspx?src=https%3A%2F%2Fwww.uni-bremen.de%2Ffileadmin%2Fuser_upload%2Fsites%2Fzpa%2Fpdf%2Fallgemein%2FDokumentation_Nutzung_KI_-_AI_Use_Documentation.docx&wdOrigin=BROWSELINK} beide 29.06.2025}