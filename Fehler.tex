

\footnote{
    Um den Begriff \emph{TradWife}, die traditionelle Frau, die \enquote{granola from scratch} macht, wird der patriarchale Reichtum ihres Mannes zur Schau gestellt, indem sie nie bei \enquote{echter} Sorgearbeit zu sehen ist. Es wurde geschafft die \emph{Trophy Wife} nicht nur in der Sphäre ihres obejktifizierten Körpers für patriarchale Herrschafts-Hegemonie zu nutzen -- jetzt wird der \emph{TradWife} auch scheinbar eigene Selbstwirksamkeit \emph{zugestanden}, indem sie performative Arbeit verrichten darf. Eine Hierarchisierung zu Menschen, die selbst putzen und Windeln wechseln müssen wird subtil aufgebaut. Andere Lebensrealitäten werden implizit als minderwertig dargestellt. Die Dinge, die in einem 20 sekündigen Video unterschwellig kommuniziert werden, zu durchdringen ist nicht trivial. Da Beispiele wie dieses aber omnipräsent sind, ist Medienkompetenz wichtig. 
    
    \noindent Weshalb Sorgearbeit eine wichtige Konfliktlinie darstellt, taucht in Zusamenhang zur Lohnarbeit und Gerechtigkeitsfragen übrigens immer wieder im Material der \gls{ank} auf.
}. 






% Ein Schulbuch kann sehr gut sein, wenn das Budget knapp ist, ist die Chance hoch, das stattdesssen ein nicht aktuelles Schulbuch von geringerer Qualität in der Praxis genutzt wird. 

\subsection{Fragen an das Material} % Nach welchen Maßstäben Unterrichtsmaterial analysieren?
Ein eher offen formulierter Bildungsplan ist kein Zufall. % Aus ähnlichen Gründen, die einen offen formulierten Bildungsplan nahelegen,
Daher wäre es kontraindiziert, Unterrichtsmaterial nach starren Vorgaben zu bewerten.
Dennoch soll eingegrenzt werden, nach welchen Maßstäben Unterrichtsmaterial bewertet werden könnte. Offen bedeutet nicht beliebig.

Die erste Anlaufstelle dafür ist der Bildungsplan selbst. In Anlehnung an die Gliederung des Bildungsplans kann das Unterrichtsmaterial anhand folgender Punkte untersucht werden:
\begin{itemize} 
    \item Kompetenzen jeweils der beruflichen \& politischen Bildung
    \item Lebensweltorientierungen % Modell der didaktischen Rekonstruktion
    \begin{itemize}
        \item Arbeits-,  Berufs- und Lebensweltorientierungen
        \item Problem-, und Wissenschaftsorientierungen
        \item Zukunfts-, Gegenwarts- und Vergangenheitsorientierungen
    \end{itemize}
    \item Methodische Grundsätze
    \item Und anhand folgender sieben politischen Handlungsfelder:
    \begin{itemize}
        \item Demokratie 
        \item Gesellschaft 
        \item Arbeitsleben
        \item Öffentlichkeit im digitalen Zeitalter
        \item Wirtschaftspolitik
        \item Globale Zusammenhänge 
        \item Nachhaltigkeit 
    \end{itemize}
\end{itemize}

Im Bildungsplan sind die drei wichtigsten Kompetenzen in Anlehnung an den \gls{bbk} formuliert. 

\begin{itemize}
    \item Politische Urteilsfähigkeit
    \item Politische Handlungsfähigkeit
    \item Methodische Fähigkeiten 
\end{itemize}

Daher soll im nächsten Schritt analysiert werden, inwieweit das Unterrichtsmaterial diese Kompetenzen fördert. 

% Maßgeblich dafür können Vergleiche zu Bewertungskriterien sein, die bereits von anderen genutzt worden sind.
% Auch maßgeblich soll sein, inwiefern schon das Unterrichtsmaterial in Bezug auf den Bildungsplan und dessen, unter Anderem aus dem \gls{bbk} abgeleiteten, Kriterien vereinbar scheint. 


Fragen, die darüber hinaus und ergänzend an das Material gestellt werden sollen, sind:
\begin{itemize}
    \item Welche Kompetenzen werden an welcher Stelle gefördert?
    \item Wie und durch welche Operationalisierung werden die Kompetenzen gefördert?
    \item Wie werden Schüler*innenvorstellungen berücksichtigt? % zB diametral entgegen falscher Vorstellungen oder auf richtige Vorstellungen aufbauend? Außerdem: Stichwort: Reifizierung \autocite[]{Reinfried2009}
    Siehe auch die reflexiven Fragen für das Modell der didaktischen Rekonstruktion bei \textcite[411-412]{Reinfried2009}.
    \item An welchen Stellen soll induktiv, an welchen deduktiv vorgegangen werden?
    \item Ist das im Modell der Didaktischen Rekonstruktion sinnvoll? \autocite[]{Reinfried2009} Sollte im Sinne des Konstruktivismus besonders induktives Vorgehen seitens der Schüler*innen antizipiert werden?
    \item Wie wird Schüler*innenaktivität erzeugt? % Für Konstruktivismus wichtig
    \item Wird bestehendes Material benutzt und analysiert oder wird auch eigene Produktion angeregt?
    \item Wie ist die Ergebnissicherung eingebunden?
    \item Welche Handlungsfelder werden angesprochen?
    \item Inwieweit bietet ist das Material auf die Lebensrealität und realistische Partizipationsmöglichkeiten ausgerichtet?
    \item Inwieweit impliziert das Material politisches Handeln, welches bestehende Systeme in Frage stellt? Ist das noch vertretbar (im Konkreten: mit dem \gls{bbk} vereinbar)? Ist im Gegenzug ein Weglassen solcher Perspektiven vertretbar (im Konkreten: mit dem \gls{bbk} vereinbar)?
    \item Welche Medien werden eingesetzt? Wie ist die Wahl der Medien begründet?
    \item Medienkompetenz % siehe Demokratie ANK Dinger Kommentare für die Quellen \emph{M1} und \emph{M2} letzte Seite
    \item Inklusion
    \item Digitalisierung (Methodenkompetenz. Sinnvoll eingesetzt?)
    \item Was wird vom Material an Möglichkeiten der Binnendifferenzierung geboten?
    \item Wo findet eine Binnendifferenzierung statt; sowohl im Inhalt als auch den Methoden? Ist die Wahl der Methoden aus der Didaktik zu begründen? Welche alternative Methoden wären möglich gewesen? Werden unterschiedliche Methoden für unterschiedliche Lerngruppen Angeboten?
    \item Handhabbarkeit, Anwendbarkeit, praktisches, pragmatisch. Handwerkliche Betrachtung
    \item Umfang, Zeitvorgaben, zur Verfügung Stellung realistisch?
    \item Unwahrscheinlich: Aber, ist das Material Altersgruppen geeignet oder gar übergriffig?
    \item Welche Beeinflussungen sind zu erkennen? Lassen die sich diese legitimieren?
\end{itemize}
Darüber hinaus soll untersucht werden, inwieweit sich Intentionen der Arbeitnehmerkammer Bremen im Material finden lassen und ob die Beeinflussungen des Akteurs sich in einem Rahmen bewegen, welcher der Intention des \gls{bbk} nicht entgegensteht. \#Kontroversitätsgebot 

\subsection{Quellenangaben}
\begin{itemize}
    \item Wie lässt sich die Qualität der Quellen bewerten?
    \item Wie wird auf Medienkompetenz eingegangen?
 \end{itemize}





















\subsection{Bildungsplan als Analysemaßstab}
% Der Bildungsplan \autocite{bplan} in Politik für duale Studiengänge im Land Bremen\dots

Wie zu Kapitelanfang angedeutet ist der Bildungsplan vergleichsweise offen formuliert, ohne detaillierte Vorgaben zu machen. Angesichts der überbordenden Themenauswahl und interdisziplinären Verstrickung eines Faches wie Politik eine konsequente Entscheidung, die im Bildungsplan selbst argumentativ legitimiert wird \autocite[diggah, welche Seite habe ich das gelesen]{bplan}. Reinhold \textcite[17-18]{Hedtke2016} kritisiert für den Wirtschaftsunterricht den fehlenden Bezug auf die Fachwissenschaften sowie auch eine fehlende Berücksichtigung der zahlreichen Interdependenzen zu benachbarten Fachwissenschaften. 















Im Sachunterricht in der Elementarstufe wird kritisiert, wenn der Unterricht wenig mit der eigentlichen Sache zu tun hat \autocite[2-4]{Scholz2004}. Im Politikunterricht ist das Anschauen jedoch weniger auf Dinge bezogen. Analog dazu lässt sich jedoch die bereits im \gls{bbk} geforderte Handlungskompetenz sehen. Spannend ist es also zu untersuchen, inwiefern durch das Material politisches Handeln beobachtbar oder gar an der eigenen Gruppe erlebbar wird.







    \item 
    \item Überraschung, ich habe extra \emph{Ding} geschrieben. Weil das ziemlich ersetzbar für viele (informationslastige) Produkte ist, welche mit öffentlichen Ressourcen realisiert werden.
    \item diggah, als ob, ich habe kein bock mehr. es wird eh schlimmer anstatt besser

Programmierende Freunde von mir scherzen über \emph{pull requests} über \enquote{git} für eine Möglichkeit moderne, digitale Möglichkeiten für direktere Formen der Demokratie nutzen zu können.



BLUB BLUB





Insbesondere Funktionen zu forken und einen zweiten Branch aufzumachen bevor gemergt wird ist hier logische Grundlage und in der Entwicklung von Software mit git auch de-facto Standard.)
    \item Es sollte einen Versionverlauf haben (Ebenfalls de-facto Standard in der Software Entwicklung). Die Open Knowledge Foundation (ein eingetragener Verein) hatte schon eigenständig versucht Gesetzgebeung, welche viele strukturelle Ähnlichkeiten zu den Versionsverläufen von Software aufweist, in git abzubilden. Das Projekt ist jedoch nicht aktuell. Die Idee bleibt.








% \paragraph{Was fehlt in dieser Arbeit?}
\paragraph{Reflexion des Bearbeitungsprozesses}
% Die Grenzen und Schwierigkeiten der eigenen Forschung aufzuzeigen, ist Bestandteil des wissenschaftlichen Arbeitens. Die Schwächen einer Abschlussarbeit, die primär den Zweck verfolgt ein Zertifikatserwerb zu rechtfertigen, seperat herauszustellen, ist nicht das standardisierte Vorgehen. 

Zu Beginn der Bearbeitung war geplant, dass ein Analyseraster einen wesentlichen Bestandteil der Arbeit ausmachen soll. 

% Der Erwartungshorizont formuliert kaum präziser Kompetenzen. Er bietet hauptsächlich Lösungen für die Fragestellungen, die sich eindeutig beantworten lassen und sich nicht erst wesentlich im Partner*innen-, Gruppen-, oder Unterrichtsgespräch ergeben. 

Die Aufgabenstellungen und das Material auf die sie sich beziehen, werden \gls{zt} nach den voregegebenen Maßgaben tiefgehend analysiert. Eine anwendbare Rasterung durch die man beliebiges Unterrichtsmaterial schieben kann, um ein ersets Urteil über die Qualität zu erhalten, wurde hingegen nicht erreicht. 











\subsection{Handlungsfähigkeit = Reproduktion des Status quo versus Handlungsfähigkeit = Überwindung von Resignation und Zynismus}
Im Zuge der vorliegenden Arbeit wird an verschiedenen Stellen auf das Potential zur Arbeitserleichterung durch vorgefertigtes Unterrichtsmaterial eingegangen. Ein Blick, der pragmatisch ist und in sich Handeln zum Ziel hat. Angesichts der auch im dritten Jahrtausend bewegten Weltgeschichte lässt sich als geschulter Beobachter schnell in Zynismus verfallen. Zu Ende gedacht und gemacht -- \gls{bzw} gelassen -- käme das einem Aufgeben gleich. Den Lehrtätigkeiten und den Wissenschaften, welche einem gelingenden Lernen zuarbeiten sollen, würde ein solches Aufgeben die Legitimationsgrundlage entziehen\footnote{
    Das würde Folgendes bedeuten: 

    \url{https://youtu.be/E9UgBzmU30E?si=3UG7mIwTNfkn2C0H&t=1414} (Zugriff am 21.05.2025) Zeitstempel Beginn schon im Fußnoten-Link \autocite[][Als Meme von $23\,^{\prime}34\,^{\prime\prime}$ bis $23\,^{\prime}50\,^{\prime\prime}$ schauen]{Wolle}. Zur Einordnung, da dort auch % eine Antwort geschrieben steht und weil es % https://proofwiki.org/wiki/Symbols:Prime/Minutes_and_Seconds
    eine Auseinandersetzung an den Konfliktlinien von Pragmatismus, Bürokratie und Rechtsstaatlichkeit illustriert wird zwei Artikel des RedaktionsNetzwerk Deutschland \autocites{Schwarzer.05.02.2021}{Schwarzer.08.02.2021}.
}. Wenn man das Denkmodell des kategorischen Imperativ Immanuel Kants darauf anwendet, erschiene ein Aufgeben unlogisch. Die Erhaltung der Handlungsfähigkeit ist innerhalb des Systems der Wissenschaft und der Bildung\footnote{
    Das gilt natürlich für zahlreiche Bereiche. 
} also inhärent wichtig. 

HIER ZU \GLS{bbk} KRITIK SCHREIBEN, HAT MIR GUT GEFALLEN
\autocite[]{Roler2016}







\subsection{Schlussbetrachtung}
Keine Hinweise im Material auf mögliche andere Interessen der Arbeitgeber und interessensnaher Institutionen.

Aufgabe der Lehrkraft Kontroversitätsgebot möglicherweise klarer darzustellen.

Auf jeden Fall Bestandteil der Lebenswelt der Schüler*innen.

Im Interesse der Schüler*innen informiert zu sein.

Theoretisch auch für zukünftige Führungskräfte. Sogar bei kapitalistischer Konkurrenzlogik im Gegensatz zur Kooperationslogik. Kenne deinen Feind!

Wie gut werden die Intentionen einer Gegenseite aus Arbeitnehmendensicht dargestellt??
Auch das hilft bei der Durchsetzung eigener Interessen.

Das Material hält sich nett an die vom Bildungsplan vorgegebenen Bereiche. 

Der bplan selbst ist weniger an wirtschaftswissenschaftlicher Logik ausgerichtet und nimmt eher das komplexe Gesamtbild von Politik in den Blick.
Verweis auf verzeckte Bremer Ausgestaltung?

ALLES SCHWIERIG











% des \emph{impliziten Lernen}
% So wird in einer Aufgabe zum Gender Pay Gap in \emph{Arbeitsleben A2 mit M5}

% \enquote{Arbeitsleben B1} Debatte

% im \enquote{Monstrum Bürokratie} (Friedrich Merz 2025, andauernd) mittelfristig umgesetzt zu erleben. 

% bezeichneten Methoden dargeboten             explizit als

% Nach wie vor gilt der Ausspruch des englischen Staatsmannes Winston Churchill vom 11. November 1947 bei einer Rede im Unterhaus: "Demokratie ist die schlechteste aller Regierungsformen – abgesehen von all den anderen Formen, die von Zeit zu Zeit ausprobiert worden sind." Oder, mit dem Demokratieforscher Manfred G. Schmidt formuliert, " […] die zweitbeste Demokratie ist immer noch besser als die beste Nicht-Demokratie". Die Demokratie mag nur als das kleinere Übel angesehen werden, vereint aber andererseits so viele Vorteile auf sich, dass sie weiterhin als die beste bekannte Herrschaftsform bezeichnet werden kann.
%% https://www.bpb.de/shop/zeitschriften/izpb/demokratie-332/248593/demokratie-in-der-krise-und-doch-die-beste-herrschaftsform/ 16.06.2025
% JUNGE ICH WILL NICHT MEHR
An anderer Stelle bezieht sich Frau Zandonella auf eine Studie, welche \enquote{erstmals für Deutschland nachweisen [konnte], dass politische Entscheidungen mit höherer Wahrscheinlichkeit mit den Einstellungen höherer Einkommensgruppen übereinstimmen, wohingegen für einkommensarme Gruppen entweder keine systematische Übereinstimmung festzustellen ist oder sogar ein negativer Zusammenhang} \autocite[177]{Elsasser.2017}.



\begin{quote}
    Lernende sind nicht dumm. Wenn in der Schule demokratische Übungen immer das bleiben: Übungen. Lernen sie womöglich, dass ihr politischen Ideen (also ihr Handeln im schulischen Wirkungsbereich) keine Auswirkungen hat. Das halte ich für gefährlich.
\end{quote}











Die Qualität der Quelle von der \gls{ank} ist als hoch anzusehen. Für Interviews nicht unbedingt Standard, sind mehrere getroffene Aussagen von Martina Zandonella mit einem Link belegt. Der eine Link führt zum Beispiel zu einer Präsentation\footnote{
    \url{https://www.arbeitnehmerkammer.de/fileadmin/user_upload/Veranstaltungen/Veranstaltungsdokumentation/Downloads/Demokratische_Dividende_Jirjahn.pdf} (besucht am 28.06.2025)
} von Uwe Jirjahn, Porfessor an der Universität Trier.






Von anderer Seite ließe sich bestimmt argumentieren, dass es auch gefährlich ist, wenn Menschen dazu erzogen werden, ihre Bedingungen als veränderbar wahrzunehmen. Zum Beispiel indem sie sich als kooperierend sehen und ihnen das Zusammenschließen als Option bewusst ist.
Aber da genau das vom demokratischen Gedanken gedeckt ist, können die auf Rechtsgrundlage ihre faschistische Fresse halten. HIER WISSENSCHAFTLICH MECKERN MEIN FREUND, ABER RECHT HAST DU JA





Die Inhalte der \gls{ank} sind recht transparent gestaltet und belegen ihre Ausasgen meist. Das bedeutet nicht, dass sich kein Bias erkennen lässt. Aber es wird argumentiert und Aussagen zu bestehendem Wissen verknüpft. 
\bigskip

Politik ist Formgebend








Als Begründung für wildes Quellengebashe:
Es kann natürlich entschieden werden, unterschiedliche Maßstäbe anzulegen und starke Hierarchien und Ungleichheiten auszuleben. Die Wahrscheinlichkeit damit etwas zwischen Trotzreaktion und Gegenwehr bis hin zu Desinteresse und Lethargie auszlösen, dürfte damit aber ansteigen. 

\citetitle{Elsasser.2017} 

Wahlbeteiligung vgl. \enquote{Demokratie}

Da Rhetorik durchaus hilfreich sein kann, soll folgende Aussage die Problemstellung illustrieren:











\subsection{Ökonomische versus politische Sozialisation?}
In der Forschung zu politischer Bildung in staatlichen Institutionen wird seit geraumer Zeit verhandelt, wie die Einflussnahme externer Akteure zu bewerten sei\footnote{\gls{vgl} zahlreiche Arbeiten von Reinhold \textcite[\gls{zb}][]{Hedtke2016}}. 

Autor*innen, welche einen bildungstheoretischen und/oder einen politikwissenschaftlichen Hintergrund aufweisen, lesen sich insofern ähnlich, als dass ihnen gemein ist eine gute politische Bildung interdisziplinär zu gestalten.

% Insbesondere für den berufsbildenden Bereich lässt sich feststellen, dass Forderungen erwachsen, die \enquote{betriebswirtschaftlichen Verwertungslogiken}

% Reinhold Hedtke 
% dritte Säule \autocite[]{kerschensteiner1966}
% Diggah, wie soll ich Forschungsüberblick geben, wenn alleine Reinhold Hedtke gefühlt gut 200 Veröffentlichungen zur gleichen schmackhaften Soße hat?








\subsection{Subjektive Einschätzung}
Gleichwohl das wissenschaftliche Arbeiten nach Objektivität streben sollte und sich diverser Kulturtechniken bedient, um diesem Ideal nahe zu kommen, ist eine subjektive Einflussnahme der Forschenden kaum auszuschließen. Aus diesem Grunde soll im Anschluss an % wirklich nach?
die kriteriengeleitete Analyse noch der subjektive Eindruck des Autors dargestellt werden. 





%%%%%%%%%%%%%%%%%%%%%%%%%%%%%%%%%%    Das leidige Thema mit den GESETZEN    %%%%%%%%%%%%%%%%%%%%%%%%%%%%%%%%%%%%%%%%%%%%%%%%%%
VIELLEICHT ALLGEMEIN ÜBER BESCHISSENEN VERSIONSVERLAUF VON GESETZEN REDEN?

Gesetze unterliegen zahlreichen Änderungen. In aller Regel werden beschlossene Änderungen von Bundesgetzen in einem \gls{bgbl} veröffentlicht  

% HIERRRRRRR FEHLER.TEX EINFÜGEN

So kann auch in diesem Fall der Verlauf vom Entwurf bis zur Veröffentlichung nachvollzogen werden:

Beim \gls{bmas} findet sich tatsächlich eine nette Übersicht vom \enquote{Referenzentwurf} über den \enquote{Kabinettsbeschluss} bis zum \enquote{Abschluss des Gesetzes}.
Alle Schriftstücke sind, wie es sein sollte, verlinkt. Der Referenzentwurf vom \gls{bmas} ist dabei einfach brav als PDF anzutreffen \autocite{BMAS-21.07.2020}. 
Die Drucksache des Bundestages hingegen ist kopiergeschützt, wtf \autocite{Bundestag.31.08.2020}?
Das Bundesgesetzblatt ist zwar ein eher nur maschinenlesbarer Link, aber hat immerhin einen netten Viewer mit ziemlich guter Suchfunktion integriert und ist damit nicht kopiergeschützt \autocite{BGBl.2020-I-Nr67}. 
%%%%%%%%%%%%%%%%%%%%%%%%%%%%%%%%%%%%%%%%%%%%%%%%%%%%%%%%%%%%%%%%%%%%%%%%%%%%%%%%%%%%


%%%%%%%%%%%%%%%%%%%%%%%%%%%%%%%%      Zu ZIELGRUPPE      %%%%%%%%%%%%%%%%%%%%%%%%%%%%%%%%%%%%%%%%%%%%%%%%%%%%
% \enquote{Fremdzuschreibungen und defizitorientierte Ansprachen von Personengruppen sind deshalb problematisch, weil sie die in unserer Gesellschaft ungleich verteilten Zugänge zu u.a. Bildung und Chancen mitunter eher reproduzieren als sie – wie von einer inklusiven politischen Bildung} \autocite[]{Beckmann2022}
% Dennoch ist es wichtig, die Zielgruppe zu kennen und darauf einzugehen.

2024-12-11
Wie wird die Zielgruppe angesprochen?
Wird eine heterogene Zielgruppe angesprochen?
Bla über Defizitorientierung
% https://profession-politischebildung.de/grundlagen/grundbegriffe/defizitorientierung/#:~:text=Defizitorientierung%20meint%20die%20Fokussierung%20auf,Bildungsangeboten%20sowie%20im%20p%C3%A4dagogischen%20Handeln.
% die website zitiert z.B: bei Citavi Holzer 2010

% \paragraph{Schüler*innenvorstellungen}
% Nach dem Modell der didaktischen Rekonstruktion \autocite[]{Reinfried2009} ist die Inbezugnahme von Schüler*innenvorstellungen ein zentrales Element für besseren Unterricht.
% Es soll also untersucht werden, an welchen Stellen das Material Schüler*innenvorstellungen aufgreift oder immerhin Raum dafür lässt. 

% Die \gls{ank} als Gegenspieler zu wirtschaftsnahem Material? Engartner 2023:7
%%%%%%%%%%%%%%%%%%%%%%%%%%%%%%%%%%%%%%%%%%%%%%%%%%%%%%%%%%%%%%%%%%%%%%%%%%%%%%%%%%%%

%%%%%%%%%%%%%%%%%%%%%%%%%%%%%%%%%%%%%%%%%%%%%%%%%%%%%%%%%%%%%%%%%%%%%%%%%%%%%%%%%%%%
\paragraph{Politische Bildung in der Fachliteratur -- Ein Auszug}
Im Bremer Schulsystem werden in der Mittelstufe klassische Schulfächer wie Chemie, Physik und Biologie gemeinsam im Fach \gls{nw} \autocite{vogel2010nw} oder im sozialwissenschaftlichen Bereich wurden Geschichte, Politik und Geographie zu meiner Zeit an der Gesamtschule als \gls{wuk} \autocite{vogel2006gs} oder heutzutage an Oberschulen als \gls{gp} \autocite{vogel2010gp} unterrichtet.
% GEHT DIE ICH FORM??
->
Autor:innnen haben auch keinen Bock mehr auf monodisziplinär. 

Die \gls{bpb} hat den Sammelband veröffentlicht.
%%%%%%%%%%%%%%%%%%%%%%%%%%%%%%%%%%%%%%%%%%%%%%%%%%%%%%%%%%%%%%%%%%%%%%%%%%%%%%%%%%%%

%%%%%%%%%%%%%%%%%%%%%%%%%%%%%%%%%%%%%%%%%%%%%%%%%%%%%%%%%%%%%%%%%%%%%%%%%%%%%%%%%%%%
\subsection{Bildungsplan für berufsbildende Schulen oder allgemeine politische Bildung mit Zielgruppenorientierung?}
% Politik an berufsbildenden Schulen in Bremen wird neben Deutsch als einziges Fach für alle Fachrichtungen berufsübergreifend unterrichtet. 
Daher findet sich strukturell wenig Unterscheidung zu dem Fach Politik an allgemeinbildenden Schulen.
Der Bremer Bildungsplan für Politik in der Berufsbildung ist mit der Veröffentlichung 2023 auch wesentlich aktueller als die Bildungspläne welche Politik beinhalten von 2006, 2008 und 2010 für die allgemeinbildenden Schulen in Bremen. 
Mit der Arbeitnehmerkammer Bremen wird sich außerdem auf einen lokalen Akteur bezogen, welcher voraussichtlich auch für einen Großteil der Bremer Schüler:innen an allgemeinbildenden Schulen relevant werden könnte. 
Zumal die politische Bildung an Berufsschulen sich explizit nicht auf den beruflichen Bereich beschränken soll, sondern gesamtgesellschaftliche Zusammenhänge zum Thema hat. Genau wie an allgemeinbildenden Schulen und bei politischer Bildung im Allgemeinen.

% Politische Bildung ist (oder sollte) in zahlreichen Lebensbereichen verortet (sein). Daher sei auf das Offensichtliche hingewiesen. Es handelt sich hier um Unterrichtsmaterial, welches für die staatlich institutionalisierte Bildung eingesetzt werden soll. 
% Dieses Bildungsmaterial soll jedoch mit einem anderen staatlichen Akteur herausgegeben werden. 

Wie werden Demokratiekompetenzen gefördert? Wird Wissen über Demokratiekompetenzen gefördert oder gar direkt demokratisches oder eben undemokratisches Handeln erprobt und reflektiert?
%%%%%%%%%%%%%%%%%%%%%%%%%%%%%%%%%%%%%%%%%%%%%%%%%%%%%%%%%%%%%%%%%%%%%%%%%%%%%%%%%%%%

%%%%%%%%%%%%%%%%%%%%%%%%%%%%%%%%%%%%%%%%%%%%%%%%%%%%%%%%%%%%%%%%%%%%%%%%%%%%%%%%%%%%
\subsection{Was sagt die Wissenschaft zu politischer Bildung an (Berufs)schulen?}
Anja Besand

Reinhold Hedtke (den habe ich schon)

Bettina Zurstrassen (Herausgeberin Sammelband von der bpb) 

Christine Engartner
%%%%%%%%%%%%%%%%%%%%%%%%%%%%%%%%%%%%%%%%%%%%%%%%%%%%%%%%%%%%%%%%%%%%%%%%%%%%%%%%%%%%

%%%%%%%%%%%%%%%%%%%%%%%%%%%%%%%%%%   ARBEITSLEBEN B3 - M4    %%%%%%%%%%%%%%%%%%%%%%%%%%%%%%%%%%%%%%%%%%%%%
\textsc{Aufgabe 1} 1) Gruppe: Vollzeitbeschäftigung und Normalarbeitsverhältnis S. 1-2 \quad 
\begin{myitemize}
    \item Die Hälfte der Vollzeitbeschäftigten möchte kürzer arbeiten.
    \item Ein Viertel derTeilzeitbeschäftigte möchte länger arbeiten.
    \item Über ein Drittel denkt wegen Arbeitszeitwünschen über eine Jobwechsel nach.
    \item 
\end{myitemize}

\textsc{Aufgabe 1} 2) Gruppe: Wunsch nach Arbeitszeitverkürzung M4 S. 2-4 \quad ---

\textsc{Aufgabe 1} 3) Gruppe: Arbeitszeit- und Arbeitsverdichtung M4 S. 4-5 \quad ---


\textsc{Aufgabe 3} 1) \quad ---

\textsc{Aufgabe 3} 2) \quad ---

\textsc{Aufgabe 3} 3) \quad ---
%%%%%%%%%%%%%%%%%%%%%%%%%%%%%%%%%%%%%%%%%%%%%%%%%%%%%%%%%%%%%%%%%%%%%%%%%%%%%%%%%%%%%%%%%%%%%%%%%%%%

%%%%%%%%% eine schreckliche approximation %%%%%%%%%%%%%%%%%%%%%%%%%%%%%%%%%%%%%%%%%%%%%%%%%%%
\begin{align}
f(x) &\approx -3{,}1 \cdot 10^{-7} \cdot x^3 + 9 \cdot 10^{-4} \cdot x^2 - 0{,}146 \cdot x + 94 \\
     &\text{für } 0 < x < 1200
\end{align}
%%%%%%%%%%%%%%%%%%%%%%%%%%%%%%%%%%%%%%%%%%%%%%%%%%%%%%%%%%%%%%%%%%%%%%%%%%%%%%%%%%%
f (x) = x Df = {0 < x < 100} (1)
f (x) = 0, 2x + 80 Df = {100 < x < 520} (2)
f (x) = 0, 3x + 28 Df = {520 < x < 1000} (3)
f (x) = 0, 1x + 228 Df = {1000 < x < 1200} (4)


\addsec{Dokumentation der Nutzung von KI-basierten Anwendungen und Werkzeugen}
Die folgende Tabelle wurde in Anlehnung an die Vorlage der Universität Bremen erstellt:
\\

\footnotesize{
    \url{https://www.uni-bremen.de/zpa/formulare} führt zu: 
    \\

    \url{https://view.officeapps.live.com/op/view.aspx?src=https%3A%2F%2Fwww.uni-bremen.de%2Ffileadmin%2Fuser_upload%2Fsites%2Fzpa%2Fpdf%2Fallgemein%2FDokumentation_Nutzung_KI_-_AI_Use_Documentation.docx&wdOrigin=BROWSELINK} 
    \\
    
    beide 29.06.2025
    }

\clearpage
\newpage

\newgeometry{left=10mm, right=10mm, top=10mm, bottom=20mm} %%%

        \begin{landscape}
    \begin{table}
\centering 
\begin{tabularx}{\linewidth}{| >{\raggedright\arraybackslash} c || >{\raggedright\arraybackslash} X | >{\raggedright\arraybackslash} X | >{\raggedright\arraybackslash} X | >{\raggedright\arraybackslash} X | >{\raggedright\arraybackslash} X |} % {c|c|c|c|c|c} this was for tabular not tabularx

    \hline
                                & 
    KI-basiertes Hilfsmittel 
        
    AI-based Tool               & 
    Einsatzform 

    Purpose                     & 
    Betroffene Teile der Arbeit 
        
    Aspect of the Work Affected & 
    Beschreibung der Eingabe (Prompt) 
  
    Prompt (Entry)              & 
    Bemerkung 
    
    Comment                     \\ 
    \hline
    \hline
    %%%%%%%

    
    1                                                                                                                                           & 
    GitHub Copilot (Chat) in Visual Studio Code

    Inline Chat \& Chat on Secondary Sidebar

    mostly with GPT-4.1 as \gls{llm}                                                                                                            & 
    Fragen zu \LaTeX{} Code, keine inhaltlichen Fragen                                                                                          &
    alle, insbesondere \gls{zb} der Code zur Form dieser Tabelle oder Fragen zum Darstellen von eingebunden PDF-Seiten im Inhaltsverzeichnis    & 
    \gls{zb} \enquote{ia there \textbackslash{}vfill in latex?},  
    
    \enquote{what does \textbackslash{}arraybackslash in tabularx do?}, 
    
    \enquote{what can the parameter in lines 14 or 27 change?} oder
    
    \enquote{how do I make several entries to the toc like I treid in lines 26 to 62?}                                                          & 
    Es werden nicht alle Prompts aufgeführt. Es wurde ausschließlich für den Code Hintergrund benutzt und nie für den Inhalt. 
    
    Das Meiste wurde dann ohnehin doch wieder über Suchmaschinen und dann Foren und Anleitungen lesen erledigt. Häufig eben erst im Anschluss an das initiale Ausprobieren von \gls{ki} \\ 
    \hline
    %%%%%%%


    2                                                                                                                               &
    Visual Studio Code Autovervollständigung                                                                                        &
    Codevervollständigung und Autovervollständigung von einzelnen Worten, ähnlich dem Tippen auf Smartphones                        &
    alle                                                                                                                            &
    keine Prompts. 
    
    Bei der Eingabe \enquote{Effiz} wird dann \gls{zb} \enquote{Effizienz} vorgeschlagen und nach Druck auf Enter ausgeschrieben    &
    Die Grenze von maschinellem Lernen und Skripts oder überhaupt Software Hilfsmitteln ist fließend. Auch wenn schon länger existierende Autovervollständigungen für einzelne Worte keinen \gls{ki}-Boom und derartige gesellschaftliche Diskurse ausgelöst hat, wie es die \gls{llm} derzeit hervorrufen.                                             \\ 
    \hline

\pagebreak

    3                                               &
    ChatGPT.com                                     &
    Erstellen einer Graphik                         &
    Abbildung                                       &
    \enquote{Bitte plotte  die vier Funktionen unten in einer Abbildung. Beschrifte die x-Achse in hunderter Schritten. Mach keine Überschrift. Nenne die x-Achse anzurechnendes Einkommen in €. Nenne die y-Achse Absetzungsbetrag in €
    f (x) = x Df = {0 < x < 100} (1)
    f (x) = 0, 2x + 80 Df = {100 < x < 520} (2)
    f (x) = 0, 3x + 28 Df = {520 < x < 1000} (3)
    f (x) = 0, 1x + 228 Df = {1000 < x < 1200} (4)} &
    ---                                             \\
    \hline
    %%%%%%%

\end{tabularx}

\caption{Dokumentation der Nutzung von KI-basierten Anwendungen und Werkzeugen -- Documentation of the Use of AI-based Applications and Tools}

\label{KIHilfsmittel}

    \end{table}
        \end{landscape}

\restoregeometry %%%


% \footnote{\url{https://www.uni-bremen.de/zpa/formulare} führt zu \url{https://view.officeapps.live.com/op/view.aspx?src=https%3A%2F%2Fwww.uni-bremen.de%2Ffileadmin%2Fuser_upload%2Fsites%2Fzpa%2Fpdf%2Fallgemein%2FDokumentation_Nutzung_KI_-_AI_Use_Documentation.docx&wdOrigin=BROWSELINK} beide 29.06.2025}








\textsc{Aufgabe 2} a) \quad
ZU M1 VON DEMOKRATIE B3 - WAS GAR NICHT VERWENDET WIRD
\begin{myitemize}
    \item Mitbestimmung im Betrieb lässt Selbstwirksamkeit erfahren.
    \item Das kann sich auf die \enquote{parlametarische Demokratie} auswirken (\emph{spill-over Effekt})
    \item Menschen mit mehr Geld und Ressourcen wählen eher als Menschen mit wenig.
    \item Menschen ohen deutsche Staatsbürgherschasft dürfen auch überhaupt nicht wählen und arbeiten zusaätzlich häufig in prekäreren Verhältnissen. 
    \item Das ist für die demokratische Legitimation bedenklich.
    \item Lösungsansätze: 
        \begin{myitemize}
            \item Betriebliche Mitbestimmung stärken
            \item Schnellere Einbürgerungen 
            \item Repräsentation in den Parlamaenten von Frauen und Menschen mit Migrationsgeschichte ausbauen.
        \end{myitemize} 
\end{myitemize}


%%%%%%%%%%%%%%%%%%%%%%%%%%%%%%% KOMPETENZBLA %%%%%%%%%%%%%%%%%%%%%%%%%%%%%%%%%%%%%%%%%%%%%%%%%%%%%%%%%%%%%%%%%%%%%%%%%%%%%%%%%%%%%%%%%%%%%%%%%%%%%%%%%%%%%%%%%%%%
\subsubsection{Jonge, ich verliere mich in dem Komeptenzgebumsel; hier noch bla für weiter Schwafeln}
VIELLEICHT MAL EINE BRÜCKE ZU DEM EMOTIONS-BUMS SCHLAGEN? PASST HIER PRIMA REIN



Ein Politikunterricht, der nicht die politische Handlungsfähigkeit als Ziel hat, bleibt daher hinter seinem Potential zurück.



Eine Berufsschule ist zwar nicht mehr derart der Schulpflicht unterworfen, wie eine allgemeinbildende, aber die grundsätzlichen Strukturen, um welche es hier geht, sollten vergleichbar sein.

 \textcite[467-469]{Nonnenmacher2010}

Autorengruppe Fachdidaktik:
Wolfgang Sander, Sibylle Reinhardt, Andreas Petrik, Dirk Lange, Peter Henkenborg, Reinhold Hedtke, Tilman Grammes, Anja Besand
\autocite[]{Sander.2016}
digggaaaaahhh wohin gehört der bums?\autocite[]{Sander.2016}

%%%%%%%%%%%%%%%%%%%%%%%%%%%%%%%%%%%%%%%%%%%%%%%%%%%%%%%%%%%%%
ZITAT
 Wolfgang Hilligen hatte in der schon erwähnten Untersuchung von 1955 Zur Auflösung der Fußnote[7] seine didaktischen Grundsätze bereits skizziert und damit zur Formulierung der Hessischen Richtlinien von 1957 beigetragen; seit diesem Jahr gibt es von ihm eines der bekanntesten Schulbücher („Sehen-Beurteilen-Handeln“) für den politischen Unterricht. Seine didaktische Konzeption entstand nicht aus einer vorgängigen wissenschaftlich-systematischen Überlegung, sondern umgekehrt aus den Schwierigkeiten der Unterrichtspraxis selbst, für deren Lösung er nach einer verallgemeinerungsfähigen, d. h. auch für andere Lehrer in gleicher Lage nützlichen Theorie suchte. Weil er diese im Laufe der Zeit immer wieder modifizierte und präzisierte, ist sein Wirken bis in die achtziger Jahre hinein eine wichtige Quelle für das Studium der Schwierigkeiten, die angesichts fortschreitender politischer und wissenschaftlicher Veränderungen mit einem solchen Vorhaben verbunden sind.  (Giesecke 1999, S. 15)

\enquote{Sehen-Beurteilen-Handeln} auch schon in den 50er Jahren ein Ding, guck hier \autocite[15]{Giesecke.1999}

ZITAT
 3. Politisches und pädagogisches Handeln unter liegen unterschiedlichen Strategien und Erfolgskriterien. Das eine ist darauf aus, die Wirklichkeit zu verändern, das andere, sie im Rahmen geplanter Lehr- und Lernarrangements verständlich zu machen. Welche Schlußfolgerungen die Lernen den daraus ziehen, müssen sie selbst entscheiden. Insofern bleibt immer fraglich, ob Lehrziele auch tatsächlich zu Lernzielen werden. Die Didaktik kann von sich aus die Wirklichkeit nicht gestalten, über die sie aufklären will. \autocite[22]{Giesecke.1999}


Fischer Prinzip \autocite[]{Grammes.2005}

ZITAT
 Zum Verhandeln gehören neben argumentativen Strategien, die der konsensuellen Entscheidungsfindung dienen, auch Strategien, die durch den Einsatz von Machtpotenzialen, Konfliktfähigkeit, ökonomischen Ressourcen oder Tausch versuchen, Verhandlungsprozesse abzukürzen und zu hierarchisch autoritären oder hierarchisch majoritären Entscheidungen zu gelangen. Diese lassen sich im Unterricht zwar nicht erfahren, sollten aber gewusst werden. Entscheiden als Teil des realen partizipativen politischen Handelns lässt sich im Unterricht nur begrenzt fördern. Auch wenn Entscheiden durch kooperative oder handlungsorientierte Methoden geübt werden kann, bleibt es gegenüber der realen Politik unterkomplex. Allerdings lassen sich an konkreten politischen Fällen, Problemen, Konflikten oder Entscheidungsprozessen unterschiedliche Strategien und ihre Wirksamkeit analysieren, deren Ergebnisse den Lernenden dann als Fachwissen zur Verfügung stehen. (Massing 2012, S. 27)

JOOONGE, DAS IST AUCH BEI KAPITEL RELFEXION

bin bei gloe2020 s. 116 pdf118

% \subsubsection{Brudi, SOLL DAS VIELLEICHT NOCH IN DAS ELLENLANGE KOMPETENZKAPITEL? KA MAN}
% Massing meint Verhandeln mit Macht kann nicht erfahren werden. Was ist mit DSP diggi. \autocite[27]{Massing2012}
% Gefunden nach \autocite[111]{Gloe2020}
% \enquote{Die Strategien des Verhandelns, neben Argumentationsstrategien z. B. auch der Einsatz von Machtpotenzialen u. Ä., können in Lernprozessen »nicht erfahren, sollten aber gewusst werden« (Massing 2012: 27). Ebenso lässt sich die Kompetenzfacette Entscheiden nur begrenzt in Lernprozessen fördern: »Auch wenn Entscheiden durch kooperative oder handlungsorientierte Methoden geübt werden kann, bleibt es gegenüber der realen Politik unterkomplex« (ebd.).}

ZITAT
 Das Leitmotiv einer Pädagogik der Lernhilfe hat sicher den Vorteil einer pragmatischen Sicht auf den Lehrerberuf. Problematisch an dem Modell ist allerdings, dass es die Schule von den sozialen Problemen von Schülerinnen und Schülern in der heutigen Gesellschaft entkoppelt (Böhnisch/Schroer 2011). \autocite[51]{Sander.2016}

Kompetenzen können kritisiert werden, es gibt Ähnlichkeiten. \enquote{Sie leisten die Verbindung zwischen den jeweiligen Bildungszielen der Fächer und den Aufgabenstellungen des Unterrichts (Detjen et al. 2012, S. 20)} \autocite[18]{Massing.2022}

Ähnlich vernichtend wie \textcite[][]{Roler2016} den Beutelsbacher Konsens auseinanderbastelt, ließe sich der nur bemüht als praxisnah zu empfindenden Debatte um Kompetenzen gegenüberstehen. Allerdings 
%%%%%%%%%%%%%%%%%%%%%%%%%%%%%%%%%%%%%%%%%%%%%%%%%%%%%%%%%%%%%%%%%%%%%%%%%%%%%%%%%%%%%%%%%%%%%%%%%%%%%%%%%%%%%%%%%%%%%%%%%%%%%%%%%%%%%%%%%%%%%%%%%%%%%%%%%%%%%%%%%%%%%%%%%%%%%%%%%%%%%%%%%%%%%%%%%%%%%%%%%%%%%%%%%%%%%%%%%%%%%%%%%%%%%%%%%%%%%%%%%%%%%%%%%%%%%%%%%%%%%%%%%%%%%%%%%%%%%%%%%%







%%%%%%%%%%%%%%%%%%%%%%%%% TABELLENZEILE FÜR  ERWARTUNGSHORIZONT %%%%%%%%%%%%%%%%%%%%%%%%%%%%%%%%%%
     2 & 
    Copilot in Microsoft Edge & 
    Zusammenfassung von Texten & 
    Die Erwartungshorizonte.       
    \gls{ki}-Einsatz ist an den jeweiligen Stellen der Arbeit direkt angegeben &
    \enquote{fasse mir stichpunktartig die punkte des pdf zusammen} & 
    \\
    \hline
%%%%%%%%%%%%%%%%%%%%%%%%%%%%%%%%%%%%%%%%%%%%%%%%%%%%%%%%%%%%%%%%%%%%%%%%%%%%%%%%%%%%



%%%%%%%%%% DIESE Version MACHT FEHLER wegen der Fußnoten?? %%%%%%%%%%%%%%%%%%%%%%%%%
Beim \gls{bmas} findet sich tatsächlich eine nette Übersicht vom \enquote{Referenzentwurf} über den \enquote{Kabinettsbeschluss} bis zum \enquote{Abschluss des Gesetzes}\footnote{
    \url{https://www.bmas.de/DE/Service/Gesetze-und-Gesetzesvorhaben/arbeitsschutzkontrollgesetz.html} 19.06.2025
}. 
Alle Schriftstücke sind, wie es sein sollte, verlinkt. Der Referenzentwurf vom \gls{bmas}\footnote{
    \url{https://www.bmas.de/SharedDocs/Downloads/DE/Gesetze/Regierungsentwuerfe/reg-arbeitsschutzkontrollgesetz.pdf?__blob=publicationFile&v=3} 19.06.2025
} ist dabei einfach brav als PDF anzutreffen \autocite{BMAS-21.07.2020}. 
Die Drucksache\footnote{
    \url{https://dserver.bundestag.de/btd/19/219/1921978.pdf} 19.06.2025
} des Bundestages hingegen ist kopiergeschützt, wtf \autocite{Bundestag.31.08.2020}?
Das Bundesgesetzblatt ist zwar ein eher maschinenlesbarer Link\footnote{
    \url{https://www.bgbl.de/xaver/bgbl/start.xav?start=%2F%2F%2A%5B%40attr_id%3D%27bgbl120s3334.pdf%27%5D#/switch/tocPane?_ts=1750343088377} 19.06.2025
} aber hat immerhin einen netten Viewer mit ziemlich guter Suchfunktion integriert und ist damit nicht kopiergeschützt \autocite{BGBl.2020-I-Nr67}. 
%%%%%%%%%%%%%%%%%%%%%%%%%%%%%%%%%%%%%%%%%%%%%%%%%%%%%%%%%%%%%%%%%%%%%%%%%%%%%%%%%%%%%%%%%%%%%%%%%%%%%%%%%%%%%%
%%%%%%%%%%%%%%%% ZWEITE VERSION OHNE FUßNOTEN, die geht %%%%%%%%%%%%%%%%%%%%%%%%%%%%%%%%%%%%%%%%%%%%%%%%%
Beim \gls{bmas} findet sich tatsächlich eine nette Übersicht vom \enquote{Referenzentwurf} über den \enquote{Kabinettsbeschluss} bis zum \enquote{Abschluss des Gesetzes}.
Alle Schriftstücke sind, wie es sein sollte, verlinkt. Der Referenzentwurf vom \gls{bmas} ist dabei einfach brav als PDF anzutreffen \autocite{BMAS-21.07.2020}. 
Die Drucksache des Bundestages hingegen ist kopiergeschützt, wtf \autocite{Bundestag.31.08.2020}?
Das Bundesgesetzblatt ist zwar ein eher maschinenlesbarer Link, aber hat immerhin einen netten Viewer mit ziemlich guter Suchfunktion integriert und ist damit nicht kopiergeschützt \autocite{BGBl.2020-I-Nr67}. 
%%%%%%%%%%%%%%%%%%%%%%%%%%%%%%%%%%%%%%%%%%%%%%%%%%%%%%%%%%%%%%%%%%%%%%%%%%%%%%%%%%%%%%%%%%%%%%%%%%%%%%%%%



%%%%%%%%%%%%% FUNKTIONIERENDE VERSION %%%%%%%%%%%%%%%%%%%%%%%%%%%%%%%%%%%%%%%%%%%%%%%%%%%%%%%%%%%%%%%%%%%%%%%%%%%%
        \begin{landscape}
    \begin{table}
\centering 
\begin{tabularx}{\linewidth}{| X | X | X | X | X | X |} % {c|c|c|c|c|c} this was for tabulat without x

    \hline
     & 
    KI-basiertes Hilfsmittel 
        
    AI-based Tool & 
    Einsatzform 

    Purpose & 
    Betroffene Teile der Arbeit 
        
    Aspect of the Work Affected & 
    Beschreibung der Eingabe (Prompt) 
  
    Prompt (Entry) & 
    Bemerkung 
    
    Comment\\
    \hline
    \hline
    
    1 & 
    Copilot in Visual Studio Code
    
    mostly with GPT-4.1 as \gls{llm} & 
    Fragen zu \LaTeX Code, keine inhaltlichen Fragen &
    alle, insbesondere \gls{zb} der Code zur Form dieser Tabelle & 
    \gls{zb} \enquote{ia there \textbackslash{}vfill in latex?} & 
    Es werden nicht alle Prompts aufgeführt. Es wurde ausschließlich für den Code Hintergrund benutzt und nie für den Inhalt. 
    
    Das Meiste wurde dann ohnehin doch wieder über Suchmaschinen und Anleitungen lesen erledigt. Häufig eben im Anschluss an das Ausprobieren von \gls{ki} \\
    \hline


     2 & 
    Copilot in Microsoft Edge & 
    Zusammenfassung von Texten & 
    Die Erwartungshorizonte       
    An den jeweiligen Stellen der Arbeit direkt angegeben &
    \enquote{fasse mir stichpunktartig die punkte des pdf zusammen} & 
    \\

\end{tabularx}

\caption{Dokumentation der Nutzung von KI-basierten Anwendungen und Werkzeugen -- Documentation of the Use of AI-based Applications and Tools}

\label{KIHilfsmittel}

    \end{table}

        \end{landscape}
%%%%%%%%%%%%%%%%%%%%%%%%%%%%%%%%%%%%%%%%%%%%%%%%%%%%%%%%%%%%%%%%%%%%%%%%%%%%%%%%%%%%%%%%%%%%%%%%%%%%%%%%%%%%%%%%%%%%%%%%%%%%%%%%%%%%%%%



%%%%%%%%%% ALTE VERSION %%%%%%%%%%%%%%%%%%%%%%%%%%%%%%%%%%%%%%%%%%%%%%%%%%%%%%%%%%%%%%%%%%%%%%%%%%%%%%%
\begin{landscape}

\begin{table}
    \centering
    \begin{tabularx{\linewidth}{ 
        | >{\raggedright\arraybackslash}X 
        | >{\raggedright\arraybackslash}X 
        | >{\raggedright\arraybackslash}X 
        | >{\raggedright\arraybackslash}X 
        | >{\raggedright\arraybackslash}X 
        | >{\raggedright\arraybackslash}X | 
        }
        } % {c|c|c|c|c|c} this was for tabulat without x
        
        & 
         KI-basiertes Hilfsmittel AI-based Tool & 
         Einsatzform Purpose & 
         Betroffene Teile der Arbeit Aspect of the Work Affected & 
         Beschreibung der Eingabe (Prompt) Prompt (Entry) & 
         Bemerkung Comment\\
        \hline
        
        1 & 
        Copilot in Visual Studio Code
        
        mostly with GPT-4.1 as \gls{llm} & 
        alle &
        Fragen zu LaTeX Code, keine inhaltlichen Fragen & 
        \gls{zb} \enquote{ia there \textbackslash{}vfill in latex?} & 
        es werden nicht alle Prompts aufgeführt. Es wurde ausschließlich für den Code Hintergrund benutzt und nie für den Inhalt. Das Meiste wurde dennoch über Suchmaschinen und Anleitungen lesen erledigt. Meist auch noch im Anschluss an das Ausprobieren von \gls{ki} \\


        2 & 
        Copilot in Microsoft Edge & 
        Zusammenfassung von Texten, meist bei den Erwartungshorizonten. An den jeweiligen Stellen der Arbeit direkt angegeben & 
        \enquote{fasse mir stichpunktartig die punkte des pdf zusammen} & 
         & \\
    \end{tabularx}
    \caption{Dokumentation der Nutzung von KI-basierten Anwendungen und Werkzeugen -- Documentation of the Use of AI-based Applications and Tools}
    \label{KIHilfsmittel}
\end{table}

\end{landscape}
%%%%%%%%%%%%%%%%%%%%%%%%%%%%%%%%%%%%%%%%%%%%%%%%%%%%%%%%%%%%%%%%%%%%%%%%%%%%%%%%%%%%%%%%%%%%%%%%%%%%%%%%%%%%%%%
Allgemeine Entwicklungen
\begin{myitemize}
\item Rückgang der Berufskrankheiten-Anzeigen in Bremen 2017 um 20\% gegenüber 2016 – Ursache unklar.
\item 2017: 899 Verdachtsanzeigen, 401 Anerkennungen, 139 neue Renten, 100 Todesfälle.
\item Bremen hat im Vergleich zum Bund ein ungünstigeres Verhältnis von Anzeigen zu Todesfällen.
\end{myitemize}


Hauptursachen für Berufskrankheiten
\begin{myitemize}
\item Asbestbedingte Erkrankungen dominieren (z. B. Asbestose, Mesotheliom).
\item Weitere häufige Erkrankungen: Hautkrankheiten, Lärmschwerhörigkeit, Hautkrebs durch UV-Strahlung.
\item Asbest: Ursache für 94 von 100 Todesfällen 2017.
\end{myitemize}


Geschlechtsspezifische Unterschiede
\begin{myitemize}
\item Frauen: häufig betroffen durch Haut-, Infektions- und Wirbelsäulenerkrankungen in frauendominierten Berufen.
\item Männer: häufiger betroffen durch asbestbedingte Erkrankungen, Lärm, UV-Strahlung – typisch für Industrie und Bau.
\end{myitemize}


Anerkennungsverfahren & Hürden
\begin{myitemize}
\item Berufskrankheit = juristischer Begriff, nicht medizinisch.
\item Hohe Beweislast bei Betroffenen: Nachweis der beruflichen Ursache erforderlich.
\item Verfahren oft langwierig, Anerkennung schwierig, besonders bei fehlender Dokumentation oder Betriebsauflösung.
\item Manche Krankheiten (z.B. chronisches Ekzem) nur anerkennbar nach Tätigkeitsaufgabe.
\end{myitemize}


Beratung & Unterstützung
\begin{myitemize}
\item Beratungsstelle der Arbeitnehmerkammer Bremen bietet kostenlose Hilfe.
\item 2018: 260 Beratungen, v.a. zu Asbest, Muskel-Skelett-Erkrankungen, Lärm, Krebs, Asthma.
\item Fehlende unabhängige arbeitsmedizinische Expertise seit Wegfall des Landesgewerbearztes.
\end{myitemize}


Arbeitsschutz & Prävention
\begin{myitemize}
\item Schwerpunktaktion 2018 zu UV-Strahlung: große Defizite bei Prävention in Betrieben.
\item Nur 41 von 115 Betrieben hatten Gefährdungsbeurteilung zu UV-Strahlung.
\item Schutzmaßnahmen oft unzureichend oder von Beschäftigten selbst finanziert.
\end{myitemize}


Politische Forderungen & Handlungsbedarf
\begin{myitemize}
\item Anerkennungssystem muss reformiert werden: z.B. Beweiserleichterung bei Beweisnot.
\item Erweiterung der Berufskrankheiten-Liste notwendig (z.B. frauentypische Erkrankungen).
\item Bessere Ausstattung der Gewerbeaufsicht und Rückkehr eines Landesgewerbearztes gefordert.
\item Systematischer Arbeitsschutz und gezielte Prävention müssen gestärkt werden.
\end{myitemize}
%%%%%%%%%%%%%%%%%%%%%%%%%%%%%%%%%%%%%%%%%%%%%%%%%%%%%%%%%%%%%%%%%%%%%%%%%%%%%%%%%%%%%%%%%%%%%%%%%%%%%%%%%%%%%%%%%%%%%%%%%%%%%%%%
Aufgabe 1 - Die Lösung zu Aufgabe 1 ist von Copilot zusammengefasst und neu formatiert:

\textbf{Allgemeine Entwicklungen}
\begin{myitemize}
\item Rückgang der Berufskrankheiten-Anzeigen in Bremen 2017 um 20\% gegenüber 2016 – Ursache unklar.
\item 2017: 899 Verdachtsanzeigen, 401 Anerkennungen, 139 neue Renten, 100 Todesfälle.
\item Bremen hat im Vergleich zum Bund ein ungünstigeres Verhältnis von Anzeigen zu Todesfällen.
\end{myitemize}


\textbf{Hauptursachen für Berufskrankheiten}
\begin{myitemize}
\item Asbestbedingte Erkrankungen dominieren (z.B. Asbestose, Mesotheliom).
\item Weitere häufige Erkrankungen: Hautkrankheiten, Lärmschwerhörigkeit, Hautkrebs durch UV-Strahlung.
\item Asbest: Ursache für 94 von 100 Todesfällen 2017.
\end{myitemize}


\textbf{Geschlechtsspezifische Unterschiede}
\begin{myitemize}
\item Frauen: häufig betroffen durch Haut-, Infektions- und Wirbelsäulenerkrankungen in frauendominierten Berufen.
\item Männer: häufiger betroffen durch asbestbedingte Erkrankungen, Lärm, UV-Strahlung – typisch für Industrie und Bau.
\end{myitemize}


\textbf{Anerkennungsverfahren \& Hürden}
\begin{myitemize}
\item Berufskrankheit = juristischer Begriff, nicht medizinisch.
\item Hohe Beweislast bei Betroffenen: Nachweis der beruflichen Ursache erforderlich.
\item Verfahren oft langwierig, Anerkennung schwierig, besonders bei fehlender Dokumentation oder Betriebsauflösung.
\item Manche Krankheiten (z.B. chronisches Ekzem) nur anerkennbar nach Tätigkeitsaufgabe.
\end{myitemize}


\textbf{Beratung \& Unterstützung}
\begin{myitemize}
\item Beratungsstelle der Arbeitnehmerkammer Bremen bietet kostenlose Hilfe.
\item 2018: 260 Beratungen, v.a. zu Asbest, Muskel-Skelett-Erkrankungen, Lärm, Krebs, Asthma.
\item Fehlende unabhängige arbeitsmedizinische Expertise seit Wegfall des Landesgewerbearztes.
\end{myitemize}


\textbf{Arbeitsschutz \& Prävention}
\begin{myitemize}
\item Schwerpunktaktion 2018 zu UV-Strahlung: große Defizite bei Prävention in Betrieben.
\item Nur 41 von 115 Betrieben hatten Gefährdungsbeurteilung zu UV-Strahlung.
\item Schutzmaßnahmen oft unzureichend oder von Beschäftigten selbst finanziert.
\end{myitemize}


\textbf{Politische Forderungen \& Handlungsbedarf}
\begin{myitemize}
\item Anerkennungssystem muss reformiert werden: z.B. Beweiserleichterung bei Beweisnot.
\item Erweiterung der Berufskrankheiten-Liste notwendig (z.B. frauentypische Erkrankungen).
\item Bessere Ausstattung der Gewerbeaufsicht und Rückkehr eines Landesgewerbearztes gefordert.
\item Systematischer Arbeitsschutz und gezielte Prävention müssen gestärkt werden.
\end{myitemize}
%%%%%%%%%%%%%%%%%%%%%%%%%%%%%%%%%%%%%%%%%%%%%%%%%%%%%%%%%%%%%%%%%%%%%%%


%%----------------------------------------------%%
\usepackage[]{minitoc}
\setcounter{secttocdepth}{4}
\setlength{\stcindent}{24pt}

\dosecttoc
\dosectlof
\dosectlot

%%----------------------------------------------%%
\secttoc
%%----------------------------------------------%%



%%%%%%%%%%%%%%%%%%%    BBK    %%%%%%%%%%%%%%%%%%%%%%%%%
\begin{quote}
    \textbf{1. Überwältigungsverbot.}

    Es ist nicht erlaubt, den Schüler - mit welchen Mitteln auch immer - im Sinne erwünschter Meinungen zu überrumpeln und damit an der \enquote{Gewinnung eines selbständigen Urteils} zu hindern. Hier genau verläuft nämlich die Grenze zwischen Politischer Bildung und Indoktrination. Indoktrination aber ist unvereinbar mit der Rolle des Lehrers in einer demokratischen Gesellschaft und der - rundum akzeptierten - Zielvorstellung von der Mündigkeit des Schülers. 

    \textbf{2. Was in Wissenschaft und Politik kontrovers ist, muss auch im Unterricht kontrovers erscheinen.}

    Diese Forderung ist mit der vorgenannten aufs engste verknüpft, denn wenn unterschiedliche Standpunkte unter den Tisch fallen, Optionen unterschlagen werden, Alternativen unerörtert bleiben, ist der Weg zur Indoktrination beschritten. Zu fragen ist, ob der Lehrer nicht sogar eine Korrekturfunktion haben sollte, d.h. ob er nicht solche Standpunkte und Alternativen besonders herausarbeiten muss, die den Schülern (und anderen Teilnehmern politischer Bildungsveranstaltungen) von ihrer jeweiligen politischen und sozialen Herkunft her fremd sind. Bei der Konstatierung dieses zweiten Grundprinzips wird deutlich, warum der persönliche Standpunkt des Lehrers, seine wissenschaftstheoretische Herkunft und seine politische Meinung verhältnismäßig uninteressant werden. Um ein bereits genanntes Beispiel erneut aufzugreifen: Sein Demokratieverständnis stellt kein Problem dar, denn auch dem entgegenstehende andere Ansichten kommen ja zum Zuge. 
    
    \textbf{3. Der Schüler muss in die Lage versetzt werden, eine politische Situation und seine eigene Interessenlage zu analysieren,} 

    sowie nach Mitteln und Wegen zu suchen, die vorgefundene politische Lage im Sinne seiner Interessen zu beeinflussen. Eine solche Zielsetzung schließt in sehr starkem Maße die Betonung operationaler Fähigkeiten ein, was eine logische Konsequenz aus den beiden vorgenannten Prinzipien ist. Der in diesem Zusammenhang gelegentlich - etwa gegen Herman Giesecke und Rolf Schmiederer - erhobene Vorwurf einer \enquote{Rückkehr zur Formalität}, um die eigenen Inhalte nicht korrigieren zu müssen, trifft insofern nicht, als es hier nicht um die Suche nach einem Maximal-, sondern nach einem Minimalkonsens geht. 
    
    \autocite[Im Original mit anderen Hervorhebungen:][179-180]{Wehling1977}
\end{quote}
%%%%%%%%%%%%%%%%%%%%%

%%%%%%%%%%%%%%%%%%%%%%%%%%%%%%%%%%%%%%     WIRTSCHAFTSPOLITIK IM APENDIX ALS PDF       %%%%%%%%%%%%%%%%%
\includepdf[
    nup=2x1, 
    addtotoc={
    1, subsection, 2, Wirtschaftspolitik A1, WIRTSCHAFTSPOLITIK-A1, 
    3, subsection, 2, Wirtschaftspolitik A2, WIRTSCHAFTSPOLITIK-A2, 
    5, subsection, 2, Wirtschaftspolitik A3, WIRTSCHAFTSPOLITIK-A3, 
    7, subsection, 2, Wirtschaftspolitik B1, WIRTSCHAFTSPOLITIK-B1, 
    9, subsection, 2, Wirtschaftspolitik B2, WIRTSCHAFTSPOLITIK-B2, 
    11, subsection, 2, Wirtschaftspolitik C1, WIRTSCHAFTSPOLITIK-C1, 
    13, subsection, 2, Wirtschaftspolitik C2, WIRTSCHAFTSPOLITIK-C2
    }
]
{WIRTSCHAFTSPOLITIK.pdf} 
%%%%%%%%%%%%%%%%%%%%%%%%